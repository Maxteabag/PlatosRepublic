\section{Argument}

The argument of the Republic is the search after Justice, the nature of which is first hinted at by Cephalus, the just and blameless old man --then discussed on the basis of proverbial morality by Socrates and Polemarchus --then caricatured by Thrasymachus and partially explained by Socrates --reduced to an abstraction by Glaucon and Adeimantus, and having become invisible in the individual reappears at length in the ideal State which is constructed by Socrates. The first care of the rulers is to be education, of which an outline is drawn after the old Hellenic model, providing only for an improved religion and morality, and more simplicity in music and gymnastic, a manlier strain of poetry, and greater harmony of the individual and the State. We are thus led on to the conception of a higher State, in which "no man calls anything his own," and in which there is neither "marrying nor giving in marriage," and "kings are philosophers" and "philosophers are kings;" and there is another and higher education, intellectual as well as moral and religious, of science as well as of art, and not of youth only but of the whole of life. Such a State is hardly to be realized in this world and would quickly degenerate. To the perfect ideal succeeds the government of the soldier and the lover of honor, this again declining into democracy, and democracy into tyranny, in an imaginary but regular order having not much resemblance to the actual facts. When "the wheel has come full circle" we do not begin again with a new period of human life; but we have passed from the best to the worst, and there we end. The subject is then changed and the old quarrel of poetry and philosophy which had been more lightly treated in the earlier books of the Republic is now resumed and fought out to a conclusion. Poetry is discovered to be an imitation thrice removed from the truth, and Homer, as well as the dramatic poets, having been condemned as an imitator, is sent into banishment along with them. And the idea of the State is supplemented by the revelation of a future life.

The division into books, like all similar divisions, is probably later than the age of Plato. The natural divisions are five in number; --(1) Book I and the first half of Book II down to the paragraph beginning, "I had always admired the genius of Glaucon and Adeimantus," which is introductory; the first book containing a refutation of the popular and sophistical notions of justice, and concluding, like some of the earlier Dialogues, without arriving at any definite result. To this is appended a restatement of the nature of justice according to common opinion, and an answer is demanded to the question --What is justice, stripped of appearances? The second division (2) includes the remainder of the second and the whole of the third and fourth books, which are mainly occupied with the construction of the first State and the first education. The third division (3) consists of the fifth, sixth, and seventh books, in which philosophy rather than justice is the subject of inquiry, and the second State is constructed on principles of communism and ruled by philosophers, and the contemplation of the idea of good takes the place of the social and political virtues. In the eighth and ninth books (4) the perversions of States and of the individuals who correspond to them are reviewed in succession; and the nature of pleasure and the principle of tyranny are further analyzed in the individual man. The tenth book (5) is the conclusion of the whole, in which the relations of philosophy to poetry are finally determined, and the happiness of the citizens in this life, which has now been assured, is crowned by the vision of another.

Or a more general division into two parts may be adopted; the first (Books I - IV) containing the description of a State framed generally in accordance with Hellenic notions of religion and morality, while in the second (Books V - X) the Hellenic State is transformed into an ideal kingdom of philosophy, of which all other governments are the perversions. These two points of view are really opposed, and the opposition is only veiled by the genius of Plato. The Republic, like the Phaedrus, is an imperfect whole; the higher light of philosophy breaks through the regularity of the Hellenic temple, which at last fades away into the heavens. Whether this imperfection of structure arises from an enlargement of the plan; or from the imperfect reconcilement in the writer's own mind of the struggling elements of thought which are now first brought together by him; or, perhaps, from the composition of the work at different times --are questions, like the similar question about the Iliad and the Odyssey, which are worth asking, but which cannot have a distinct answer. In the age of Plato there was no regular mode of publication, and an author would have the less scruple in altering or adding to a work which was known only to a few of his friends. There is no absurdity in supposing that he may have laid his labors aside for a time, or turned from one work to another; and such interruptions would be more likely to occur in the case of a long than of a short writing. In all attempts to determine the chronological he order of the Platonic writings on internal evidence, this uncertainty about any single Dialogue being composed at one time is a disturbing element, which must be admitted to affect longer works, such as the Republic and the Laws, more than shorter ones. But, on the other hand, the seeming discrepancies of the Republic may only arise out of the discordant elements which the philosopher has attempted to unite in a single whole, perhaps without being himself able to recognize the inconsistency which is obvious to us. For there is a judgment of after ages which few great writers have ever been able to anticipate for themselves. They do not perceive the want of connection in their own writings, or the gaps in their systems which are visible enough to those who come after them. In the beginnings of literature and philosophy, amid the first efforts of thought and language, more inconsistencies occur than now, when the paths of speculation are well worn and the meaning of words precisely defined. For consistency, too, is the growth of time; and some of the greatest creations of the human mind have been wanting in unity. Tried by this test, several of the Platonic Dialogues, according to our modern ideas, appear to be defective, but the deficiency is no proof that they were composed at different times or by different hands. And the supposition that the Republic was written uninterruptedly and by a continuous effort is in some degree confirmed by the numerous references from one part of the work to another.

The second title, "Concerning Justice," is not the one by which the Republic is quoted, either by Aristotle or generally in antiquity, and, like the other second titles of the Platonic Dialogues, may therefore be assumed to be of later date. Morgenstern and others have asked whether the definition of justice, which is the professed aim, or the construction of the State is the principal argument of the work. The answer is, that the two blend in one, and are two faces of the same truth; for justice is the order of the State, and the State is the visible embodiment of justice under the conditions of human society. The one is the soul and the other is the body, and the Greek ideal of the State, as of the individual, is a fair mind in a fair body. In Hegelian phraseology the State is the reality of which justice is the ideal. Or, described in Christian language, the kingdom of God is within, and yet develops into a Church or external kingdom; "the house not made with hands, eternal in the heavens," is reduced to the proportions of an earthly building. Or, to use a Platonic image, justice and the State are the warp and the woof which run through the whole texture. And when the constitution of the State is completed, the conception of justice is not dismissed, but reappears under the same or different names throughout the work, both as the inner law of the individual soul, and finally as the principle of rewards and punishments in another life. The virtues are based on justice, of which common honesty in buying and selling is the shadow, and justice is based on the idea of good, which is the harmony of the world, and is reflected both in the institutions of States and in motions of the heavenly bodies. The Timaeus, which takes up the political rather than the ethical side of the Republic, and is chiefly occupied with hypotheses concerning the outward world, yet contains many indications that the same law is supposed to reign over the State, over nature, and over man.

Too much, however, has been made of this question both in ancient and in modern times. There is a stage of criticism in which all works, whether of nature or of art, are referred to design. Now in ancient writings, and indeed in literature generally, there remains often a large element which was not comprehended in the original design. For the plan grows under the author's hand; new thoughts occur to him in the act of writing; he has not worked out the argument to the end before he begins. The reader who seeks to find some one idea under which the whole may be conceived, must necessarily seize on the vaguest and most general. Thus Stallbaum, who is dissatisfied with the ordinary explanations of the argument of the Republic, imagines himself to have found the true argument "in the representation of human life in a State perfected by justice and governed according to the idea of good." There may be some use in such general descriptions, but they can hardly be said to express the design of the writer. The truth is, that we may as well speak of many designs as of one; nor need anything be excluded from the plan of a great work to which the mind is naturally led by the association of ideas, and which does not interfere with the general purpose. What kind or degree of unity is to be sought after in a building, in the plastic arts, in poetry, in prose, is a problem which has to be determined relatively to the subject-matter. To Plato himself, the inquiry "what was the intention of the writer," or "what was the principal argument of the Republic" would have been hardly intelligible, and therefore had better be at once dismissed.

Is not the Republic the vehicle of three or four great truths which, to Plato's own mind, are most naturally represented in the form of the State? Just as in the Jewish prophets the reign of Messiah, or "the day of the Lord," or the suffering Servant or people of God, or the "Sun of righteousness with healing in his wings" only convey, to us at least, their great spiritual ideals, so through the Greek State Plato reveals to us his own thoughts about divine perfection, which is the idea of good --like the sun in the visible world; --about human perfection, which is justice --about education beginning in youth and continuing in later years --about poets and sophists and tyrants who are the false teachers and evil rulers of mankind --about "the world" which is the embodiment of them --about a kingdom which exists nowhere upon earth but is laid up in heaven to be the pattern and rule of human life. No such inspired creation is at unity with itself, any more than the clouds of heaven when the sun pierces through them. Every shade of light and dark, of truth, and of fiction which is the veil of truth, is allowable in a work of philosophical imagination. It is not all on the same plane; it easily passes from ideas to myths and fancies, from facts to figures of speech. It is not prose but poetry, at least a great part of it, and ought not to be judged by the rules of logic or the probabilities of history. The writer is not fashioning his ideas into an artistic whole; they take possession of him and are too much for him. We have no need therefore to discuss whether a State such as Plato has conceived is practicable or not, or whether the outward form or the inward life came first into the mind of the writer. For the practicability of his ideas has nothing to do with their truth; and the highest thoughts to which he attains may be truly said to bear the greatest "marks of design" --justice more than the external frame-work of the State, the idea of good more than justice. The great science of dialectic or the organization of ideas has no real content; but is only a type of the method or spirit in which the higher knowledge is to be pursued by the spectator of all time and all existence. It is in the fifth, sixth, and seventh books that Plato reaches the "summit of speculation," and these, although they fail to satisfy the requirements of a modern thinker, may therefore be regarded as the most important, as they are also the most original, portions of the work.

It is not necessary to discuss at length a minor question which has been raised by Boeckh, respecting the imaginary date at which the conversation was held (the year 411 B. C. which is proposed by him will do as well as any other); for a writer of fiction, and especially a writer who, like Plato, is notoriously careless of chronology, only aims at general probability. Whether all the persons mentioned in the Republic could ever have met at any one time is not a difficulty which would have occurred to an Athenian reading the work forty years later, or to Plato himself at the time of writing (any more than to Shakespeare respecting one of his own dramas); and need not greatly trouble us now. Yet this may be a question having no answer "which is still worth asking," because the investigation shows that we can not argue historically from the dates in Plato; it would be useless therefore to waste time in inventing far-fetched reconcilements of them in order avoid chronological difficulties, such, for example, as the conjecture of C. F. Hermann, that Glaucon and Adeimantus are not the brothers but the uncles of Plato, or the fancy of Stallbaum that Plato intentionally left anachronisms indicating the dates at which some of his Dialogues were written.

