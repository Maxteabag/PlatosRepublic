\section{Characters}

The principal characters in the Republic are Cephalus, Polemarchus, Thrasymachus, Socrates, Glaucon, and Adeimantus. Cephalus appears in the introduction only, Polemarchus drops at the end of the first argument, and Thrasymachus is reduced to silence at the close of the first book. The main discussion is carried on by Socrates, Glaucon, and Adeimantus. Among the company are Lysias (the orator) and Euthydemus, the sons of Cephalus and brothers of Polemarchus, an unknown Charmantides --these are mute auditors; also there is Cleitophon, who once interrupts, where, as in the Dialogue which bears his name, he appears as the friend and ally of Thrasymachus.

Cephalus, the patriarch of house, has been appropriately engaged in offering a sacrifice. He is the pattern of an old man who has almost done with life, and is at peace with himself and with all mankind. He feels that he is drawing nearer to the world below, and seems to linger around the memory of the past. He is eager that Socrates should come to visit him, fond of the poetry of the last generation, happy in the consciousness of a well-spent life, glad at having escaped from the tyranny of youthful lusts. His love of conversation, his affection, his indifference to riches, even his garrulity, are interesting traits of character. He is not one of those who have nothing to say, because their whole mind has been absorbed in making money. Yet he acknowledges that riches have the advantage of placing men above the temptation to dishonesty or falsehood. The respectful attention shown to him by Socrates, whose love of conversation, no less than the mission imposed upon him by the Oracle, leads him to ask questions of all men, young and old alike, should also be noted. Who better suited to raise the question of justice than Cephalus, whose life might seem to be the expression of it? The moderation with which old age is pictured by Cephalus as a very tolerable portion of existence is characteristic, not only of him, but of Greek feeling generally, and contrasts with the exaggeration of Cicero in the De Senectute. The evening of life is described by Plato in the most expressive manner, yet with the fewest possible touches. As Cicero remarks (Ep. ad Attic. iv. 16), the aged Cephalus would have been out of place in the discussion which follows, and which he could neither have understood nor taken part in without a violation of dramatic propriety.

His "son and heir" Polemarchus has the frankness and impetuousness of youth; he is for detaining Socrates by force in the opening scene, and will not "let him off" on the subject of women and children. Like Cephalus, he is limited in his point of view, and represents the proverbial stage of morality which has rules of life rather than principles; and he quotes Simonides as his father had quoted Pindar. But after this he has no more to say; the answers which he makes are only elicited from him by the dialectic of Socrates. He has not yet experienced the influence of the Sophists like Glaucon and Adeimantus, nor is he sensible of the necessity of refuting them; he belongs to the pre-Socratic or pre-dialectical age. He is incapable of arguing, and is bewildered by Socrates to such a degree that he does not know what he is saying. He is made to admit that justice is a thief, and that the virtues follow the analogy of the arts. From his brother Lysias we learn that he fell a victim to the Thirty Tyrants, but no allusion is here made to his fate, nor to the circumstance that Cephalus and his family were of Syracusan origin, and had migrated from Thurii to Athens.

The "Chalcedonian giant," Thrasymachus, of whom we have already heard in the Phaedrus, is the personification of the Sophists, according to Plato's conception of them, in some of their worst characteristics. He is vain and blustering, refusing to discourse unless he is paid, fond of making an oration, and hoping thereby to escape the inevitable Socrates; but a mere child in argument, and unable to foresee that the next "move" (to use a Platonic expression) will "shut him up." He has reached the stage of framing general notions, and in this respect is in advance of Cephalus and Polemarchus. But he is incapable of defending them in a discussion, and vainly tries to cover his confusion in banter and insolence. Whether such doctrines as are attributed to him by Plato were really held either by him or by any other Sophist is uncertain; in the infancy of philosophy serious errors about morality might easily grow up --they are certainly put into the mouths of speakers in Thucydides; but we are concerned at present with Plato's description of him, and not with the historical reality. The inequality of the contest adds greatly to the humor of the scene. The pompous and empty Sophist is utterly helpless in the hands of the great master of dialectic, who knows how to touch all the springs of vanity and weakness in him. He is greatly irritated by the irony of Socrates, but his noisy and imbecile rage only lays him more and more open to the thrusts of his assailant. His determination to cram down their throats, or put "bodily into their souls" his own words, elicits a cry of horror from Socrates. The state of his temper is quite as worthy of remark as the process of the argument. Nothing is more amusing than his complete submission when he has been once thoroughly beaten. At first he seems to continue the discussion with reluctance, but soon with apparent good-will, and he even testifies his interest at a later stage by one or two occasional remarks. When attacked by Glaucon he is humorously protected by Socrates "as one who has never been his enemy and is now his friend." From Cicero and Quintilian and from Aristotle's Rhetoric we learn that the Sophist whom Plato has made so ridiculous was a man of note whose writings were preserved in later ages. The play on his name which was made by his contemporary Herodicus, "thou wast ever bold in battle," seems to show that the description of him is not devoid of verisimilitude.

The character of Adeimantus is graver and deeper, and the more profound objections are commonly put into his mouth. Glaucon is more demonstrative, and generally opens the game. Adeimantus pursues the argument further. Glaucon has more of the liveliness and quick sympathy of youth; Adeimantus has the maturer judgment of a grown-up man of the world. In the second book, when Glaucon insists that justice and injustice shall be considered without regard to their consequences, Adeimantus remarks that they are regarded by mankind in general only for the sake of their consequences; and in a similar vein of reflection he urges at the beginning of the fourth book that Socrates falls in making his citizens happy, and is answered that happiness is not the first but the second thing, not the direct aim but the indirect consequence of the good government of a State. In the discussion about religion and mythology, Adeimantus is the respondent, but Glaucon breaks in with a slight jest, and carries on the conversation in a lighter tone about music and gymnastic to the end of the book. It is Adeimantus again who volunteers the criticism of common sense on the Socratic method of argument, and who refuses to let Socrates pass lightly over the question of women and children. It is Adeimantus who is the respondent in the more argumentative, as Glaucon in the lighter and more imaginative portions of the Dialogue. For example, throughout the greater part of the sixth book, the causes of the corruption of philosophy and the conception of the idea of good are discussed with Adeimantus. Then Glaucon resumes his place of principal respondent; but he has a difficulty in apprehending the higher education of Socrates, and makes some false hits in the course of the discussion. Once more Adeimantus returns with the allusion to his brother Glaucon whom he compares to the contentious State; in the next book he is again superseded, and Glaucon continues to the end.

Thus in a succession of characters Plato represents the successive stages of morality, beginning with the Athenian gentleman of the olden time, who is followed by the practical man of that day regulating his life by proverbs and saws; to him succeeds the wild generalization of the Sophists, and lastly come the young disciples of the great teacher, who know the sophistical arguments but will not be convinced by them, and desire to go deeper into the nature of things. These too, like Cephalus, Polemarchus, Thrasymachus, are clearly distinguished from one another. Neither in the Republic, nor in any other Dialogue of Plato, is a single character repeated.

The delineation of Socrates in the Republic is not wholly consistent. In the first book we have more of the real Socrates, such as he is depicted in the Memorabilia of Xenophon, in the earliest Dialogues of Plato, and in the Apology. He is ironical, provoking, questioning, the old enemy of the Sophists, ready to put on the mask of Silenus as well as to argue seriously. But in the sixth book his enmity towards the Sophists abates; he acknowledges that they are the representatives rather than the corrupters of the world. He also becomes more dogmatic and constructive, passing beyond the range either of the political or the speculative ideas of the real Socrates. In one passage Plato himself seems to mimic that the time had now come for Socrates, who had passed his whole life in philosophy, to give his own opinion and not to be always repeating the notions of other men. There is no evidence that either the idea of good or the conception of a perfect State were comprehended in the Socratic teaching, though he certainly dwelt on the nature of the universal and of final causes (cp. Xen. Mem. i. 4; Phaedo 97); and a deep thinker like him in his thirty or forty years of public teaching, could hardly have falled to touch on the nature of family relations, for which there is also some positive evidence in the Memorabilia (Mem. i. 2, 51 foll.) The Socratic method is nominally retained; and every inference is either put into the mouth of the respondent or represented as the common discovery of him and Socrates. But any one can see that this is a mere form, of which the affectation grows wearisome as the work advances. The method of inquiry has passed into a method of teaching in which by the help of interlocutors the same thesis is looked at from various points of view.

The nature of the process is truly characterized by Glaucon, when he describes himself as a companion who is not good for much in an investigation, but can see what he is shown, and may, perhaps, give the answer to a question more fluently than another.

Neither can we be absolutely certain that, Socrates himself taught the immortality of the soul, which is unknown to his disciple Glaucon in the Republic; nor is there any reason to suppose that he used myths or revelations of another world as a vehicle of instruction, or that he would have banished poetry or have denounced the Greek mythology. His favorite oath is retained, and a slight mention is made of the daemonium, or internal sign, which is alluded to by Socrates as a phenomenon peculiar to himself. A real element of Socratic teaching, which is more prominent in the Republic than in any of the other Dialogues of Plato, is the use of example and illustration ('taphorhtika auto prhospherhontez'): "Let us apply the test of common instances." "You," says Adeimantus, ironically, in the sixth book, "are so unaccustomed to speak in images." And this use of examples or images, though truly Socratic in origin, is enlarged by the genius of Plato into the form of an allegory or parable, which embodies in the concrete what has been already described, or is about to be described, in the abstract. Thus the figure of the cave in Book VII is a recapitulation of the divisions of knowledge in Book VI. The composite animal in Book IX is an allegory of the parts of the soul. The noble captain and the ship and the true pilot in Book VI are a figure of the relation of the people to the philosophers in the State which has been described.

Plato is most true to the character of his master when he describes him as "not of this world." And with this representation of him the ideal State and the other paradoxes of the Republic are quite in accordance, though they can not be shown to have been speculations of Socrates. To him, as to other great teachers both philosophical and religious, when they looked upward, the world seemed to be the embodiment of error and evil. The common sense of mankind has revolted against this view, or has only partially admitted it. And even in Socrates himself the sterner judgment of the multitude at times passes into a sort of ironical pity or love. Men in general are incapable of philosophy, and are therefore at enmity with the philosopher; but their misunderstanding of him is unavoidable: for they have never seen him as he truly is in his own image; they are only acquainted with artificial systems possessing no native force of truth --words which admit of many applications. Their leaders have nothing to measure with, and are therefore ignorant of their own stature. But they are to be pitied or laughed at, not to be quarrelled with; they mean well with their nostrums, if they could only learn that they are cutting off a Hydra's head. This moderation towards those who are in error is one of the most characteristic features of Socrates in the Republic. In all the different representations of Socrates, whether of Xenophon or Plato, and the differences of the earlier or later Dialogues, he always retains the character of the unwearied and disinterested seeker after truth, without which he would have ceased to be Socrates.

Leaving the characters we may now analyze the contents of the Republic, and then proceed to consider (1) The general aspects of this Hellenic ideal of the State, (2) The modern lights in which the thoughts of Plato may be read.
