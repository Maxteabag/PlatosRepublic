\chapter{Book VI}

Socrates - GLAUCON

And thus, Glaucon, after the argument has gone a weary way, the true and the false philosophers have at length appeared in view.

I do not think, he said, that the way could have been shortened.
I suppose not, I said; and yet I believe that we might have had a better view of both of them if the discussion could have been confined to this one subject and if there were not many other questions awaiting us, which he who desires to see in what respect the life of the just differs from that of the unjust must consider.

And what is the next question? he asked.
Surely, I said, the one which follows next in order. Inasmuch as philosophers only are able to grasp the eternal and unchangeable, and those who wander in the region of the many and variable are not philosophers, I must ask you which of the two classes should be the rulers of our State?

And how can we rightly answer that question?
Whichever of the two are best able to guard the laws and institutions of our State --let them be our guardians.

Very good.
Neither, I said, can there be any question that the guardian who is to keep anything should have eyes rather than no eyes?

There can be no question of that.
And are not those who are verily and indeed wanting in the knowledge of the true being of each thing, and who have in their souls no clear pattern, and are unable as with a painter's eye to look at the absolute truth and to that original to repair, and having perfect vision of the other world to order the laws about beauty, goodness, justice in this, if not already ordered, and to guard and preserve the order of them --are not such persons, I ask, simply blind?

Truly, he replied, they are much in that condition.
And shall they be our guardians when there are others who, besides being their equals in experience and falling short of them in no particular of virtue, also know the very truth of each thing?

There can be no reason, he said, for rejecting those who have this greatest of all great qualities; they must always have the first place unless they fail in some other respect.

Suppose then, I said, that we determine how far they can unite this and the other excellences.

By all means.
In the first place, as we began by observing, the nature of the philosopher has to be ascertained. We must come to an understanding about him, and, when we have done so, then, if I am not mistaken, we shall also acknowledge that such an union of qualities is possible, and that those in whom they are united, and those only, should be rulers in the State.

What do you mean?
Let us suppose that philosophical minds always love knowledge of a sort which shows them the eternal nature not varying from generation and corruption.

Agreed.
And further, I said, let us agree that they are lovers of all true being; there is no part whether greater or less, or more or less honourable, which they are willing to renounce; as we said before of the lover and the man of ambition.

True.
And if they are to be what we were describing, is there not another quality which they should also possess?

What quality?
Truthfulness: they will never intentionally receive into their mind falsehood, which is their detestation, and they will love the truth.

Yes, that may be safely affirmed of them.
'May be,' my friend, I replied, is not the word; say rather 'must be affirmed:' for he whose nature is amorous of anything cannot help loving all that belongs or is akin to the object of his affections.

Right, he said.
And is there anything more akin to wisdom than truth?
How can there be?
Can the same nature be a lover of wisdom and a lover of falsehood?
Never.
The true lover of learning then must from his earliest youth, as far as in him lies, desire all truth?

Assuredly.
But then again, as we know by experience, he whose desires are strong in one direction will have them weaker in others; they will be like a stream which has been drawn off into another channel.

True.
He whose desires are drawn towards knowledge in every form will be absorbed in the pleasures of the soul, and will hardly feel bodily pleasure --I mean, if he be a true philosopher and not a sham one.

That is most certain.
Such an one is sure to be temperate and the reverse of covetous; for the motives which make another man desirous of having and spending, have no place in his character.

Very true.
Another criterion of the philosophical nature has also to be considered.

What is that?
There should be no secret corner of illiberality; nothing can more antagonistic than meanness to a soul which is ever longing after the whole of things both divine and human.

Most true, he replied.
Then how can he who has magnificence of mind and is the spectator of all time and all existence, think much of human life?

He cannot.
Or can such an one account death fearful?
No indeed.
Then the cowardly and mean nature has no part in true philosophy?
Certainly not.
Or again: can he who is harmoniously constituted, who is not covetous or mean, or a boaster, or a coward-can he, I say, ever be unjust or hard in his dealings?

Impossible.
Then you will soon observe whether a man is just and gentle, or rude and unsociable; these are the signs which distinguish even in youth the philosophical nature from the unphilosophical.

True.
There is another point which should be remarked.
What point?
Whether he has or has not a pleasure in learning; for no one will love that which gives him pain, and in which after much toil he makes little progress.

Certainly not.
And again, if he is forgetful and retains nothing of what he learns, will he not be an empty vessel?

That is certain.
Labouring in vain, he must end in hating himself and his fruitless occupation? Yes.

Then a soul which forgets cannot be ranked among genuine philosophic natures; we must insist that the philosopher should have a good memory?

Certainly.
And once more, the inharmonious and unseemly nature can only tend to disproportion?

Undoubtedly.
And do you consider truth to be akin to proportion or to disproportion?

To proportion.
Then, besides other qualities, we must try to find a naturally well-proportioned and gracious mind, which will move spontaneously towards the true being of everything.

Certainly.
Well, and do not all these qualities, which we have been enumerating, go together, and are they not, in a manner, necessary to a soul, which is to have a full and perfect participation of being?

They are absolutely necessary, he replied.
And must not that be a blameless study which he only can pursue who has the gift of a good memory, and is quick to learn, --noble, gracious, the friend of truth, justice, courage, temperance, who are his kindred?

The god of jealousy himself, he said, could find no fault with such a study.

And to men like him, I said, when perfected by years and education, and to these only you will entrust the State.

Socrates - ADEIMANTUS

Here Adeimantus interposed and said: To these statements, Socrates, no one can offer a reply; but when you talk in this way, a strange feeling passes over the minds of your hearers: They fancy that they are led astray a little at each step in the argument, owing to their own want of skill in asking and answering questions; these littles accumulate, and at the end of the discussion they are found to have sustained a mighty overthrow and all their former notions appear to be turned upside down. And as unskilful players of draughts are at last shut up by their more skilful adversaries and have no piece to move, so they too find themselves shut up at last; for they have nothing to say in this new game of which words are the counters; and yet all the time they are in the right. The observation is suggested to me by what is now occurring. For any one of us might say, that although in words he is not able to meet you at each step of the argument, he sees as a fact that the votaries of philosophy, when they carry on the study, not only in youth as a part of education, but as the pursuit of their maturer years, most of them become strange monsters, not to say utter rogues, and that those who may be considered the best of them are made useless to the world by the very study which you extol.

Well, and do you think that those who say so are wrong?
I cannot tell, he replied; but I should like to know what is your opinion.

Hear my answer; I am of opinion that they are quite right.
Then how can you be justified in saying that cities will not cease from evil until philosophers rule in them, when philosophers are acknowledged by us to be of no use to them?

You ask a question, I said, to which a reply can only be given in a parable.

Yes, Socrates; and that is a way of speaking to which you are not at all accustomed, I suppose.

I perceive, I said, that you are vastly amused at having plunged me into such a hopeless discussion; but now hear the parable, and then you will be still more amused at the meagreness of my imagination: for the manner in which the best men are treated in their own States is so grievous that no single thing on earth is comparable to it; and therefore, if I am to plead their cause, I must have recourse to fiction, and put together a figure made up of many things, like the fabulous unions of goats and stags which are found in pictures. Imagine then a fleet or a ship in which there is a captain who is taller and stronger than any of the crew, but he is a little deaf and has a similar infirmity in sight, and his knowledge of navigation is not much better. The sailors are quarrelling with one another about the steering --every one is of opinion that he has a right to steer, though he has never learned the art of navigation and cannot tell who taught him or when he learned, and will further assert that it cannot be taught, and they are ready to cut in pieces any one who says the contrary. They throng about the captain, begging and praying him to commit the helm to them; and if at any time they do not prevail, but others are preferred to them, they kill the others or throw them overboard, and having first chained up the noble captain's senses with drink or some narcotic drug, they mutiny and take possession of the ship and make free with the stores; thus, eating and drinking, they proceed on their voyage in such a manner as might be expected of them. Him who is their partisan and cleverly aids them in their plot for getting the ship out of the captain's hands into their own whether by force or persuasion, they compliment with the name of sailor, pilot, able seaman, and abuse the other sort of man, whom they call a good-for-nothing; but that the true pilot must pay attention to the year and seasons and sky and stars and winds, and whatever else belongs to his art, if he intends to be really qualified for the command of a ship, and that he must and will be the steerer, whether other people like or not-the possibility of this union of authority with the steerer's art has never seriously entered into their thoughts or been made part of their calling. Now in vessels which are in a state of mutiny and by sailors who are mutineers, how will the true pilot be regarded? Will he not be called by them a prater, a star-gazer, a good-for-nothing?

Of course, said Adeimantus.
Then you will hardly need, I said, to hear the interpretation of the figure, which describes the true philosopher in his relation to the State; for you understand already.

Certainly.
Then suppose you now take this parable to the gentleman who is surprised at finding that philosophers have no honour in their cities; explain it to him and try to convince him that their having honour would be far more extraordinary.

I will.
Say to him, that, in deeming the best votaries of philosophy to be useless to the rest of the world, he is right; but also tell him to attribute their uselessness to the fault of those who will not use them, and not to themselves. The pilot should not humbly beg the sailors to be commanded by him --that is not the order of nature; neither are 'the wise to go to the doors of the rich' --the ingenious author of this saying told a lie --but the truth is, that, when a man is ill, whether he be rich or poor, to the physician he must go, and he who wants to be governed, to him who is able to govern. The ruler who is good for anything ought not to beg his subjects to be ruled by him; although the present governors of mankind are of a different stamp; they may be justly compared to the mutinous sailors, and the true helmsmen to those who are called by them good-for-nothings and star-gazers.

Precisely so, he said.
For these reasons, and among men like these, philosophy, the noblest pursuit of all, is not likely to be much esteemed by those of the opposite faction; not that the greatest and most lasting injury is done to her by her opponents, but by her own professing followers, the same of whom you suppose the accuser to say, that the greater number of them are arrant rogues, and the best are useless; in which opinion I agreed.

Yes.
And the reason why the good are useless has now been explained?
True.
Then shall we proceed to show that the corruption of the majority is also unavoidable, and that this is not to be laid to the charge of philosophy any more than the other?

By all means.
And let us ask and answer in turn, first going back to the description of the gentle and noble nature. Truth, as you will remember, was his leader, whom he followed always and in all things; failing in this, he was an impostor, and had no part or lot in true philosophy.

Yes, that was said.
Well, and is not this one quality, to mention no others, greatly at variance with present notions of him?

Certainly, he said.
And have we not a right to say in his defence, that the true lover of knowledge is always striving after being --that is his nature; he will not rest in the multiplicity of individuals which is an appearance only, but will go on --the keen edge will not be blunted, nor the force of his desire abate until he have attained the knowledge of the true nature of every essence by a sympathetic and kindred power in the soul, and by that power drawing near and mingling and becoming incorporate with very being, having begotten mind and truth, he will have knowledge and will live and grow truly, and then, and not till then, will he cease from his travail.

Nothing, he said, can be more just than such a description of him.
And will the love of a lie be any part of a philosopher's nature? Will he not utterly hate a lie?

He will.
And when truth is the captain, we cannot suspect any evil of the band which he leads?

Impossible.
Justice and health of mind will be of the company, and temperance will follow after?

True, he replied.
Neither is there any reason why I should again set in array the philosopher's virtues, as you will doubtless remember that courage, magnificence, apprehension, memory, were his natural gifts. And you objected that, although no one could deny what I then said, still, if you leave words and look at facts, the persons who are thus described are some of them manifestly useless, and the greater number utterly depraved; we were then led to enquire into the grounds of these accusations, and have now arrived at the point of asking why are the majority bad, which question of necessity brought us back to the examination and definition of the true philosopher.

Exactly.
And we have next to consider the of the philosophic nature, why so many are spoiled and so few escape spoiling --I am speaking of those who were said to be useless but not wicked --and, when we have done with them, we will speak of the imitators of philosophy, what manner of men are they who aspire after a profession which is above them and of which they are unworthy, and then, by their manifold inconsistencies, bring upon philosophy, and upon all philosophers, that universal reprobation of which we speak.

What are these corruptions? he said.
I will see if I can explain them to you. Every one will admit that a nature having in perfection all the qualities which we required in a philosopher, is a rare plant which is seldom seen among men.

Rare indeed.
And what numberless and powerful causes tend to destroy these rare natures!

What causes?
In the first place there are their own virtues, their courage, temperance, and the rest of them, every one of which praise worthy qualities (and this is a most singular circumstance) destroys and distracts from philosophy the soul which is the possessor of them.

That is very singular, he replied.
Then there are all the ordinary goods of life --beauty, wealth, strength, rank, and great connections in the State --you understand the sort of things --these also have a corrupting and distracting effect.

I understand; but I should like to know more precisely what you mean about them.

Grasp the truth as a whole, I said, and in the right way; you will then have no difficulty in apprehending the preceding remarks, and they will no longer appear strange to you.

And how am I to do so? he asked.
Why, I said, we know that all germs or seeds, whether vegetable or animal, when they fail to meet with proper nutriment or climate or soil, in proportion to their vigour, are all the more sensitive to the want of a suitable environment, for evil is a greater enemy to what is good than what is not.

Very true.
There is reason in supposing that the finest natures, when under alien conditions, receive more injury than the inferior, because the contrast is greater.

Certainly.
And may we not say, Adeimantus, that the most gifted minds, when they are ill-educated, become pre-eminently bad? Do not great crimes and the spirit of pure evil spring out of a fulness of nature ruined by education rather than from any inferiority, whereas weak natures are scarcely capable of any very great good or very great evil?

There I think that you are right.
And our philosopher follows the same analogy-he is like a plant which, having proper nurture, must necessarily grow and mature into all virtue, but, if sown and planted in an alien soil, becomes the most noxious of all weeds, unless he be preserved by some divine power. Do you really think, as people so often say, that our youth are corrupted by Sophists, or that private teachers of the art corrupt them in any degree worth speaking of? Are not the public who say these things the greatest of all Sophists? And do they not educate to perfection young and old, men and women alike, and fashion them after their own hearts?

When is this accomplished? he said.
When they meet together, and the world sits down at an assembly, or in a court of law, or a theatre, or a camp, or in any other popular resort, and there is a great uproar, and they praise some things which are being said or done, and blame other things, equally exaggerating both, shouting and clapping their hands, and the echo of the rocks and the place in which they are assembled redoubles the sound of the praise or blame --at such a time will not a young man's heart, as they say, leap within him? Will any private training enable him to stand firm against the overwhelming flood of popular opinion? or will he be carried away by the stream? Will he not have the notions of good and evil which the public in general have --he will do as they do, and as they are, such will he be?

Yes, Socrates; necessity will compel him.
And yet, I said, there is a still greater necessity, which has not been mentioned.

What is that?
The gentle force of attainder or confiscation or death which, as you are aware, these new Sophists and educators who are the public, apply when their words are powerless.

Indeed they do; and in right good earnest.
Now what opinion of any other Sophist, or of any private person, can be expected to overcome in such an unequal contest?

None, he replied.
No, indeed, I said, even to make the attempt is a great piece of folly; there neither is, nor has been, nor is ever likely to be, any different type of character which has had no other training in virtue but that which is supplied by public opinion --I speak, my friend, of human virtue only; what is more than human, as the proverb says, is not included: for I would not have you ignorant that, in the present evil state of governments, whatever is saved and comes to good is saved by the power of God, as we may truly say.

I quite assent, he replied.
Then let me crave your assent also to a further observation.
What are you going to say?
Why, that all those mercenary individuals, whom the many call Sophists and whom they deem to be their adversaries, do, in fact, teach nothing but the opinion of the many, that is to say, the opinions of their assemblies; and this is their wisdom. I might compare them to a man who should study the tempers and desires of a mighty strong beast who is fed by him-he would learn how to approach and handle him, also at what times and from what causes he is dangerous or the reverse, and what is the meaning of his several cries, and by what sounds, when another utters them, he is soothed or infuriated; and you may suppose further, that when, by continually attending upon him, he has become perfect in all this, he calls his knowledge wisdom, and makes of it a system or art, which he proceeds to teach, although he has no real notion of what he means by the principles or passions of which he is speaking, but calls this honourable and that dishonourable, or good or evil, or just or unjust, all in accordance with the tastes and tempers of the great brute. Good he pronounces to be that in which the beast delights and evil to be that which he dislikes; and he can give no other account of them except that the just and noble are the necessary, having never himself seen, and having no power of explaining to others the nature of either, or the difference between them, which is immense. By heaven, would not such an one be a rare educator?

Indeed, he would.
And in what way does he who thinks that wisdom is the discernment of the tempers and tastes of the motley multitude, whether in painting or music, or, finally, in politics, differ from him whom I have been describing For when a man consorts with the many, and exhibits to them his poem or other work of art or the service which he has done the State, making them his judges when he is not obliged, the so-called necessity of Diomede will oblige him to produce whatever they praise. And yet the reasons are utterly ludicrous which they give in confirmation of their own notions about the honourable and good. Did you ever hear any of them which were not?

No, nor am I likely to hear.
You recognise the truth of what I have been saying? Then let me ask you to consider further whether the world will ever be induced to believe in the existence of absolute beauty rather than of the many beautiful, or of the absolute in each kind rather than of the many in each kind?

Certainly not.
Then the world cannot possibly be a philosopher?
Impossible.
And therefore philosophers must inevitably fall under the censure of the world?

They must.
And of individuals who consort with the mob and seek to please them?
That is evident.
Then, do you see any way in which the philosopher can be preserved in his calling to the end? and remember what we were saying of him, that he was to have quickness and memory and courage and magnificence --these were admitted by us to be the true philosopher's gifts.

Yes.
Will not such an one from his early childhood be in all things first among all, especially if his bodily endowments are like his mental ones?

Certainly, he said.
And his friends and fellow-citizens will want to use him as he gets older for their own purposes?

No question.
Falling at his feet, they will make requests to him and do him honour and flatter him, because they want to get into their hands now, the power which he will one day possess.

That often happens, he said.
And what will a man such as he be likely to do under such circumstances, especially if he be a citizen of a great city, rich and noble, and a tall proper youth? Will he not be full of boundless aspirations, and fancy himself able to manage the affairs of Hellenes and of barbarians, and having got such notions into his head will he not dilate and elevate himself in the fulness of vain pomp and senseless pride?

To be sure he will.
Now, when he is in this state of mind, if some one gently comes to him and tells him that he is a fool and must get understanding, which can only be got by slaving for it, do you think that, under such adverse circumstances, he will be easily induced to listen?

Far otherwise.
And even if there be some one who through inherent goodness or natural reasonableness has had his eyes opened a little and is humbled and taken captive by philosophy, how will his friends behave when they think that they are likely to lose the advantage which they were hoping to reap from his companionship? Will they not do and say anything to prevent him from yielding to his better nature and to render his teacher powerless, using to this end private intrigues as well as public prosecutions?

There can be no doubt of it.
And how can one who is thus circumstanced ever become a philosopher?
Impossible.
Then were we not right in saying that even the very qualities which make a man a philosopher may, if he be ill-educated, divert him from philosophy, no less than riches and their accompaniments and the other so-called goods of life?

We were quite right.
Thus, my excellent friend, is brought about all that ruin and failure which I have been describing of the natures best adapted to the best of all pursuits; they are natures which we maintain to be rare at any time; this being the class out of which come the men who are the authors of the greatest evil to States and individuals; and also of the greatest good when the tide carries them in that direction; but a small man never was the doer of any great thing either to individuals or to States.

That is most true, he said.
And so philosophy is left desolate, with her marriage rite incomplete: for her own have fallen away and forsaken her, and while they are leading a false and unbecoming life, other unworthy persons, seeing that she has no kinsmen to be her protectors, enter in and dishonour her; and fasten upon her the reproaches which, as you say, her reprovers utter, who affirm of her votaries that some are good for nothing, and that the greater number deserve the severest punishment.

That is certainly what people say.
Yes; and what else would you expect, I said, when you think of the puny creatures who, seeing this land open to them --a land well stocked with fair names and showy titles --like prisoners running out of prison into a sanctuary, take a leap out of their trades into philosophy; those who do so being probably the cleverest hands at their own miserable crafts? For, although philosophy be in this evil case, still there remains a dignity about her which is not to be found in the arts. And many are thus attracted by her whose natures are imperfect and whose souls are maimed and disfigured by their meannesses, as their bodies are by their trades and crafts. Is not this unavoidable?

Yes.
Are they not exactly like a bald little tinker who has just got out of durance and come into a fortune; he takes a bath and puts on a new coat, and is decked out as a bridegroom going to marry his master's daughter, who is left poor and desolate?

A most exact parallel.
What will be the issue of such marriages? Will they not be vile and bastard?

There can be no question of it.
And when persons who are unworthy of education approach philosophy and make an alliance with her who is a rank above them what sort of ideas and opinions are likely to be generated? Will they not be sophisms captivating to the ear, having nothing in them genuine, or worthy of or akin to true wisdom?

No doubt, he said.
Then, Adeimantus, I said, the worthy disciples of philosophy will be but a small remnant: perchance some noble and well-educated person, detained by exile in her service, who in the absence of corrupting influences remains devoted to her; or some lofty soul born in a mean city, the politics of which he contemns and neglects; and there may be a gifted few who leave the arts, which they justly despise, and come to her; --or peradventure there are some who are restrained by our friend Theages' bridle; for everything in the life of Theages conspired to divert him from philosophy; but ill-health kept him away from politics. My own case of the internal sign is hardly worth mentioning, for rarely, if ever, has such a monitor been given to any other man. Those who belong to this small class have tasted how sweet and blessed a possession philosophy is, and have also seen enough of the madness of the multitude; and they know that no politician is honest, nor is there any champion of justice at whose side they may fight and be saved. Such an one may be compared to a man who has fallen among wild beasts --he will not join in the wickedness of his fellows, but neither is he able singly to resist all their fierce natures, and therefore seeing that he would be of no use to the State or to his friends, and reflecting that he would have to throw away his life without doing any good either to himself or others, he holds his peace, and goes his own way. He is like one who, in the storm of dust and sleet which the driving wind hurries along, retires under the shelter of a wall; and seeing the rest of mankind full of wickedness, he is content, if only he can live his own life and be pure from evil or unrighteousness, and depart in peace and good-will, with bright hopes.

Yes, he said, and he will have done a great work before he departs.
A great work --yes; but not the greatest, unless he find a State suitable to him; for in a State which is suitable to him, he will have a larger growth and be the saviour of his country, as well as of himself.

The causes why philosophy is in such an evil name have now been sufficiently explained: the injustice of the charges against her has been shown-is there anything more which you wish to say?

Nothing more on that subject, he replied; but I should like to know which of the governments now existing is in your opinion the one adapted to her.

Not any of them, I said; and that is precisely the accusation which I bring against them --not one of them is worthy of the philosophic nature, and hence that nature is warped and estranged; --as the exotic seed which is sown in a foreign land becomes denaturalized, and is wont to be overpowered and to lose itself in the new soil, even so this growth of philosophy, instead of persisting, degenerates and receives another character. But if philosophy ever finds in the State that perfection which she herself is, then will be seen that she is in truth divine, and that all other things, whether natures of men or institutions, are but human; --and now, I know that you are going to ask, what that State is.

No, he said; there you are wrong, for I was going to ask another question --whether it is the State of which. we are the founders and inventors, or some other?

Yes, I replied, ours in most respects; but you may remember my saying before, that some living authority would always be required in the State having the same idea of the constitution which guided you when as legislator you were laying down the laws.

That was said, he replied.
Yes, but not in a satisfactory manner; you frightened us by interposing objections, which certainly showed that the discussion would be long and difficult; and what still remains is the reverse of easy.

What is there remaining?
The question how the study of philosophy may be so ordered as not to be the ruin of the State: All great attempts are attended with risk; 'hard is the good,' as men say.

Still, he said, let the point be cleared up, and the enquiry will then be complete.

I shall not be hindered, I said, by any want of will, but, if at all, by a want of power: my zeal you may see for yourselves; and please to remark in what I am about to say how boldly and unhesitatingly I declare that States should pursue philosophy, not as they do now, but in a different spirit.

In what manner?
At present, I said, the students of philosophy are quite young; beginning when they are hardly past childhood, they devote only the time saved from moneymaking and housekeeping to such pursuits; and even those of them who are reputed to have most of the philosophic spirit, when they come within sight of the great difficulty of the subject, I mean dialectic, take themselves off. In after life when invited by some one else, they may, perhaps, go and hear a lecture, and about this they make much ado, for philosophy is not considered by them to be their proper business: at last, when they grow old, in most cases they are extinguished more truly than Heracleitus' sun, inasmuch as they never light up again.

But what ought to be their course?
Just the opposite. In childhood and youth their study, and what philosophy they learn, should be suited to their tender years: during this period while they are growing up towards manhood, the chief and special care should be given to their bodies that they may have them to use in the service of philosophy; as life advances and the intellect begins to mature, let them increase the gymnastics of the soul; but when the strength of our citizens fails and is past civil and military duties, then let them range at will and engage in no serious labour, as we intend them to live happily here, and to crown this life with a similar happiness in another.

How truly in earnest you are, Socrates! he said; I am sure of that; and yet most of your hearers, if I am not mistaken, are likely to be still more earnest in their opposition to you, and will never be convinced; Thrasymachus least of all.

Do not make a quarrel, I said, between Thrasymachus and me, who have recently become friends, although, indeed, we were never enemies; for I shall go on striving to the utmost until I either convert him and other men, or do something which may profit them against the day when they live again, and hold the like discourse in another state of existence.

You are speaking of a time which is not very near.
Rather, I replied, of a time which is as nothing in comparison with eternity. Nevertheless, I do not wonder that the many refuse to believe; for they have never seen that of which we are now speaking realised; they have seen only a conventional imitation of philosophy, consisting of words artificially brought together, not like these of ours having a natural unity. But a human being who in word and work is perfectly moulded, as far as he can be, into the proportion and likeness of virtue --such a man ruling in a city which bears the same image, they have never yet seen, neither one nor many of them --do you think that they ever did?

No indeed.
No, my friend, and they have seldom, if ever, heard free and noble sentiments; such as men utter when they are earnestly and by every means in their power seeking after truth for the sake of knowledge, while they look coldly on the subtleties of controversy, of which the end is opinion and strife, whether they meet with them in the courts of law or in society.

They are strangers, he said, to the words of which you speak.
And this was what we foresaw, and this was the reason why truth forced us to admit, not without fear and hesitation, that neither cities nor States nor individuals will ever attain perfection until the small class of philosophers whom we termed useless but not corrupt are providentially compelled, whether they will or not, to take care of the State, and until a like necessity be laid on the State to obey them; or until kings, or if not kings, the sons of kings or princes, are divinely inspired ' d with a true love of true philosophy. That either or both of these alternatives are impossible, I see no reason to affirm: if they were so, we might indeed be justly ridiculed as dreamers and visionaries. Am I not right?

Quite right.
If then, in the countless ages of the past, or at the present hour in some foreign clime which is far away and beyond our ken, the perfected philosopher is or has been or hereafter shall be compelled by a superior power to have the charge of the State, we are ready to assert to the death, that this our constitution has been, and is --yea, and will be whenever the Muse of Philosophy is queen. There is no impossibility in all this; that there is a difficulty, we acknowledge ourselves.

My opinion agrees with yours, he said.
But do you mean to say that this is not the opinion of the multitude?

I should imagine not, he replied.
O my friend, I said, do not attack the multitude: they will change their minds, if, not in an aggressive spirit, but gently and with the view of soothing them and removing their dislike of over-education, you show them your philosophers as they really are and describe as you were just now doing their character and profession, and then mankind will see that he of whom you are speaking is not such as they supposed --if they view him in this new light, they will surely change their notion of him, and answer in another strain. Who can be at enmity with one who loves them, who that is himself gentle and free from envy will be jealous of one in whom there is no jealousy? Nay, let me answer for you, that in a few this harsh temper may be found but not in the majority of mankind.

I quite agree with you, he said.
And do you not also think, as I do, that the harsh feeling which the many entertain towards philosophy originates in the pretenders, who rush in uninvited, and are always abusing them, and finding fault with them, who make persons instead of things the theme of their conversation? and nothing can be more unbecoming in philosophers than this.

It is most unbecoming.
For he, Adeimantus, whose mind is fixed upon true being, has surely no time to look down upon the affairs of earth, or to be filled with malice and envy, contending against men; his eye is ever directed towards things fixed and immutable, which he sees neither injuring nor injured by one another, but all in order moving according to reason; these he imitates, and to these he will, as far as he can, conform himself. Can a man help imitating that with which he holds reverential converse?

Impossible.
And the philosopher holding converse with the divine order, becomes orderly and divine, as far as the nature of man allows; but like every one else, he will suffer from detraction.

Of course.
And if a necessity be laid upon him of fashioning, not only himself, but human nature generally, whether in States or individuals, into that which he beholds elsewhere, will he, think you, be an unskilful artificer of justice, temperance, and every civil virtue?

Anything but unskilful.
And if the world perceives that what we are saying about him is the truth, will they be angry with philosophy? Will they disbelieve us, when we tell them that no State can be happy which is not designed by artists who imitate the heavenly pattern?

They will not be angry if they understand, he said. But how will they draw out the plan of which you are speaking?

They will begin by taking the State and the manners of men, from which, as from a tablet, they will rub out the picture, and leave a clean surface. This is no easy task. But whether easy or not, herein will lie the difference between them and every other legislator, --they will have nothing to do either with individual or State, and will inscribe no laws, until they have either found, or themselves made, a clean surface.

They will be very right, he said.
Having effected this, they will proceed to trace an outline of the constitution?

No doubt.
And when they are filling in the work, as I conceive, they will often turn their eyes upwards and downwards: I mean that they will first look at absolute justice and beauty and temperance, and again at the human copy; and will mingle and temper the various elements of life into the image of a man; and thus they will conceive according to that other image, which, when existing among men, Homer calls the form and likeness of God.

Very true, he said.
And one feature they will erase, and another they will put in, they have made the ways of men, as far as possible, agreeable to the ways of God?

Indeed, he said, in no way could they make a fairer picture.
And now, I said, are we beginning to persuade those whom you described as rushing at us with might and main, that the painter of constitutions is such an one as we are praising; at whom they were so very indignant because to his hands we committed the State; and are they growing a little calmer at what they have just heard?

Much calmer, if there is any sense in them.
Why, where can they still find any ground for objection? Will they doubt that the philosopher is a lover of truth and being?

They would not be so unreasonable.
Or that his nature, being such as we have delineated, is akin to the highest good?

Neither can they doubt this.
But again, will they tell us that such a nature, placed under favourable circumstances, will not be perfectly good and wise if any ever was? Or will they prefer those whom we have rejected?

Surely not.
Then will they still be angry at our saying, that, until philosophers bear rule, States and individuals will have no rest from evil, nor will this our imaginary State ever be realised?

I think that they will be less angry.
Shall we assume that they are not only less angry but quite gentle, and that they have been converted and for very shame, if for no other reason, cannot refuse to come to terms?

By all means, he said.
Then let us suppose that the reconciliation has been effected. Will any one deny the other point, that there may be sons of kings or princes who are by nature philosophers?

Surely no man, he said.
And when they have come into being will any one say that they must of necessity be destroyed; that they can hardly be saved is not denied even by us; but that in the whole course of ages no single one of them can escape --who will venture to affirm this?

Who indeed!
But, said I, one is enough; let there be one man who has a city obedient to his will, and he might bring into existence the ideal polity about which the world is so incredulous.

Yes, one is enough.
The ruler may impose the laws and institutions which we have been describing, and the citizens may possibly be willing to obey them?

Certainly.
And that others should approve of what we approve, is no miracle or impossibility?

I think not.
But we have sufficiently shown, in what has preceded, that all this, if only possible, is assuredly for the best.

We have.
And now we say not only that our laws, if they could be enacted, would be for the best, but also that the enactment of them, though difficult, is not impossible.

Very good.
And so with pain and toil we have reached the end of one subject, but more remains to be discussed; --how and by what studies and pursuits will the saviours of the constitution be created, and at what ages are they to apply themselves to their several studies?

Certainly.
I omitted the troublesome business of the possession of women, and the procreation of children, and the appointment of the rulers, because I knew that the perfect State would be eyed with jealousy and was difficult of attainment; but that piece of cleverness was not of much service to me, for I had to discuss them all the same. The women and children are now disposed of, but the other question of the rulers must be investigated from the very beginning. We were saying, as you will remember, that they were to be lovers of their country, tried by the test of pleasures and pains, and neither in hardships, nor in dangers, nor at any other critical moment were to lose their patriotism --he was to be rejected who failed, but he who always came forth pure, like gold tried in the refiner's fire, was to be made a ruler, and to receive honours and rewards in life and after death. This was the sort of thing which was being said, and then the argument turned aside and veiled her face; not liking to stir the question which has now arisen.

I perfectly remember, he said.
Yes, my friend, I said, and I then shrank from hazarding the bold word; but now let me dare to say --that the perfect guardian must be a philosopher.

Yes, he said, let that be affirmed.
And do not suppose that there will be many of them; for the gifts which were deemed by us to be essential rarely grow together; they are mostly found in shreds and patches.

What do you mean? he said.
You are aware, I replied, that quick intelligence, memory, sagacity, cleverness, and similar qualities, do not often grow together, and that persons who possess them and are at the same time high-spirited and magnanimous are not so constituted by nature as to live orderly and in a peaceful and settled manner; they are driven any way by their impulses, and all solid principle goes out of them.

Very true, he said.
On the other hand, those steadfast natures which can better be depended upon, which in a battle are impregnable to fear and immovable, are equally immovable when there is anything to be learned; they are always in a torpid state, and are apt to yawn and go to sleep over any intellectual toil.

Quite true.
And yet we were saying that both qualities were necessary in those to whom the higher education is to be imparted, and who are to share in any office or command.

Certainly, he said.
And will they be a class which is rarely found?
Yes, indeed.
Then the aspirant must not only be tested in those labours and dangers and pleasures which we mentioned before, but there is another kind of probation which we did not mention --he must be exercised also in many kinds of knowledge, to see whether the soul will be able to endure the highest of all, will faint under them, as in any other studies and exercises.

Yes, he said, you are quite right in testing him. But what do you mean by the highest of all knowledge?

You may remember, I said, that we divided the soul into three parts; and distinguished the several natures of justice, temperance, courage, and wisdom?

Indeed, he said, if I had forgotten, I should not deserve to hear more.

And do you remember the word of caution which preceded the discussion of them?

To what do you refer?
We were saying, if I am not mistaken, that he who wanted to see them in their perfect beauty must take a longer and more circuitous way, at the end of which they would appear; but that we could add on a popular exposition of them on a level with the discussion which had preceded. And you replied that such an exposition would be enough for you, and so the enquiry was continued in what to me seemed to be a very inaccurate manner; whether you were satisfied or not, it is for you to say.

Yes, he said, I thought and the others thought that you gave us a fair measure of truth.

But, my friend, I said, a measure of such things Which in any degree falls short of the whole truth is not fair measure; for nothing imperfect is the measure of anything, although persons are too apt to be contented and think that they need search no further.

Not an uncommon case when people are indolent.
Yes, I said; and there cannot be any worse fault in a guardian of the State and of the laws.

True.
The guardian then, I said, must be required to take the longer circuit, and toll at learning as well as at gymnastics, or he will never reach the highest knowledge of all which, as we were just now saying, is his proper calling.

What, he said, is there a knowledge still higher than this --higher than justice and the other virtues?

Yes, I said, there is. And of the virtues too we must behold not the outline merely, as at present --nothing short of the most finished picture should satisfy us. When little things are elaborated with an infinity of pains, in order that they may appear in their full beauty and utmost clearness, how ridiculous that we should not think the highest truths worthy of attaining the highest accuracy!

A right noble thought; but do you suppose that we shall refrain from asking you what is this highest knowledge?

Nay, I said, ask if you will; but I am certain that you have heard the answer many times, and now you either do not understand me or, as I rather think, you are disposed to be troublesome; for you have of been told that the idea of good is the highest knowledge, and that all other things become useful and advantageous only by their use of this. You can hardly be ignorant that of this I was about to speak, concerning which, as you have often heard me say, we know so little; and, without which, any other knowledge or possession of any kind will profit us nothing. Do you think that the possession of all other things is of any value if we do not possess the good? or the knowledge of all other things if we have no knowledge of beauty and goodness?

Assuredly not.
You are further aware that most people affirm pleasure to be the good, but the finer sort of wits say it is knowledge

Yes.
And you are aware too that the latter cannot explain what they mean by knowledge, but are obliged after all to say knowledge of the good?

How ridiculous!
Yes, I said, that they should begin by reproaching us with our ignorance of the good, and then presume our knowledge of it --for the good they define to be knowledge of the good, just as if we understood them when they use the term 'good' --this is of course ridiculous.

Most true, he said.
And those who make pleasure their good are in equal perplexity; for they are compelled to admit that there are bad pleasures as well as good.

Certainly.
And therefore to acknowledge that bad and good are the same?
True.
There can be no doubt about the numerous difficulties in which this question is involved.

There can be none.
Further, do we not see that many are willing to do or to have or to seem to be what is just and honourable without the reality; but no one is satisfied with the appearance of good --the reality is what they seek; in the case of the good, appearance is despised by every one.

Very true, he said.
Of this then, which every soul of man pursues and makes the end of all his actions, having a presentiment that there is such an end, and yet hesitating because neither knowing the nature nor having the same assurance of this as of other things, and therefore losing whatever good there is in other things, --of a principle such and so great as this ought the best men in our State, to whom everything is entrusted, to be in the darkness of ignorance?

Certainly not, he said.
I am sure, I said, that he who does not know now the beautiful and the just are likewise good will be but a sorry guardian of them; and I suspect that no one who is ignorant of the good will have a true knowledge of them.

That, he said, is a shrewd suspicion of yours.
And if we only have a guardian who has this knowledge our State will be perfectly ordered?

Of course, he replied; but I wish that you would tell me whether you conceive this supreme principle of the good to be knowledge or pleasure, or different from either.

Aye, I said, I knew all along that a fastidious gentleman like you would not be contented with the thoughts of other people about these matters.

True, Socrates; but I must say that one who like you has passed a lifetime in the study of philosophy should not be always repeating the opinions of others, and never telling his own.

Well, but has any one a right to say positively what he does not know?

Not, he said, with the assurance of positive certainty; he has no right to do that: but he may say what he thinks, as a matter of opinion.

And do you not know, I said, that all mere opinions are bad, and the best of them blind? You would not deny that those who have any true notion without intelligence are only like blind men who feel their way along the road?

Very true.
And do you wish to behold what is blind and crooked and base, when others will tell you of brightness and beauty?

Glaucon - SOCRATES

Still, I must implore you, Socrates, said Glaucon, not to turn away just as you are reaching the goal; if you will only give such an explanation of the good as you have already given of justice and temperance and the other virtues, we shall be satisfied.

Yes, my friend, and I shall be at least equally satisfied, but I cannot help fearing that I shall fall, and that my indiscreet zeal will bring ridicule upon me. No, sweet sirs, let us not at present ask what is the actual nature of the good, for to reach what is now in my thoughts would be an effort too great for me. But of the child of the good who is likest him, I would fain speak, if I could be sure that you wished to hear --otherwise, not.

By all means, he said, tell us about the child, and you shall remain in our debt for the account of the parent.

I do indeed wish, I replied, that I could pay, and you receive, the account of the parent, and not, as now, of the offspring only; take, however, this latter by way of interest, and at the same time have a care that i do not render a false account, although I have no intention of deceiving you.

Yes, we will take all the care that we can: proceed.
Yes, I said, but I must first come to an understanding with you, and remind you of what I have mentioned in the course of this discussion, and at many other times.

What?
The old story, that there is a many beautiful and a many good, and so of other things which we describe and define; to all of them 'many' is applied.

True, he said.
And there is an absolute beauty and an absolute good, and of other things to which the term 'many' is applied there is an absolute; for they may be brought under a single idea, which is called the essence of each.

Very true.
The many, as we say, are seen but not known, and the ideas are known but not seen.

Exactly.
And what is the organ with which we see the visible things?
The sight, he said.
And with the hearing, I said, we hear, and with the other senses perceive the other objects of sense?

True.
But have you remarked that sight is by far the most costly and complex piece of workmanship which the artificer of the senses ever contrived?

No, I never have, he said.
Then reflect; has the ear or voice need of any third or additional nature in order that the one may be able to hear and the other to be heard?

Nothing of the sort.
No, indeed, I replied; and the same is true of most, if not all, the other senses --you would not say that any of them requires such an addition?

Certainly not.
But you see that without the addition of some other nature there is no seeing or being seen?

How do you mean?
Sight being, as I conceive, in the eyes, and he who has eyes wanting to see; colour being also present in them, still unless there be a third nature specially adapted to the purpose, the owner of the eyes will see nothing and the colours will be invisible.

Of what nature are you speaking?
Of that which you term light, I replied.
True, he said.
Noble, then, is the bond which links together sight and visibility, and great beyond other bonds by no small difference of nature; for light is their bond, and light is no ignoble thing?

Nay, he said, the reverse of ignoble.
And which, I said, of the gods in heaven would you say was the lord of this element? Whose is that light which makes the eye to see perfectly and the visible to appear?

You mean the sun, as you and all mankind say.
May not the relation of sight to this deity be described as follows?
How?
Neither sight nor the eye in which sight resides is the sun?
No.
Yet of all the organs of sense the eye is the most like the sun?
By far the most like.
And the power which the eye possesses is a sort of effluence which is dispensed from the sun?

Exactly.
Then the sun is not sight, but the author of sight who is recognised by sight.

True, he said.
And this is he whom I call the child of the good, whom the good begat in his own likeness, to be in the visible world, in relation to sight and the things of sight, what the good is in the intellectual world in relation to mind and the things of mind.

Will you be a little more explicit? he said.
Why, you know, I said, that the eyes, when a person directs them towards objects on which the light of day is no longer shining, but the moon and stars only, see dimly, and are nearly blind; they seem to have no clearness of vision in them?

Very true.
But when they are directed towards objects on which the sun shines, they see clearly and there is sight in them?

Certainly.
And the soul is like the eye: when resting upon that on which truth and being shine, the soul perceives and understands and is radiant with intelligence; but when turned towards the twilight of becoming and perishing, then she has opinion only, and goes blinking about, and is first of one opinion and then of another, and seems to have no intelligence?

Just so.
Now, that which imparts truth to the known and the power of knowing to the knower is what I would have you term the idea of good, and this you will deem to be the cause of science, and of truth in so far as the latter becomes the subject of knowledge; beautiful too, as are both truth and knowledge, you will be right in esteeming this other nature as more beautiful than either; and, as in the previous instance, light and sight may be truly said to be like the sun, and yet not to be the sun, so in this other sphere, science and truth may be deemed to be like the good, but not the good; the good has a place of honour yet higher.

What a wonder of beauty that must be, he said, which is the author of science and truth, and yet surpasses them in beauty; for you surely cannot mean to say that pleasure is the good?

God forbid, I replied; but may I ask you to consider the image in another point of view?

In what point of view?
You would say, would you not, that the sun is only the author of visibility in all visible things, but of generation and nourishment and growth, though he himself is not generation?

Certainly.
In like manner the good may be said to be not only the author of knowledge to all things known, but of their being and essence, and yet the good is not essence, but far exceeds essence in dignity and power.

Glaucon said, with a ludicrous earnestness: By the light of heaven, how amazing!

Yes, I said, and the exaggeration may be set down to you; for you made me utter my fancies.

And pray continue to utter them; at any rate let us hear if there is anything more to be said about the similitude of the sun.

Yes, I said, there is a great deal more.
Then omit nothing, however slight.
I will do my best, I said; but I should think that a great deal will have to be omitted.

You have to imagine, then, that there are two ruling powers, and that one of them is set over the intellectual world, the other over the visible. I do not say heaven, lest you should fancy that I am playing upon the name ('ourhanoz, orhatoz'). May I suppose that you have this distinction of the visible and intelligible fixed in your mind?

I have.
Now take a line which has been cut into two unequal parts, and divide each of them again in the same proportion, and suppose the two main divisions to answer, one to the visible and the other to the intelligible, and then compare the subdivisions in respect of their clearness and want of clearness, and you will find that the first section in the sphere of the visible consists of images. And by images I mean, in the first place, shadows, and in the second place, reflections in water and in solid, smooth and polished bodies and the like: Do you understand?

Yes, I understand.
Imagine, now, the other section, of which this is only the resemblance, to include the animals which we see, and everything that grows or is made.

Very good.
Would you not admit that both the sections of this division have different degrees of truth, and that the copy is to the original as the sphere of opinion is to the sphere of knowledge?

Most undoubtedly.
Next proceed to consider the manner in which the sphere of the intellectual is to be divided.

In what manner?
Thus: --There are two subdivisions, in the lower or which the soul uses the figures given by the former division as images; the enquiry can only be hypothetical, and instead of going upwards to a principle descends to the other end; in the higher of the two, the soul passes out of hypotheses, and goes up to a principle which is above hypotheses, making no use of images as in the former case, but proceeding only in and through the ideas themselves.

I do not quite understand your meaning, he said.
Then I will try again; you will understand me better when I have made some preliminary remarks. You are aware that students of geometry, arithmetic, and the kindred sciences assume the odd and the even and the figures and three kinds of angles and the like in their several branches of science; these are their hypotheses, which they and everybody are supposed to know, and therefore they do not deign to give any account of them either to themselves or others; but they begin with them, and go on until they arrive at last, and in a consistent manner, at their conclusion?

Yes, he said, I know.
And do you not know also that although they make use of the visible forms and reason about them, they are thinking not of these, but of the ideals which they resemble; not of the figures which they draw, but of the absolute square and the absolute diameter, and so on --the forms which they draw or make, and which have shadows and reflections in water of their own, are converted by them into images, but they are really seeking to behold the things themselves, which can only be seen with the eye of the mind?

That is true.
And of this kind I spoke as the intelligible, although in the search after it the soul is compelled to use hypotheses; not ascending to a first principle, because she is unable to rise above the region of hypothesis, but employing the objects of which the shadows below are resemblances in their turn as images, they having in relation to the shadows and reflections of them a greater distinctness, and therefore a higher value.

I understand, he said, that you are speaking of the province of geometry and the sister arts.

And when I speak of the other division of the intelligible, you will understand me to speak of that other sort of knowledge which reason herself attains by the power of dialectic, using the hypotheses not as first principles, but only as hypotheses --that is to say, as steps and points of departure into a world which is above hypotheses, in order that she may soar beyond them to the first principle of the whole; and clinging to this and then to that which depends on this, by successive steps she descends again without the aid of any sensible object, from ideas, through ideas, and in ideas she ends.

I understand you, he replied; not perfectly, for you seem to me to be describing a task which is really tremendous; but, at any rate, I understand you to say that knowledge and being, which the science of dialectic contemplates, are clearer than the notions of the arts, as they are termed, which proceed from hypotheses only: these are also contemplated by the understanding, and not by the senses: yet, because they start from hypotheses and do not ascend to a principle, those who contemplate them appear to you not to exercise the higher reason upon them, although when a first principle is added to them they are cognizable by the higher reason. And the habit which is concerned with geometry and the cognate sciences I suppose that you would term understanding and not reason, as being intermediate between opinion and reason.

You have quite conceived my meaning, I said; and now, corresponding to these four divisions, let there be four faculties in the soul-reason answering to the highest, understanding to the second, faith (or conviction) to the third, and perception of shadows to the last-and let there be a scale of them, and let us suppose that the several faculties have clearness in the same degree that their objects have truth.

I understand, he replied, and give my assent, and accept your arrangement.