\chapter{Book IV}

Adeimantus - SOCRATES

Here Adeimantus interposed a question: How would you answer, Socrates, said he, if a person were to say that you are making these people miserable, and that they are the cause of their own unhappiness; the city in fact belongs to them, but they are none the better for it; whereas other men acquire lands, and build large and handsome houses, and have everything handsome about them, offering sacrifices to the gods on their own account, and practising hospitality; moreover, as you were saying just now, they have gold and silver, and all that is usual among the favourites of fortune; but our poor citizens are no better than mercenaries who are quartered in the city and are always mounting guard?

Yes, I said; and you may add that they are only fed, and not paid in addition to their food, like other men; and therefore they cannot, if they would, take a journey of pleasure; they have no money to spend on a mistress or any other luxurious fancy, which, as the world goes, is thought to be happiness; and many other accusations of the same nature might be added.

But, said he, let us suppose all this to be included in the charge.
You mean to ask, I said, what will be our answer?
Yes.
If we proceed along the old path, my belief, I said, is that we shall find the answer. And our answer will be that, even as they are, our guardians may very likely be the happiest of men; but that our aim in founding the State was not the disproportionate happiness of any one class, but the greatest happiness of the whole; we thought that in a State which is ordered with a view to the good of the whole we should be most likely to find Justice, and in the ill-ordered State injustice: and, having found them, we might then decide which of the two is the happier. At present, I take it, we are fashioning the happy State, not piecemeal, or with a view of making a few happy citizens, but as a whole; and by-and-by we will proceed to view the opposite kind of State. Suppose that we were painting a statue, and some one came up to us and said, Why do you not put the most beautiful colours on the most beautiful parts of the body --the eyes ought to be purple, but you have made them black --to him we might fairly answer, Sir, you would not surely have us beautify the eyes to such a degree that they are no longer eyes; consider rather whether, by giving this and the other features their due proportion, we make the whole beautiful. And so I say to you, do not compel us to assign to the guardians a sort of happiness which will make them anything but guardians; for we too can clothe our husbandmen in royal apparel, and set crowns of gold on their heads, and bid them till the ground as much as they like, and no more. Our potters also might be allowed to repose on couches, and feast by the fireside, passing round the winecup, while their wheel is conveniently at hand, and working at pottery only as much as they like; in this way we might make every class happy-and then, as you imagine, the whole State would be happy. But do not put this idea into our heads; for, if we listen to you, the husbandman will be no longer a husbandman, the potter will cease to be a potter, and no one will have the character of any distinct class in the State. Now this is not of much consequence where the corruption of society, and pretension to be what you are not, is confined to cobblers; but when the guardians of the laws and of the government are only seemingly and not real guardians, then see how they turn the State upside down; and on the other hand they alone have the power of giving order and happiness to the State. We mean our guardians to be true saviours and not the destroyers of the State, whereas our opponent is thinking of peasants at a festival, who are enjoying a life of revelry, not of citizens who are doing their duty to the State. But, if so, we mean different things, and he is speaking of something which is not a State. And therefore we must consider whether in appointing our guardians we would look to their greatest happiness individually, or whether this principle of happiness does not rather reside in the State as a whole. But the latter be the truth, then the guardians and auxillaries, and all others equally with them, must be compelled or induced to do their own work in the best way. And thus the whole State will grow up in a noble order, and the several classes will receive the proportion of happiness which nature assigns to them.

I think that you are quite right.
I wonder whether you will agree with another remark which occurs to me.

What may that be?
There seem to be two causes of the deterioration of the arts.
What are they?
Wealth, I said, and poverty.
How do they act?
The process is as follows: When a potter becomes rich, will he, think you, any longer take the same pains with his art?

Certainly not.
He will grow more and more indolent and careless?
Very true.
And the result will be that he becomes a worse potter?
Yes; he greatly deteriorates.
But, on the other hand, if he has no money, and cannot provide himself tools or instruments, he will not work equally well himself, nor will he teach his sons or apprentices to work equally well.

Certainly not.
Then, under the influence either of poverty or of wealth, workmen and their work are equally liable to degenerate?

That is evident.
Here, then, is a discovery of new evils, I said, against which the guardians will have to watch, or they will creep into the city unobserved.

What evils?
Wealth, I said, and poverty; the one is the parent of luxury and indolence, and the other of meanness and viciousness, and both of discontent.

That is very true, he replied; but still I should like to know, Socrates, how our city will be able to go to war, especially against an enemy who is rich and powerful, if deprived of the sinews of war.

There would certainly be a difficulty, I replied, in going to war with one such enemy; but there is no difficulty where there are two of them.

How so? he asked.
In the first place, I said, if we have to fight, our side will be trained warriors fighting against an army of rich men.

That is true, he said.
And do you not suppose, Adeimantus, that a single boxer who was perfect in his art would easily be a match for two stout and well-to-do gentlemen who were not boxers?

Hardly, if they came upon him at once.
What, not, I said, if he were able to run away and then turn and strike at the one who first came up? And supposing he were to do this several times under the heat of a scorching sun, might he not, being an expert, overturn more than one stout personage?

Certainly, he said, there would be nothing wonderful in that.
And yet rich men probably have a greater superiority in the science and practice of boxing than they have in military qualities.

Likely enough.
Then we may assume that our athletes will be able to fight with two or three times their own number?

I agree with you, for I think you right.
And suppose that, before engaging, our citizens send an embassy to one of the two cities, telling them what is the truth: Silver and gold we neither have nor are permitted to have, but you may; do you therefore come and help us in war, of and take the spoils of the other city: Who, on hearing these words, would choose to fight against lean wiry dogs, rather th than, with the dogs on their side, against fat and tender sheep?

That is not likely; and yet there might be a danger to the poor State if the wealth of many States were to be gathered into one.

But how simple of you to use the term State at all of any but our own!

Why so?
You ought to speak of other States in the plural number; not one of them is a city, but many cities, as they say in the game. For indeed any city, however small, is in fact divided into two, one the city of the poor, the other of the rich; these are at war with one another; and in either there are many smaller divisions, and you would be altogether beside the mark if you treated them all as a single State. But if you deal with them as many, and give the wealth or power or persons of the one to the others, you will always have a great many friends and not many enemies. And your State, while the wise order which has now been prescribed continues to prevail in her, will be the greatest of States, I do not mean to say in reputation or appearance, but in deed and truth, though she number not more than a thousand defenders. A single State which is her equal you will hardly find, either among Hellenes or barbarians, though many that appear to be as great and many times greater.

That is most true, he said.
And what, I said, will be the best limit for our rulers to fix when they are considering the size of the State and the amount of territory which they are to include, and beyond which they will not go?

What limit would you propose?
I would allow the State to increase so far as is consistent with unity; that, I think, is the proper limit.

Very good, he said.
Here then, I said, is another order which will have to be conveyed to our guardians: Let our city be accounted neither large nor small, but one and self-sufficing.

And surely, said he, this is not a very severe order which we impose upon them.

And the other, said I, of which we were speaking before is lighter still, -I mean the duty of degrading the offspring of the guardians when inferior, and of elevating into the rank of guardians the offspring of the lower classes, when naturally superior. The intention was, that, in the case of the citizens generally, each individual should be put to the use for which nature which nature intended him, one to one work, and then every man would do his own business, and be one and not many; and so the whole city would be one and not many.

Yes, he said; that is not so difficult.
The regulations which we are prescribing, my good Adeimantus, are not, as might be supposed, a number of great principles, but trifles all, if care be taken, as the saying is, of the one great thing, --a thing, however, which I would rather call, not great, but sufficient for our purpose.

What may that be? he asked.
Education, I said, and nurture: If our citizens are well educated, and grow into sensible men, they will easily see their way through all these, as well as other matters which I omit; such, for example, as marriage, the possession of women and the procreation of children, which will all follow the general principle that friends have all things in common, as the proverb says.

That will be the best way of settling them.
Also, I said, the State, if once started well, moves with accumulating force like a wheel. For good nurture and education implant good constitutions, and these good constitutions taking root in a good education improve more and more, and this improvement affects the breed in man as in other animals.

Very possibly, he said.
Then to sum up: This is the point to which, above all, the attention of our rulers should be directed, --that music and gymnastic be preserved in their original form, and no innovation made. They must do their utmost to maintain them intact. And when any one says that mankind most regard

The newest song which the singers have, they will be afraid that he may be praising, not new songs, but a new kind of song; and this ought not to be praised, or conceived to be the meaning of the poet; for any musical innovation is full of danger to the whole State, and ought to be prohibited. So Damon tells me, and I can quite believe him;-he says that when modes of music change, of the State always change with them.

Yes, said Adeimantus; and you may add my suffrage to Damon's and your own.

Then, I said, our guardians must lay the foundations of their fortress in music?

Yes, he said; the lawlessness of which you speak too easily steals in.

Yes, I replied, in the form of amusement; and at first sight it appears harmless.

Why, yes, he said, and there is no harm; were it not that little by little this spirit of licence, finding a home, imperceptibly penetrates into manners and customs; whence, issuing with greater force, it invades contracts between man and man, and from contracts goes on to laws and constitutions, in utter recklessness, ending at last, Socrates, by an overthrow of all rights, private as well as public.

Is that true? I said.
That is my belief, he replied.
Then, as I was saying, our youth should be trained from the first in a stricter system, for if amusements become lawless, and the youths themselves become lawless, they can never grow up into well-conducted and virtuous citizens.

Very true, he said.
And when they have made a good beginning in play, and by the help of music have gained the habit of good order, then this habit of order, in a manner how unlike the lawless play of the others! will accompany them in all their actions and be a principle of growth to them, and if there be any fallen places a principle in the State will raise them up again.

Very true, he said.
Thus educated, they will invent for themselves any lesser rules which their predecessors have altogether neglected.

What do you mean?
I mean such things as these: --when the young are to be silent before their elders; how they are to show respect to them by standing and making them sit; what honour is due to parents; what garments or shoes are to be worn; the mode of dressing the hair; deportment and manners in general. You would agree with me?

Yes.
But there is, I think, small wisdom in legislating about such matters, --I doubt if it is ever done; nor are any precise written enactments about them likely to be lasting.

Impossible.
It would seem, Adeimantus, that the direction in which education starts a man, will determine his future life. Does not like always attract like?

To be sure.
Until some one rare and grand result is reached which may be good, and may be the reverse of good?

That is not to be denied.
And for this reason, I said, I shall not attempt to legislate further about them.

Naturally enough, he replied.
Well, and about the business of the agora, dealings and the ordinary dealings between man and man, or again about agreements with the commencement with artisans; about insult and injury, of the commencement of actions, and the appointment of juries, what would you say? there may also arise questions about any impositions and extractions of market and harbour dues which may be required, and in general about the regulations of markets, police, harbours, and the like. But, oh heavens! shall we condescend to legislate on any of these particulars?

I think, he said, that there is no need to impose laws about them on good men; what regulations are necessary they will find out soon enough for themselves.

Yes, I said, my friend, if God will only preserve to them the laws which we have given them.

And without divine help, said Adeimantus, they will go on for ever making and mending their laws and their lives in the hope of attaining perfection.

You would compare them, I said, to those invalids who, having no self-restraint, will not leave off their habits of intemperance?

Exactly.
Yes, I said; and what a delightful life they lead! they are always doctoring and increasing and complicating their disorders, and always fancying that they will be cured by any nostrum which anybody advises them to try.

Such cases are very common, he said, with invalids of this sort.
Yes, I replied; and the charming thing is that they deem him their worst enemy who tells them the truth, which is simply that, unless they give up eating and drinking and wenching and idling, neither drug nor cautery nor spell nor amulet nor any other remedy will avail.

Charming! he replied. I see nothing charming in going into a passion with a man who tells you what is right.

These gentlemen, I said, do not seem to be in your good graces.
Assuredly not.
Nor would you praise the behaviour of States which act like the men whom I was just now describing. For are there not ill-ordered States in which the citizens are forbidden under pain of death to alter the constitution; and yet he who most sweetly courts those who live under this regime and indulges them and fawns upon them and is skilful in anticipating and gratifying their humours is held to be a great and good statesman --do not these States resemble the persons whom I was describing?

Yes, he said; the States are as bad as the men; and I am very far from praising them.

But do you not admire, I said, the coolness and dexterity of these ready ministers of political corruption?

Yes, he said, I do; but not of all of them, for there are some whom the applause of the multitude has deluded into the belief that they are really statesmen, and these are not much to be admired.

What do you mean? I said; you should have more feeling for them. When a man cannot measure, and a great many others who cannot measure declare that he is four cubits high, can he help believing what they say?

Nay, he said, certainly not in that case.
Well, then, do not be angry with them; for are they not as good as a play, trying their hand at paltry reforms such as I was describing; they are always fancying that by legislation they will make an end of frauds in contracts, and the other rascalities which I was mentioning, not knowing that they are in reality cutting off the heads of a hydra?

Yes, he said; that is just what they are doing.
I conceive, I said, that the true legislator will not trouble himself with this class of enactments whether concerning laws or the constitution either in an ill-ordered or in a well-ordered State; for in the former they are quite useless, and in the latter there will be no difficulty in devising them; and many of them will naturally flow out of our previous regulations.

What, then, he said, is still remaining to us of the work of legislation?

Nothing to us, I replied; but to Apollo, the God of Delphi, there remains the ordering of the greatest and noblest and chiefest things of all.

Which are they? he said.
The institution of temples and sacrifices, and the entire service of gods, demigods, and heroes; also the ordering of the repositories of the dead, and the rites which have to be observed by him who would propitiate the inhabitants of the world below. These are matters of which we are ignorant ourselves, and as founders of a city we should be unwise in trusting them to any interpreter but our ancestral deity. He is the god who sits in the center, on the navel of the earth, and he is the interpreter of religion to all mankind.

You are right, and we will do as you propose.
But where, amid all this, is justice? son of Ariston, tell me where. Now that our city has been made habitable, light a candle and search, and get your brother and Polemarchus and the rest of our friends to help, and let us see where in it we can discover justice and where injustice, and in what they differ from one another, and which of them the man who would be happy should have for his portion, whether seen or unseen by gods and men.

Socrates - GLAUCON

Nonsense, said Glaucon: did you not promise to search yourself, saying that for you not to help justice in her need would be an impiety?

I do not deny that I said so, and as you remind me, I will be as good as my word; but you must join.

We will, he replied.
Well, then, I hope to make the discovery in this way: I mean to begin with the assumption that our State, if rightly ordered, is perfect.

That is most certain.
And being perfect, is therefore wise and valiant and temperate and just.

That is likewise clear.
And whichever of these qualities we find in the State, the one which is not found will be the residue?

Very good.
If there were four things, and we were searching for one of them, wherever it might be, the one sought for might be known to us from the first, and there would be no further trouble; or we might know the other three first, and then the fourth would clearly be the one left.

Very true, he said.
And is not a similar method to be pursued about the virtues, which are also four in number?

Clearly.
First among the virtues found in the State, wisdom comes into view, and in this I detect a certain peculiarity.

What is that?
The State which we have been describing is said to be wise as being good in counsel?

Very true.
And good counsel is clearly a kind of knowledge, for not by ignorance, but by knowledge, do men counsel well?

Clearly.
And the kinds of knowledge in a State are many and diverse?
Of course.
There is the knowledge of the carpenter; but is that the sort of knowledge which gives a city the title of wise and good in counsel?

Certainly not; that would only give a city the reputation of skill in carpentering.

Then a city is not to be called wise because possessing a knowledge which counsels for the best about wooden implements?

Certainly not.
Nor by reason of a knowledge which advises about brazen pots, I said, nor as possessing any other similar knowledge?

Not by reason of any of them, he said.
Nor yet by reason of a knowledge which cultivates the earth; that would give the city the name of agricultural?

Yes.
Well, I said, and is there any knowledge in our recently founded State among any of the citizens which advises, not about any particular thing in the State, but about the whole, and considers how a State can best deal with itself and with other States?

There certainly is.
And what is knowledge, and among whom is it found? I asked.
It is the knowledge of the guardians, he replied, and found among those whom we were just now describing as perfect guardians.

And what is the name which the city derives from the possession of this sort of knowledge?

The name of good in counsel and truly wise.
And will there be in our city more of these true guardians or more smiths?

The smiths, he replied, will be far more numerous.
Will not the guardians be the smallest of all the classes who receive a name from the profession of some kind of knowledge?

Much the smallest.
And so by reason of the smallest part or class, and of the knowledge which resides in this presiding and ruling part of itself, the whole State, being thus constituted according to nature, will be wise; and this, which has the only knowledge worthy to be called wisdom, has been ordained by nature to be of all classes the least.

Most true.
Thus, then, I said, the nature and place in the State of one of the four virtues has somehow or other been discovered.

And, in my humble opinion, very satisfactorily discovered, he replied.

Again, I said, there is no difficulty in seeing the nature of courage; and in what part that quality resides which gives the name of courageous to the State.

How do you mean?
Why, I said, every one who calls any State courageous or cowardly, will be thinking of the part which fights and goes out to war on the State's behalf.

No one, he replied, would ever think of any other.
Certainly not.
The rest of the citizens may be courageous or may be cowardly but their courage or cowardice will not, as I conceive, have the effect of making the city either the one or the other.

The city will be courageous in virtue of a portion of herself which preserves under all circumstances that opinion about the nature of things to be feared and not to be feared in which our legislator educated them; and this is what you term courage.

I should like to hear what you are saying once more, for I do not think that I perfectly understand you.

I mean that courage is a kind of salvation.
Salvation of what?
Of the opinion respecting things to be feared, what they are and of what nature, which the law implants through education; and I mean by the words 'under all circumstances' to intimate that in pleasure or in pain, or under the influence of desire or fear, a man preserves, and does not lose this opinion. Shall I give you an illustration?

If you please.
You know, I said, that dyers, when they want to dye wool for making the true sea-purple, begin by selecting their white colour first; this they prepare and dress with much care and pains, in order that the white ground may take the purple hue in full perfection. The dyeing then proceeds; and whatever is dyed in this manner becomes a fast colour, and no washing either with lyes or without them can take away the bloom. But, when the ground has not been duly prepared, you will have noticed how poor is the look either of purple or of any other colour.

Yes, he said; I know that they have a washed-out and ridiculous appearance.

Then now, I said, you will understand what our object was in selecting our soldiers, and educating them in music and gymnastic; we were contriving influences which would prepare them to take the dye of the laws in perfection, and the colour of their opinion about dangers and of every other opinion was to be indelibly fixed by their nurture and training, not to be washed away by such potent lyes as pleasure --mightier agent far in washing the soul than any soda or lye; or by sorrow, fear, and desire, the mightiest of all other solvents. And this sort of universal saving power of true opinion in conformity with law about real and false dangers I call and maintain to be courage, unless you disagree.

But I agree, he replied; for I suppose that you mean to exclude mere uninstructed courage, such as that of a wild beast or of a slave --this, in your opinion, is not the courage which the law ordains, and ought to have another name.

Most certainly.
Then I may infer courage to be such as you describe?
Why, yes, said I, you may, and if you add the words 'of a citizen,' you will not be far wrong; --hereafter, if you like, we will carry the examination further, but at present we are we w seeking not for courage but justice; and for the purpose of our enquiry we have said enough.

You are right, he replied.
Two virtues remain to be discovered in the State-first temperance, and then justice which is the end of our search.

Very true.
Now, can we find justice without troubling ourselves about temperance?

I do not know how that can be accomplished, he said, nor do I desire that justice should be brought to light and temperance lost sight of; and therefore I wish that you would do me the favour of considering temperance first.

Certainly, I replied, I should not be justified in refusing your request.

Then consider, he said.
Yes, I replied; I will; and as far as I can at present see, the virtue of temperance has more of the nature of harmony and symphony than the preceding.

How so? he asked.
Temperance, I replied, is the ordering or controlling of certain pleasures and desires; this is curiously enough implied in the saying of 'a man being his own master' and other traces of the same notion may be found in language.

No doubt, he said.
There is something ridiculous in the expression 'master of himself'; for the master is also the servant and the servant the master; and in all these modes of speaking the same person is denoted.

Certainly.
The meaning is, I believe, that in the human soul there is a better and also a worse principle; and when the better has the worse under control, then a man is said to be master of himself; and this is a term of praise: but when, owing to evil education or association, the better principle, which is also the smaller, is overwhelmed by the greater mass of the worse --in this case he is blamed and is called the slave of self and unprincipled.

Yes, there is reason in that.
And now, I said, look at our newly created State, and there you will find one of these two conditions realised; for the State, as you will acknowledge, may be justly called master of itself, if the words 'temperance' and 'self-mastery' truly express the rule of the better part over the worse.

Yes, he said, I see that what you say is true.
Let me further note that the manifold and complex pleasures and desires and pains are generally found in children and women and servants, and in the freemen so called who are of the lowest and more numerous class.

Certainly, he said.
Whereas the simple and moderate desires which follow reason, and are under the guidance of mind and true opinion, are to be found only in a few, and those the best born and best educated.

Very true. These two, as you may perceive, have a place in our State; and the meaner desires of the are held down by the virtuous desires and wisdom of the few.

That I perceive, he said.
Then if there be any city which may be described as master of its own pleasures and desires, and master of itself, ours may claim such a designation?

Certainly, he replied.
It may also be called temperate, and for the same reasons?
Yes.
And if there be any State in which rulers and subjects will be agreed as to the question who are to rule, that again will be our State?

Undoubtedly.
And the citizens being thus agreed among themselves, in which class will temperance be found --in the rulers or in the subjects?

In both, as I should imagine, he replied.
Do you observe that we were not far wrong in our guess that temperance was a sort of harmony?

Why so?
Why, because temperance is unlike courage and wisdom, each of which resides in a part only, the one making the State wise and the other valiant; not so temperance, which extends to the whole, and runs through all the notes of the scale, and produces a harmony of the weaker and the stronger and the middle class, whether you suppose them to be stronger or weaker in wisdom or power or numbers or wealth, or anything else. Most truly then may we deem temperance to be the agreement of the naturally superior and inferior, as to the right to rule of either, both in states and individuals.

I entirely agree with you.
And so, I said, we may consider three out of the four virtues to have been discovered in our State. The last of those qualities which make a state virtuous must be justice, if we only knew what that was.

The inference is obvious.
The time then has arrived, Glaucon, when, like huntsmen, we should surround the cover, and look sharp that justice does not steal away, and pass out of sight and escape us; for beyond a doubt she is somewhere in this country: watch therefore and strive to catch a sight of her, and if you see her first, let me know.

Would that I could! but you should regard me rather as a follower who has just eyes enough to, see what you show him --that is about as much as I am good for.

Offer up a prayer with me and follow.
I will, but you must show me the way.
Here is no path, I said, and the wood is dark and perplexing; still we must push on.

Let us push on.
Here I saw something: Halloo! I said, I begin to perceive a track, and I believe that the quarry will not escape.

Good news, he said.
Truly, I said, we are stupid fellows.
Why so?
Why, my good sir, at the beginning of our enquiry, ages ago, there was justice tumbling out at our feet, and we never saw her; nothing could be more ridiculous. Like people who go about looking for what they have in their hands --that was the way with us --we looked not at what we were seeking, but at what was far off in the distance; and therefore, I suppose, we missed her.

What do you mean?
I mean to say that in reality for a long time past we have been talking of justice, and have failed to recognise her.

I grow impatient at the length of your exordium.
Well then, tell me, I said, whether I am right or not: You remember the original principle which we were always laying down at the foundation of the State, that one man should practise one thing only, the thing to which his nature was best adapted; --now justice is this principle or a part of it.

Yes, we often said that one man should do one thing only.
Further, we affirmed that justice was doing one's own business, and not being a busybody; we said so again and again, and many others have said the same to us.

Yes, we said so.
Then to do one's own business in a certain way may be assumed to be justice. Can you tell me whence I derive this inference?

I cannot, but I should like to be told.
Because I think that this is the only virtue which remains in the State when the other virtues of temperance and courage and wisdom are abstracted; and, that this is the ultimate cause and condition of the existence of all of them, and while remaining in them is also their preservative; and we were saying that if the three were discovered by us, justice would be the fourth or remaining one.

That follows of necessity.
If we are asked to determine which of these four qualities by its presence contributes most to the excellence of the State, whether the agreement of rulers and subjects, or the preservation in the soldiers of the opinion which the law ordains about the true nature of dangers, or wisdom and watchfulness in the rulers, or whether this other which I am mentioning, and which is found in children and women, slave and freeman, artisan, ruler, subject, --the quality, I mean, of every one doing his own work, and not being a busybody, would claim the palm --the question is not so easily answered.

Certainly, he replied, there would be a difficulty in saying which.
Then the power of each individual in the State to do his own work appears to compete with the other political virtues, wisdom, temperance, courage.

Yes, he said.
And the virtue which enters into this competition is justice?
Exactly.
Let us look at the question from another point of view: Are not the rulers in a State those to whom you would entrust the office of determining suits at law?

Certainly.
And are suits decided on any other ground but that a man may neither take what is another's, nor be deprived of what is his own?

Yes; that is their principle.
Which is a just principle?
Yes.
Then on this view also justice will be admitted to be the having and doing what is a man's own, and belongs to him?

Very true.
Think, now, and say whether you agree with me or not. Suppose a carpenter to be doing the business of a cobbler, or a cobbler of a carpenter; and suppose them to exchange their implements or their duties, or the same person to be doing the work of both, or whatever be the change; do you think that any great harm would result to the State?

Not much.
But when the cobbler or any other man whom nature designed to be a trader, having his heart lifted up by wealth or strength or the number of his followers, or any like advantage, attempts to force his way into the class of warriors, or a warrior into that of legislators and guardians, for which he is unfitted, and either to take the implements or the duties of the other; or when one man is trader, legislator, and warrior all in one, then I think you will agree with me in saying that this interchange and this meddling of one with another is the ruin of the State.

Most true.
Seeing then, I said, that there are three distinct classes, any meddling of one with another, or the change of one into another, is the greatest harm to the State, and may be most justly termed evil-doing?

Precisely.
And the greatest degree of evil-doing to one's own city would be termed by you injustice?

Certainly.
This then is injustice; and on the other hand when the trader, the auxiliary, and the guardian each do their own business, that is justice, and will make the city just.

I agree with you.
We will not, I said, be over-positive as yet; but if, on trial, this conception of justice be verified in the individual as well as in the State, there will be no longer any room for doubt; if it be not verified, we must have a fresh enquiry. First let us complete the old investigation, which we began, as you remember, under the impression that, if we could previously examine justice on the larger scale, there would be less difficulty in discerning her in the individual. That larger example appeared to be the State, and accordingly we constructed as good a one as we could, knowing well that in the good State justice would be found. Let the discovery which we made be now applied to the individual --if they agree, we shall be satisfied; or, if there be a difference in the individual, we will come back to the State and have another trial of the theory. The friction of the two when rubbed together may possibly strike a light in which justice will shine forth, and the vision which is then revealed we will fix in our souls.

That will be in regular course; let us do as you say.
I proceeded to ask: When two things, a greater and less, are called by the same name, are they like or unlike in so far as they are called the same?

Like, he replied.
The just man then, if we regard the idea of justice only, will be like the just State?

He will.
And a State was thought by us to be just when the three classes in the State severally did their own business; and also thought to be temperate and valiant and wise by reason of certain other affections and qualities of these same classes?

True, he said.
And so of the individual; we may assume that he has the same three principles in his own soul which are found in the State; and he may be rightly described in the same terms, because he is affected in the same manner?

Certainly, he said.
Once more then, O my friend, we have alighted upon an easy question --whether the soul has these three principles or not?

An easy question! Nay, rather, Socrates, the proverb holds that hard is the good.

Very true, I said; and I do not think that the method which we are employing is at all adequate to the accurate solution of this question; the true method is another and a longer one. Still we may arrive at a solution not below the level of the previous enquiry.

May we not be satisfied with that? he said; --under the circumstances, I am quite content.

I too, I replied, shall be extremely well satisfied.
Then faint not in pursuing the speculation, he said.
Must we not acknowledge, I said, that in each of us there are the same principles and habits which there are in the State; and that from the individual they pass into the State? --how else can they come there? Take the quality of passion or spirit; --it would be ridiculous to imagine that this quality, when found in States, is not derived from the individuals who are supposed to possess it, e.g. the Thracians, Scythians, and in general the northern nations; and the same may be said of the love of knowledge, which is the special characteristic of our part of the world, or of the love of money, which may, with equal truth, be attributed to the Phoenicians and Egyptians.

Exactly so, he said.
There is no difficulty in understanding this.
None whatever.
But the question is not quite so easy when we proceed to ask whether these principles are three or one; whether, that is to say, we learn with one part of our nature, are angry with another, and with a third part desire the satisfaction of our natural appetites; or whether the whole soul comes into play in each sort of action --to determine that is the difficulty.

Yes, he said; there lies the difficulty.
Then let us now try and determine whether they are the same or different.

How can we? he asked.
I replied as follows: The same thing clearly cannot act or be acted upon in the same part or in relation to the same thing at the same time, in contrary ways; and therefore whenever this contradiction occurs in things apparently the same, we know that they are really not the same, but different.

Good.
For example, I said, can the same thing be at rest and in motion at the same time in the same part?

Impossible.
Still, I said, let us have a more precise statement of terms, lest we should hereafter fall out by the way. Imagine the case of a man who is standing and also moving his hands and his head, and suppose a person to say that one and the same person is in motion and at rest at the same moment-to such a mode of speech we should object, and should rather say that one part of him is in motion while another is at rest.

Very true.
And suppose the objector to refine still further, and to draw the nice distinction that not only parts of tops, but whole tops, when they spin round with their pegs fixed on the spot, are at rest and in motion at the same time (and he may say the same of anything which revolves in the same spot), his objection would not be admitted by us, because in such cases things are not at rest and in motion in the same parts of themselves; we should rather say that they have both an axis and a circumference, and that the axis stands still, for there is no deviation from the perpendicular; and that the circumference goes round. But if, while revolving, the axis inclines either to the right or left, forwards or backwards, then in no point of view can they be at rest.

That is the correct mode of describing them, he replied.
Then none of these objections will confuse us, or incline us to believe that the same thing at the same time, in the same part or in relation to the same thing, can act or be acted upon in contrary ways.

Certainly not, according to my way of thinking.
Yet, I said, that we may not be compelled to examine all such objections, and prove at length that they are untrue, let us assume their absurdity, and go forward on the understanding that hereafter, if this assumption turn out to be untrue, all the consequences which follow shall be withdrawn.

Yes, he said, that will be the best way.
Well, I said, would you not allow that assent and dissent, desire and aversion, attraction and repulsion, are all of them opposites, whether they are regarded as active or passive (for that makes no difference in the fact of their opposition)?

Yes, he said, they are opposites.
Well, I said, and hunger and thirst, and the desires in general, and again willing and wishing, --all these you would refer to the classes already mentioned. You would say --would you not? --that the soul of him who desires is seeking after the object of his desires; or that he is drawing to himself the thing which he wishes to possess: or again, when a person wants anything to be given him, his mind, longing for the realisation of his desires, intimates his wish to have it by a nod of assent, as if he had been asked a question?

Very true.
And what would you say of unwillingness and dislike and the absence of desire; should not these be referred to the opposite class of repulsion and rejection?

Certainly.
Admitting this to be true of desire generally, let us suppose a particular class of desires, and out of these we will select hunger and thirst, as they are termed, which are the most obvious of them?

Let us take that class, he said.
The object of one is food, and of the other drink?
Yes.
And here comes the point: is not thirst the desire which the soul has of drink, and of drink only; not of drink qualified by anything else; for example, warm or cold, or much or little, or, in a word, drink of any particular sort: but if the thirst be accompanied by heat, then the desire is of cold drink; or, if accompanied by cold, then of warm drink; or, if the thirst be excessive, then the drink which is desired will be excessive; or, if not great, the quantity of drink will also be small: but thirst pure and simple will desire drink pure and simple, which is the natural satisfaction of thirst, as food is of hunger?

Yes, he said; the simple desire is, as you say, in every case of the simple object, and the qualified desire of the qualified object.

But here a confusion may arise; and I should wish to guard against an opponent starting up and saying that no man desires drink only, but good drink, or food only, but good food; for good is the universal object of desire, and thirst being a desire, will necessarily be thirst after good drink; and the same is true of every other desire.

Yes, he replied, the opponent might have something to say.
Nevertheless I should still maintain, that of relatives some have a quality attached to either term of the relation; others are simple and have their correlatives simple.

I do not know what you mean.
Well, you know of course that the greater is relative to the less?
Certainly.
And the much greater to the much less?
Yes.
And the sometime greater to the sometime less, and the greater that is to be to the less that is to be?

Certainly, he said.
And so of more and less, and of other correlative terms, such as the double and the half, or again, the heavier and the lighter, the swifter and the slower; and of hot and cold, and of any other relatives; --is not this true of all of them?

Yes.
And does not the same principle hold in the sciences? The object of science is knowledge (assuming that to be the true definition), but the object of a particular science is a particular kind of knowledge; I mean, for example, that the science of house-building is a kind of knowledge which is defined and distinguished from other kinds and is therefore termed architecture.

Certainly.
Because it has a particular quality which no other has?
Yes.
And it has this particular quality because it has an object of a particular kind; and this is true of the other arts and sciences?

Yes.
Now, then, if I have made myself clear, you will understand my original meaning in what I said about relatives. My meaning was, that if one term of a relation is taken alone, the other is taken alone; if one term is qualified, the other is also qualified. I do not mean to say that relatives may not be disparate, or that the science of health is healthy, or of disease necessarily diseased, or that the sciences of good and evil are therefore good and evil; but only that, when the term science is no longer used absolutely, but has a qualified object which in this case is the nature of health and disease, it becomes defined, and is hence called not merely science, but the science of medicine.

I quite understand, and I think as you do.
Would you not say that thirst is one of these essentially relative terms, having clearly a relation --

Yes, thirst is relative to drink.
And a certain kind of thirst is relative to a certain kind of drink; but thirst taken alone is neither of much nor little, nor of good nor bad, nor of any particular kind of drink, but of drink only?

Certainly.
Then the soul of the thirsty one, in so far as he is thirsty, desires only drink; for this he yearns and tries to obtain it?

That is plain.
And if you suppose something which pulls a thirsty soul away from drink, that must be different from the thirsty principle which draws him like a beast to drink; for, as we were saying, the same thing cannot at the same time with the same part of itself act in contrary ways about the same.

Impossible.
No more than you can say that the hands of the archer push and pull the bow at the same time, but what you say is that one hand pushes and the other pulls.

Exactly so, he replied.
And might a man be thirsty, and yet unwilling to drink?
Yes, he said, it constantly happens.
And in such a case what is one to say? Would you not say that there was something in the soul bidding a man to drink, and something else forbidding him, which is other and stronger than the principle which bids him?

I should say so.
And the forbidding principle is derived from reason, and that which bids and attracts proceeds from passion and disease?

Clearly.
Then we may fairly assume that they are two, and that they differ from one another; the one with which man reasons, we may call the rational principle of the soul, the other, with which he loves and hungers and thirsts and feels the flutterings of any other desire, may be termed the irrational or appetitive, the ally of sundry pleasures and satisfactions?

Yes, he said, we may fairly assume them to be different.
Then let us finally determine that there are two principles existing in the soul. And what of passion, or spirit? Is it a third, or akin to one of the preceding?

I should be inclined to say --akin to desire.
Well, I said, there is a story which I remember to have heard, and in which I put faith. The story is, that Leontius, the son of Aglaion, coming up one day from the Piraeus, under the north wall on the outside, observed some dead bodies lying on the ground at the place of execution. He felt a desire to see them, and also a dread and abhorrence of them; for a time he struggled and covered his eyes, but at length the desire got the better of him; and forcing them open, he ran up to the dead bodies, saying, Look, ye wretches, take your fill of the fair sight.

I have heard the story myself, he said.
The moral of the tale is, that anger at times goes to war with desire, as though they were two distinct things.

Yes; that is the meaning, he said.
And are there not many other cases in which we observe that when a man's desires violently prevail over his reason, he reviles himself, and is angry at the violence within him, and that in this struggle, which is like the struggle of factions in a State, his spirit is on the side of his reason; --but for the passionate or spirited element to take part with the desires when reason that she should not be opposed, is a sort of thing which thing which I believe that you never observed occurring in yourself, nor, as I should imagine, in any one else?

Certainly not.
Suppose that a man thinks he has done a wrong to another, the nobler he is the less able is he to feel indignant at any suffering, such as hunger, or cold, or any other pain which the injured person may inflict upon him --these he deems to be just, and, as I say, his anger refuses to be excited by them.

True, he said.
But when he thinks that he is the sufferer of the wrong, then he boils and chafes, and is on the side of what he believes to be justice; and because he suffers hunger or cold or other pain he is only the more determined to persevere and conquer. His noble spirit will not be quelled until he either slays or is slain; or until he hears the voice of the shepherd, that is, reason, bidding his dog bark no more.

The illustration is perfect, he replied; and in our State, as we were saying, the auxiliaries were to be dogs, and to hear the voice of the rulers, who are their shepherds.

I perceive, I said, that you quite understand me; there is, however, a further point which I wish you to consider.

What point?
You remember that passion or spirit appeared at first sight to be a kind of desire, but now we should say quite the contrary; for in the conflict of the soul spirit is arrayed on the side of the rational principle.

Most assuredly.
But a further question arises: Is passion different from reason also, or only a kind of reason; in which latter case, instead of three principles in the soul, there will only be two, the rational and the concupiscent; or rather, as the State was composed of three classes, traders, auxiliaries, counsellors, so may there not be in the individual soul a third element which is passion or spirit, and when not corrupted by bad education is the natural auxiliary of reason

Yes, he said, there must be a third.
Yes, I replied, if passion, which has already been shown to be different from desire, turn out also to be different from reason.

But that is easily proved: --We may observe even in young children that they are full of spirit almost as soon as they are born, whereas some of them never seem to attain to the use of reason, and most of them late enough.

Excellent, I said, and you may see passion equally in brute animals, which is a further proof of the truth of what you are saying. And we may once more appeal to the words of Homer, which have been already quoted by us,

He smote his breast, and thus rebuked his soul, for in this verse Homer has clearly supposed the power which reasons about the better and worse to be different from the unreasoning anger which is rebuked by it.

Very true, he said.
And so, after much tossing, we have reached land, and are fairly agreed that the same principles which exist in the State exist also in the individual, and that they are three in number.

Exactly.
Must we not then infer that the individual is wise in the same way, and in virtue of the same quality which makes the State wise?

Certainly.
Also that the same quality which constitutes courage in the State constitutes courage in the individual, and that both the State and the individual bear the same relation to all the other virtues?

Assuredly.
And the individual will be acknowledged by us to be just in the same way in which the State is just?

That follows, of course.
We cannot but remember that the justice of the State consisted in each of the three classes doing the work of its own class?

We are not very likely to have forgotten, he said.
We must recollect that the individual in whom the several qualities of his nature do their own work will be just, and will do his own work?

Yes, he said, we must remember that too.
And ought not the rational principle, which is wise, and has the care of the whole soul, to rule, and the passionate or spirited principle to be the subject and ally?

Certainly.
And, as we were saying, the united influence of music and gymnastic will bring them into accord, nerving and sustaining the reason with noble words and lessons, and moderating and soothing and civilizing the wildness of passion by harmony and rhythm?

Quite true, he said.
And these two, thus nurtured and educated, and having learned truly to know their own functions, will rule over the concupiscent, which in each of us is the largest part of the soul and by nature most insatiable of gain; over this they will keep guard, lest, waxing great and strong with the fulness of bodily pleasures, as they are termed, the concupiscent soul, no longer confined to her own sphere, should attempt to enslave and rule those who are not her natural-born subjects, and overturn the whole life of man?

Very true, he said.
Both together will they not be the best defenders of the whole soul and the whole body against attacks from without; the one counselling, and the other fighting under his leader, and courageously executing his commands and counsels?

True.
And he is to be deemed courageous whose spirit retains in pleasure and in pain the commands of reason about what he ought or ought not to fear?

Right, he replied.
And him we call wise who has in him that little part which rules, and which proclaims these commands; that part too being supposed to have a knowledge of what is for the interest of each of the three parts and of the whole?

Assuredly.
And would you not say that he is temperate who has these same elements in friendly harmony, in whom the one ruling principle of reason, and the two subject ones of spirit and desire are equally agreed that reason ought to rule, and do not rebel?

Certainly, he said, that is the true account of temperance whether in the State or individual.

And surely, I said, we have explained again and again how and by virtue of what quality a man will be just.

That is very certain.
And is justice dimmer in the individual, and is her form different, or is she the same which we found her to be in the State?

There is no difference in my opinion, he said.
Because, if any doubt is still lingering in our minds, a few commonplace instances will satisfy us of the truth of what I am saying.

What sort of instances do you mean?
If the case is put to us, must we not admit that the just State, or the man who is trained in the principles of such a State, will be less likely than the unjust to make away with a deposit of gold or silver? Would any one deny this?

No one, he replied.
Will the just man or citizen ever be guilty of sacrilege or theft, or treachery either to his friends or to his country?

Never.
Neither will he ever break faith where there have been oaths or agreements?

Impossible.
No one will be less likely to commit adultery, or to dishonour his father and mother, or to fall in his religious duties?

No one.
And the reason is that each part of him is doing its own business, whether in ruling or being ruled?

Exactly so.
Are you satisfied then that the quality which makes such men and such states is justice, or do you hope to discover some other?

Not I, indeed.
Then our dream has been realised; and the suspicion which we entertained at the beginning of our work of construction, that some divine power must have conducted us to a primary form of justice, has now been verified?

Yes, certainly.
And the division of labour which required the carpenter and the shoemaker and the rest of the citizens to be doing each his own business, and not another's, was a shadow of justice, and for that reason it was of use?

Clearly.
But in reality justice was such as we were describing, being concerned however, not with the outward man, but with the inward, which is the true self and concernment of man: for the just man does not permit the several elements within him to interfere with one another, or any of them to do the work of others, --he sets in order his own inner life, and is his own master and his own law, and at peace with himself; and when he has bound together the three principles within him, which may be compared to the higher, lower, and middle notes of the scale, and the intermediate intervals --when he has bound all these together, and is no longer many, but has become one entirely temperate and perfectly adjusted nature, then he proceeds to act, if he has to act, whether in a matter of property, or in the treatment of the body, or in some affair of politics or private business; always thinking and calling that which preserves and co-operates with this harmonious condition, just and good action, and the knowledge which presides over it, wisdom, and that which at any time impairs this condition, he will call unjust action, and the opinion which presides over it ignorance.

You have said the exact truth, Socrates.
Very good; and if we were to affirm that we had discovered the just man and the just State, and the nature of justice in each of them, we should not be telling a falsehood?

Most certainly not.
May we say so, then?
Let us say so.
And now, I said, injustice has to be considered.
Clearly.
Must not injustice be a strife which arises among the three principles --a meddlesomeness, and interference, and rising up of a part of the soul against the whole, an assertion of unlawful authority, which is made by a rebellious subject against a true prince, of whom he is the natural vassal, --what is all this confusion and delusion but injustice, and intemperance and cowardice and ignorance, and every form of vice?

Exactly so.
And if the nature of justice and injustice be known, then the meaning of acting unjustly and being unjust, or, again, of acting justly, will also be perfectly clear?

What do you mean? he said.
Why, I said, they are like disease and health; being in the soul just what disease and health are in the body.

How so? he said.
Why, I said, that which is healthy causes health, and that which is unhealthy causes disease.

Yes.
And just actions cause justice, and unjust actions cause injustice?
That is certain.
And the creation of health is the institution of a natural order and government of one by another in the parts of the body; and the creation of disease is the production of a state of things at variance with this natural order?

True.
And is not the creation of justice the institution of a natural order and government of one by another in the parts of the soul, and the creation of injustice the production of a state of things at variance with the natural order?

Exactly so, he said.
Then virtue is the health and beauty and well-being of the soul, and vice the disease and weakness and deformity of the same?

True.
And do not good practices lead to virtue, and evil practices to vice?

Assuredly.
Still our old question of the comparative advantage of justice and injustice has not been answered: Which is the more profitable, to be just and act justly and practise virtue, whether seen or unseen of gods and men, or to be unjust and act unjustly, if only unpunished and unreformed?

In my judgment, Socrates, the question has now become ridiculous. We know that, when the bodily constitution is gone, life is no longer endurable, though pampered with all kinds of meats and drinks, and having all wealth and all power; and shall we be told that when the very essence of the vital principle is undermined and corrupted, life is still worth having to a man, if only he be allowed to do whatever he likes with the single exception that he is not to acquire justice and virtue, or to escape from injustice and vice; assuming them both to be such as we have described?

Yes, I said, the question is, as you say, ridiculous. Still, as we are near the spot at which we may see the truth in the clearest manner with our own eyes, let us not faint by the way.

Certainly not, he replied.
Come up hither, I said, and behold the various forms of vice, those of them, I mean, which are worth looking at.

I am following you, he replied: proceed.
I said, The argument seems to have reached a height from which, as from some tower of speculation, a man may look down and see that virtue is one, but that the forms of vice are innumerable; there being four special ones which are deserving of note.

What do you mean? he said.
I mean, I replied, that there appear to be as many forms of the soul as there are distinct forms of the State.

How many?
There are five of the State, and five of the soul, I said.
What are they?
The first, I said, is that which we have been describing, and which may be said to have two names, monarchy and aristocracy, accordingly as rule is exercised by one distinguished man or by many.

True, he replied.
But I regard the two names as describing one form only; for whether the government is in the hands of one or many, if the governors have been trained in the manner which we have supposed, the fundamental laws of the State will be maintained.

That is true, he replied.