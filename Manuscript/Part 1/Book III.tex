\chapter{Book III}

Socrates - ADEIMANTUS

Such then, I said, are our principles of theology --some tales are to be told, and others are not to be told to our disciples from their youth upwards, if we mean them to honour the gods and their parents, and to value friendship with one another.

Yes; and I think that our principles are right, he said.
But if they are to be courageous, must they not learn other lessons besides these, and lessons of such a kind as will take away the fear of death? Can any man be courageous who has the fear of death in him?

Certainly not, he said.
And can he be fearless of death, or will he choose death in battle rather than defeat and slavery, who believes the world below to be real and terrible?

Impossible.
Then we must assume a control over the narrators of this class of tales as well as over the others, and beg them not simply to but rather to commend the world below, intimating to them that their descriptions are untrue, and will do harm to our future warriors.

That will be our duty, he said.
Then, I said, we shall have to obliterate many obnoxious passages, beginning with the verses,

I would rather he a serf on the land of a poor and portionless man than rule over all the dead who have come to nought. We must also expunge the verse, which tells us how Pluto feared,

Lest the mansions grim and squalid which the gods abhor should he seen both of mortals and immortals. And again:

O heavens! verily in the house of Hades there is soul and ghostly form but no mind at all! Again of Tiresias: --

[To him even after death did Persephone grant mind,] that he alone should be wise; but the other souls are flitting shades. Again: --

The soul flying from the limbs had gone to Hades, lamentng her fate, leaving manhood and youth. Again: --

And the soul, with shrilling cry, passed like smoke beneath the earth. And, --

As bats in hollow of mystic cavern, whenever any of the has dropped out of the string and falls from the rock, fly shrilling and cling to one another, so did they with shrilling cry hold together as they moved. And we must beg Homer and the other poets not to be angry if we strike out these and similar passages, not because they are unpoetical, or unattractive to the popular ear, but because the greater the poetical charm of them, the less are they meet for the ears of boys and men who are meant to be free, and who should fear slavery more than death.

Undoubtedly.
Also we shall have to reject all the terrible and appalling names describe the world below --Cocytus and Styx, ghosts under the earth, and sapless shades, and any similar words of which the very mention causes a shudder to pass through the inmost soul of him who hears them. I do not say that these horrible stories may not have a use of some kind; but there is a danger that the nerves of our guardians may be rendered too excitable and effeminate by them.

There is a real danger, he said.
Then we must have no more of them.
True.
Another and a nobler strain must be composed and sung by us.
Clearly.
And shall we proceed to get rid of the weepings and wailings of famous men?

They will go with the rest.
But shall we be right in getting rid of them? Reflect: our principle is that the good man will not consider death terrible to any other good man who is his comrade.

Yes; that is our principle.
And therefore he will not sorrow for his departed friend as though he had suffered anything terrible?

He will not.
Such an one, as we further maintain, is sufficient for himself and his own happiness, and therefore is least in need of other men.

True, he said.
And for this reason the loss of a son or brother, or the deprivation of fortune, is to him of all men least terrible.

Assuredly.
And therefore he will be least likely to lament, and will bear with the greatest equanimity any misfortune of this sort which may befall him.

Yes, he will feel such a misfortune far less than another.
Then we shall be right in getting rid of the lamentations of famous men, and making them over to women (and not even to women who are good for anything), or to men of a baser sort, that those who are being educated by us to be the defenders of their country may scorn to do the like.

That will be very right.
Then we will once more entreat Homer and the other poets not to depict Achilles, who is the son of a goddess, first lying on his side, then on his back, and then on his face; then starting up and sailing in a frenzy along the shores of the barren sea; now taking the sooty ashes in both his hands and pouring them over his head, or weeping and wailing in the various modes which Homer has delineated. Nor should he describe Priam the kinsman of the gods as praying and beseeching,

Rolling in the dirt, calling each man loudly by his name. Still more earnestly will we beg of him at all events not to introduce the gods lamenting and saying,

Alas! my misery! Alas! that I bore the harvest to my sorrow. But if he must introduce the gods, at any rate let him not dare so completely to misrepresent the greatest of the gods, as to make him say --

O heavens! with my eyes verily I behold a dear friend of mine chased round and round the city, and my heart is sorrowful. Or again: --

Woe is me that I am fated to have Sarpedon, dearest of men to me, subdued at the hands of Patroclus the son of Menoetius. For if, my sweet Adeimantus, our youth seriously listen to such unworthy representations of the gods, instead of laughing at them as they ought, hardly will any of them deem that he himself, being but a man, can be dishonoured by similar actions; neither will he rebuke any inclination which may arise in his mind to say and do the like. And instead of having any shame or self-control, he will be always whining and lamenting on slight occasions.

Yes, he said, that is most true.
Yes, I replied; but that surely is what ought not to be, as the argument has just proved to us; and by that proof we must abide until it is disproved by a better.

It ought not to be.
Neither ought our guardians to be given to laughter. For a fit of laughter which has been indulged to excess almost always produces a violent reaction.

So I believe.
Then persons of worth, even if only mortal men, must not be represented as overcome by laughter, and still less must such a representation of the gods be allowed.

Still less of the gods, as you say, he replied.
Then we shall not suffer such an expression to be used about the gods as that of Homer when he describes how

Inextinguishable laughter arose among the blessed gods, when they saw Hephaestus bustling about the mansion. On your views, we must not admit them.

On my views, if you like to father them on me; that we must not admit them is certain.

Again, truth should be highly valued; if, as we were saying, a lie is useless to the gods, and useful only as a medicine to men, then the use of such medicines should be restricted to physicians; private individuals have no business with them.

Clearly not, he said.
Then if any one at all is to have the privilege of lying, the rulers of the State should be the persons; and they, in their dealings either with enemies or with their own citizens, may be allowed to lie for the public good. But nobody else should meddle with anything of the kind; and although the rulers have this privilege, for a private man to lie to them in return is to be deemed a more heinous fault than for the patient or the pupil of a gymnasium not to speak the truth about his own bodily illnesses to the physician or to the trainer, or for a sailor not to tell the captain what is happening about the ship and the rest of the crew, and how things are going with himself or his fellow sailors.

Most true, he said.
If, then, the ruler catches anybody beside himself lying in the State,

Any of the craftsmen, whether he priest or physician or carpenter. he will punish him for introducing a practice which is equally subversive and destructive of ship or State.

Most certainly, he said, if our idea of the State is ever carried out.

In the next place our youth must be temperate?
Certainly.
Are not the chief elements of temperance, speaking generally, obedience to commanders and self-control in sensual pleasures?

True.
Then we shall approve such language as that of Diomede in Homer,

Friend, sit still and obey my word, and the verses which follow,

The Greeks marched breathing prowess,
...in silent awe of their leaders, and other sentiments of the same kind.

We shall.
What of this line,

O heavy with wine, who hast the eyes of a dog and the heart of a stag, and of the words which follow? Would you say that these, or any similar impertinences which private individuals are supposed to address to their rulers, whether in verse or prose, are well or ill spoken?

They are ill spoken.
They may very possibly afford some amusement, but they do not conduce to temperance. And therefore they are likely to do harm to our young men --you would agree with me there?

Yes.
And then, again, to make the wisest of men say that nothing in his opinion is more glorious than

When the tables are full of bread and meat, and the cup-bearer carries round wine which he draws from the bowl and pours into the cups, is it fit or conducive to temperance for a young man to hear such words? Or the verse

The saddest of fates is to die and meet destiny from hunger? What would you say again to the tale of Zeus, who, while other gods and men were asleep and he the only person awake, lay devising plans, but forgot them all in a moment through his lust, and was so completely overcome at the sight of Here that he would not even go into the hut, but wanted to lie with her on the ground, declaring that he had never been in such a state of rapture before, even when they first met one another

Without the knowledge of their parents; or that other tale of how Hephaestus, because of similar goings on, cast a chain around Ares and Aphrodite?

Indeed, he said, I am strongly of opinion that they ought not to hear that sort of thing.

But any deeds of endurance which are done or told by famous men, these they ought to see and hear; as, for example, what is said in the verses,

He smote his breast, and thus reproached his heart,
Endure, my heart; far worse hast thou endured!

Certainly, he said.
In the next place, we must not let them be receivers of gifts or lovers of money.

Certainly not.
Neither must we sing to them of

Gifts persuading gods, and persuading reverend kings. Neither is Phoenix, the tutor of Achilles, to be approved or deemed to have given his pupil good counsel when he told him that he should take the gifts of the Greeks and assist them; but that without a gift he should not lay aside his anger. Neither will we believe or acknowledge Achilles himself to have been such a lover of money that he took Agamemnon's or that when he had received payment he restored the dead body of Hector, but that without payment he was unwilling to do so.

Undoubtedly, he said, these are not sentiments which can be approved.

Loving Homer as I do, I hardly like to say that in attributing these feelings to Achilles, or in believing that they are truly to him, he is guilty of downright impiety. As little can I believe the narrative of his insolence to Apollo, where he says,

Thou hast wronged me, O far-darter, most abominable of deities. Verily I would he even with thee, if I had only the power, or his insubordination to the river-god, on whose divinity he is ready to lay hands; or his offering to the dead Patroclus of his own hair, which had been previously dedicated to the other river-god Spercheius, and that he actually performed this vow; or that he dragged Hector round the tomb of Patroclus, and slaughtered the captives at the pyre; of all this I cannot believe that he was guilty, any more than I can allow our citizens to believe that he, the wise Cheiron's pupil, the son of a goddess and of Peleus who was the gentlest of men and third in descent from Zeus, was so disordered in his wits as to be at one time the slave of two seemingly inconsistent passions, meanness, not untainted by avarice, combined with overweening contempt of gods and men.

You are quite right, he replied.
And let us equally refuse to believe, or allow to be repeated, the tale of Theseus son of Poseidon, or of Peirithous son of Zeus, going forth as they did to perpetrate a horrid rape; or of any other hero or son of a god daring to do such impious and dreadful things as they falsely ascribe to them in our day: and let us further compel the poets to declare either that these acts were not done by them, or that they were not the sons of gods; --both in the same breath they shall not be permitted to affirm. We will not have them trying to persuade our youth that the gods are the authors of evil, and that heroes are no better than men-sentiments which, as we were saying, are neither pious nor true, for we have already proved that evil cannot come from the gods.

Assuredly not.
And further they are likely to have a bad effect on those who hear them; for everybody will begin to excuse his own vices when he is convinced that similar wickednesses are always being perpetrated by --

The kindred of the gods, the relatives of Zeus, whose ancestral altar, the attar of Zeus, is aloft in air on the peak of Ida, and who have

the blood of deities yet flowing in their veins. And therefore let us put an end to such tales, lest they engender laxity of morals among the young.

By all means, he replied.
But now that we are determining what classes of subjects are or are not to be spoken of, let us see whether any have been omitted by us. The manner in which gods and demigods and heroes and the world below should be treated has been already laid down.

Very true.
And what shall we say about men? That is clearly the remaining portion of our subject.

Clearly so.
But we are not in a condition to answer this question at present, my friend.

Why not?
Because, if I am not mistaken, we shall have to say that about men poets and story-tellers are guilty of making the gravest misstatements when they tell us that wicked men are often happy, and the good miserable; and that injustice is profitable when undetected, but that justice is a man's own loss and another's gain --these things we shall forbid them to utter, and command them to sing and say the opposite.

To be sure we shall, he replied.
But if you admit that I am right in this, then I shall maintain that you have implied the principle for which we have been all along contending.

I grant the truth of your inference.
That such things are or are not to be said about men is a question which we cannot determine until we have discovered what justice is, and how naturally advantageous to the possessor, whether he seems to be just or not.

Most true, he said.
Enough of the subjects of poetry: let us now speak of the style; and when this has been considered, both matter and manner will have been completely treated.

I do not understand what you mean, said Adeimantus.
Then I must make you understand; and perhaps I may be more intelligible if I put the matter in this way. You are aware, I suppose, that all mythology and poetry is a narration of events, either past, present, or to come?

Certainly, he replied.
And narration may be either simple narration, or imitation, or a union of the two?

That again, he said, I do not quite understand.
I fear that I must be a ridiculous teacher when I have so much difficulty in making myself apprehended. Like a bad speaker, therefore, I will not take the whole of the subject, but will break a piece off in illustration of my meaning. You know the first lines of the Iliad, in which the poet says that Chryses prayed Agamemnon to release his daughter, and that Agamemnon flew into a passion with him; whereupon Chryses, failing of his object, invoked the anger of the God against the Achaeans. Now as far as these lines,

And he prayed all the Greeks, but especially the two sons of Atreus, the chiefs of the people, the poet is speaking in his own person; he never leads us to suppose that he is any one else. But in what follows he takes the person of Chryses, and then he does all that he can to make us believe that the speaker is not Homer, but the aged priest himself. And in this double form he has cast the entire narrative of the events which occurred at Troy and in Ithaca and throughout the Odyssey.

Yes.
And a narrative it remains both in the speeches which the poet recites from time to time and in the intermediate passages?

Quite true.
But when the poet speaks in the person of another, may we not say that he assimilates his style to that of the person who, as he informs you, is going to speak?

Certainly.
And this assimilation of himself to another, either by the use of voice or gesture, is the imitation of the person whose character he assumes?

Of course.
Then in this case the narrative of the poet may be said to proceed by way of imitation?

Very true.
Or, if the poet everywhere appears and never conceals himself, then again the imitation is dropped, and his poetry becomes simple narration. However, in order that I may make my meaning quite clear, and that you may no more say, I don't understand,' I will show how the change might be effected. If Homer had said, 'The priest came, having his daughter's ransom in his hands, supplicating the Achaeans, and above all the kings;' and then if, instead of speaking in the person of Chryses, he had continued in his own person, the words would have been, not imitation, but simple narration. The passage would have run as follows (I am no poet, and therefore I drop the metre), 'The priest came and prayed the gods on behalf of the Greeks that they might capture Troy and return safely home, but begged that they would give him back his daughter, and take the ransom which he brought, and respect the God. Thus he spoke, and the other Greeks revered the priest and assented. But Agamemnon was wroth, and bade him depart and not come again, lest the staff and chaplets of the God should be of no avail to him --the daughter of Chryses should not be released, he said --she should grow old with him in Argos. And then he told him to go away and not to provoke him, if he intended to get home unscathed. And the old man went away in fear and silence, and, when he had left the camp, he called upon Apollo by his many names, reminding him of everything which he had done pleasing to him, whether in building his temples, or in offering sacrifice, and praying that his good deeds might be returned to him, and that the Achaeans might expiate his tears by the arrows of the god,' --and so on. In this way the whole becomes simple narrative.

I understand, he said.
Or you may suppose the opposite case --that the intermediate passages are omitted, and the dialogue only left.

That also, he said, I understand; you mean, for example, as in tragedy.

You have conceived my meaning perfectly; and if I mistake not, what you failed to apprehend before is now made clear to you, that poetry and mythology are, in some cases, wholly imitative --instances of this are supplied by tragedy and comedy; there is likewise the opposite style, in which the my poet is the only speaker --of this the dithyramb affords the best example; and the combination of both is found in epic, and in several other styles of poetry. Do I take you with me?

Yes, he said; I see now what you meant.
I will ask you to remember also what I began by saying, that we had done with the subject and might proceed to the style.

Yes, I remember.
In saying this, I intended to imply that we must come to an understanding about the mimetic art, --whether the poets, in narrating their stories, are to be allowed by us to imitate, and if so, whether in whole or in part, and if the latter, in what parts; or should all imitation be prohibited?

You mean, I suspect, to ask whether tragedy and comedy shall be admitted into our State?

Yes, I said; but there may be more than this in question: I really do not know as yet, but whither the argument may blow, thither we go.

And go we will, he said.
Then, Adeimantus, let me ask you whether our guardians ought to be imitators; or rather, has not this question been decided by the rule already laid down that one man can only do one thing well, and not many; and that if he attempt many, he will altogether fall of gaining much reputation in any?

Certainly.
And this is equally true of imitation; no one man can imitate many things as well as he would imitate a single one?

He cannot.
Then the same person will hardly be able to play a serious part in life, and at the same time to be an imitator and imitate many other parts as well; for even when two species of imitation are nearly allied, the same persons cannot succeed in both, as, for example, the writers of tragedy and comedy --did you not just now call them imitations?

Yes, I did; and you are right in thinking that the same persons cannot succeed in both.

Any more than they can be rhapsodists and actors at once?
True.
Neither are comic and tragic actors the same; yet all these things are but imitations.

They are so.
And human nature, Adeimantus, appears to have been coined into yet smaller pieces, and to be as incapable of imitating many things well, as of performing well the actions of which the imitations are copies.

Quite true, he replied.
If then we adhere to our original notion and bear in mind that our guardians, setting aside every other business, are to dedicate themselves wholly to the maintenance of freedom in the State, making this their craft, and engaging in no work which does not bear on this end, they ought not to practise or imitate anything else; if they imitate at all, they should imitate from youth upward only those characters which are suitable to their profession --the courageous, temperate, holy, free, and the like; but they should not depict or be skilful at imitating any kind of illiberality or baseness, lest from imitation they should come to be what they imitate. Did you never observe how imitations, beginning in early youth and continuing far into life, at length grow into habits and become a second nature, affecting body, voice, and mind?

Yes, certainly, he said.
Then, I said, we will not allow those for whom we profess a care and of whom we say that they ought to be good men, to imitate a woman, whether young or old, quarrelling with her husband, or striving and vaunting against the gods in conceit of her happiness, or when she is in affliction, or sorrow, or weeping; and certainly not one who is in sickness, love, or labour.

Very right, he said.
Neither must they represent slaves, male or female, performing the offices of slaves?

They must not.
And surely not bad men, whether cowards or any others, who do the reverse of what we have just been prescribing, who scold or mock or revile one another in drink or out of in drink or, or who in any other manner sin against themselves and their neighbours in word or deed, as the manner of such is. Neither should they be trained to imitate the action or speech of men or women who are mad or bad; for madness, like vice, is to be known but not to be practised or imitated.

Very true, he replied.
Neither may they imitate smiths or other artificers, or oarsmen, or boatswains, or the like?

How can they, he said, when they are not allowed to apply their minds to the callings of any of these?

Nor may they imitate the neighing of horses, the bellowing of bulls, the murmur of rivers and roll of the ocean, thunder, and all that sort of thing?

Nay, he said, if madness be forbidden, neither may they copy the behaviour of madmen.

You mean, I said, if I understand you aright, that there is one sort of narrative style which may be employed by a truly good man when he has anything to say, and that another sort will be used by a man of an opposite character and education.

And which are these two sorts? he asked.
Suppose, I answered, that a just and good man in the course of a narration comes on some saying or action of another good man, --I should imagine that he will like to personate him, and will not be ashamed of this sort of imitation: he will be most ready to play the part of the good man when he is acting firmly and wisely; in a less degree when he is overtaken by illness or love or drink, or has met with any other disaster. But when he comes to a character which is unworthy of him, he will not make a study of that; he will disdain such a person, and will assume his likeness, if at all, for a moment only when he is performing some good action; at other times he will be ashamed to play a part which he has never practised, nor will he like to fashion and frame himself after the baser models; he feels the employment of such an art, unless in jest, to be beneath him, and his mind revolts at it.

So I should expect, he replied.
Then he will adopt a mode of narration such as we have illustrated out of Homer, that is to say, his style will be both imitative and narrative; but there will be very little of the former, and a great deal of the latter. Do you agree?

Certainly, he said; that is the model which such a speaker must necessarily take.

But there is another sort of character who will narrate anything, and, the worse lie is, the more unscrupulous he will be; nothing will be too bad for him: and he will be ready to imitate anything, not as a joke, but in right good earnest, and before a large company. As I was just now saying, he will attempt to represent the roll of thunder, the noise of wind and hall, or the creaking of wheels, and pulleys, and the various sounds of flutes; pipes, trumpets, and all sorts of instruments: he will bark like a dog, bleat like a sheep, or crow like a cock; his entire art will consist in imitation of voice and gesture, and there will be very little narration.

That, he said, will be his mode of speaking.
These, then, are the two kinds of style?
Yes.
And you would agree with me in saying that one of them is simple and has but slight changes; and if the harmony and rhythm are also chosen for their simplicity, the result is that the speaker, if hc speaks correctly, is always pretty much the same in style, and he will keep within the limits of a single harmony (for the changes are not great), and in like manner he will make use of nearly the same rhythm?

That is quite true, he said.
Whereas the other requires all sorts of harmonies and all sorts of rhythms, if the music and the style are to correspond, because the style has all sorts of changes.

That is also perfectly true, he replied.
And do not the two styles, or the mixture of the two, comprehend all poetry, and every form of expression in words? No one can say anything except in one or other of them or in both together.

They include all, he said.
And shall we receive into our State all the three styles, or one only of the two unmixed styles? or would you include the mixed?

I should prefer only to admit the pure imitator of virtue.
Yes, I said, Adeimantus, but the mixed style is also very charming: and indeed the pantomimic, which is the opposite of the one chosen by you, is the most popular style with children and their attendants, and with the world in general.

I do not deny it.
But I suppose you would argue that such a style is unsuitable to our State, in which human nature is not twofold or manifold, for one man plays one part only?

Yes; quite unsuitable.
And this is the reason why in our State, and in our State only, we shall find a shoemaker to be a shoemaker and not a pilot also, and a husbandman to be a husbandman and not a dicast also, and a soldier a soldier and not a trader also, and the same throughout?

True, he said.
And therefore when any one of these pantomimic gentlemen, who are so clever that they can imitate anything, comes to us, and makes a proposal to exhibit himself and his poetry, we will fall down and worship him as a sweet and holy and wonderful being; but we must also inform him that in our State such as he are not permitted to exist; the law will not allow them. And so when we have anointed him with myrrh, and set a garland of wool upon his head, we shall send him away to another city. For we mean to employ for our souls' health the rougher and severer poet or story-teller, who will imitate the style of the virtuous only, and will follow those models which we prescribed at first when we began the education of our soldiers.

We certainly will, he said, if we have the power.
Then now, my friend, I said, that part of music or literary education which relates to the story or myth may be considered to be finished; for the matter and manner have both been discussed.

I think so too, he said.
Next in order will follow melody and song.
That is obvious.
Every one can see already what we ought to say about them, if we are to be consistent with ourselves.

Socrates - GLAUCON

I fear, said Glaucon, laughing, that the words 'every one' hardly includes me, for I cannot at the moment say what they should be; though I may guess.

At any rate you can tell that a song or ode has three parts --the words, the melody, and the rhythm; that degree of knowledge I may presuppose?

Yes, he said; so much as that you may.
And as for the words, there surely be no difference words between words which are and which are not set to music; both will conform to the same laws, and these have been already determined by us?

Yes.
And the melody and rhythm will depend upon the words?
Certainly.
We were saying, when we spoke of the subject-matter, that we had no need of lamentations and strains of sorrow?

True.
And which are the harmonies expressive of sorrow? You are musical, and can tell me.

The harmonies which you mean are the mixed or tenor Lydian, and the full-toned or bass Lydian, and such like.

These then, I said, must be banished; even to women who have a character to maintain they are of no use, and much less to men. Certainly.

In the next place, drunkenness and softness and indolence are utterly unbecoming the character of our guardians.

Utterly unbecoming.
And which are the soft or drinking harmonies?
The Ionian, he replied, and the Lydian; they are termed 'relaxed.'
Well, and are these of any military use?
Quite the reverse, he replied; and if so the Dorian and the Phrygian are the only ones which you have left.

I answered: Of the harmonies I know nothing, but I want to have one warlike, to sound the note or accent which a brave man utters in the hour of danger and stern resolve, or when his cause is failing, and he is going to wounds or death or is overtaken by some other evil, and at every such crisis meets the blows of fortune with firm step and a determination to endure; and another to be used by him in times of peace and freedom of action, when there is no pressure of necessity, and he is seeking to persuade God by prayer, or man by instruction and admonition, or on the other hand, when he is expressing his willingness to yield to persuasion or entreaty or admonition, and which represents him when by prudent conduct he has attained his end, not carried away by his success, but acting moderately and wisely under the circumstances, and acquiescing in the event. These two harmonies I ask you to leave; the strain of necessity and the strain of freedom, the strain of the unfortunate and the strain of the fortunate, the strain of courage, and the strain of temperance; these, I say, leave.

And these, he replied, are the Dorian and Phrygian harmonies of which I was just now speaking.

Then, I said, if these and these only are to be used in our songs and melodies, we shall not want multiplicity of notes or a panharmonic scale?

I suppose not.
Then we shall not maintain the artificers of lyres with three corners and complex scales, or the makers of any other many-stringed curiously-harmonised instruments?

Certainly not.
But what do you say to flute-makers and flute-players? Would you admit them into our State when you reflect that in this composite use of harmony the flute is worse than all the stringed instruments put together; even the panharmonic music is only an imitation of the flute?

Clearly not.
There remain then only the lyre and the harp for use in the city, and the shepherds may have a pipe in the country.

That is surely the conclusion to be drawn from the argument.
The preferring of Apollo and his instruments to Marsyas and his instruments is not at all strange, I said.

Not at all, he replied.
And so, by the dog of Egypt, we have been unconsciously purging the State, which not long ago we termed luxurious.

And we have done wisely, he replied.
Then let us now finish the purgation, I said. Next in order to harmonies, rhythms will naturally follow, and they should be subject to the same rules, for we ought not to seek out complex systems of metre, or metres of every kind, but rather to discover what rhythms are the expressions of a courageous and harmonious life; and when we have found them, we shall adapt the foot and the melody to words having a like spirit, not the words to the foot and melody. To say what these rhythms are will be your duty --you must teach me them, as you have already taught me the harmonies.

But, indeed, he replied, I cannot tell you. I only know that there are some three principles of rhythm out of which metrical systems are framed, just as in sounds there are four notes out of which all the harmonies are composed; that is an observation which I have made. But of what sort of lives they are severally the imitations I am unable to say.

Then, I said, we must take Damon into our counsels; and he will tell us what rhythms are expressive of meanness, or insolence, or fury, or other unworthiness, and what are to be reserved for the expression of opposite feelings. And I think that I have an indistinct recollection of his mentioning a complex Cretic rhythm; also a dactylic or heroic, and he arranged them in some manner which I do not quite understand, making the rhythms equal in the rise and fall of the foot, long and short alternating; and, unless I am mistaken, he spoke of an iambic as well as of a trochaic rhythm, and assigned to them short and long quantities. Also in some cases he appeared to praise or censure the movement of the foot quite as much as the rhythm; or perhaps a combination of the two; for I am not certain what he meant. These matters, however, as I was saying, had better be referred to Damon himself, for the analysis of the subject would be difficult, you know.

Rather so, I should say.
But there is no difficulty in seeing that grace or the absence of grace is an effect of good or bad rhythm.

None at all.
And also that good and bad rhythm naturally assimilate to a good and bad style; and that harmony and discord in like manner follow style; for our principle is that rhythm and harmony are regulated by the words, and not the words by them.

Just so, he said, they should follow the words.
And will not the words and the character of the style depend on the temper of the soul?

Yes.
And everything else on the style?
Yes.
Then beauty of style and harmony and grace and good rhythm depend on simplicity, --I mean the true simplicity of a rightly and nobly ordered mind and character, not that other simplicity which is only an euphemism for folly?

Very true, he replied.
And if our youth are to do their work in life, must they not make these graces and harmonies their perpetual aim?

They must.
And surely the art of the painter and every other creative and constructive art are full of them, --weaving, embroidery, architecture, and every kind of manufacture; also nature, animal and vegetable, --in all of them there is grace or the absence of grace. And ugliness and discord and inharmonious motion are nearly allied to ill words and ill nature, as grace and harmony are the twin sisters of goodness and virtue and bear their likeness.

That is quite true, he said.
But shall our superintendence go no further, and are the poets only to be required by us to express the image of the good in their works, on pain, if they do anything else, of expulsion from our State? Or is the same control to be extended to other artists, and are they also to be prohibited from exhibiting the opposite forms of vice and intemperance and meanness and indecency in sculpture and building and the other creative arts; and is he who cannot conform to this rule of ours to be prevented from practising his art in our State, lest the taste of our citizens be corrupted by him? We would not have our guardians grow up amid images of moral deformity, as in some noxious pasture, and there browse and feed upon many a baneful herb and flower day by day, little by little, until they silently gather a festering mass of corruption in their own soul. Let our artists rather be those who are gifted to discern the true nature of the beautiful and graceful; then will our youth dwell in a land of health, amid fair sights and sounds, and receive the good in everything; and beauty, the effluence of fair works, shall flow into the eye and ear, like a health-giving breeze from a purer region, and insensibly draw the soul from earliest years into likeness and sympathy with the beauty of reason.

There can be no nobler training than that, he replied.
And therefore, I said, Glaucon, musical training is a more potent instrument than any other, because rhythm and harmony find their way into the inward places of the soul, on which they mightily fasten, imparting grace, and making the soul of him who is rightly educated graceful, or of him who is ill-educated ungraceful; and also because he who has received this true education of the inner being will most shrewdly perceive omissions or faults in art and nature, and with a true taste, while he praises and rejoices over and receives into his soul the good, and becomes noble and good, he will justly blame and hate the bad, now in the days of his youth, even before he is able to know the reason why; and when reason comes he will recognise and salute the friend with whom his education has made him long familiar.

Yes, he said, I quite agree with you in thinking that our youth should be trained in music and on the grounds which you mention.

Just as in learning to read, I said, we were satisfied when we knew the letters of the alphabet, which are very few, in all their recurring sizes and combinations; not slighting them as unimportant whether they occupy a space large or small, but everywhere eager to make them out; and not thinking ourselves perfect in the art of reading until we recognise them wherever they are found:

True --
Or, as we recognise the reflection of letters in the water, or in a mirror, only when we know the letters themselves; the same art and study giving us the knowledge of both:

Exactly --
Even so, as I maintain, neither we nor our guardians, whom we have to educate, can ever become musical until we and they know the essential forms, in all their combinations, and can recognise them and their images wherever they are found, not slighting them either in small things or great, but believing them all to be within the sphere of one art and study.

Most assuredly.
And when a beautiful soul harmonises with a beautiful form, and the two are cast in one mould, that will be the fairest of sights to him who has an eye to see it?

The fairest indeed.
And the fairest is also the loveliest?
That may be assumed.
And the man who has the spirit of harmony will be most in love with the loveliest; but he will not love him who is of an inharmonious soul?

That is true, he replied, if the deficiency be in his soul; but if there be any merely bodily defect in another he will be patient of it, and will love all the same.

I perceive, I said, that you have or have had experiences of this sort, and I agree. But let me ask you another question: Has excess of pleasure any affinity to temperance?

How can that be? he replied; pleasure deprives a man of the use of his faculties quite as much as pain.

Or any affinity to virtue in general?
None whatever.
Any affinity to wantonness and intemperance?
Yes, the greatest.
And is there any greater or keener pleasure than that of sensual love?

No, nor a madder.
Whereas true love is a love of beauty and order --temperate and harmonious?

Quite true, he said.
Then no intemperance or madness should be allowed to approach true love?

Certainly not.
Then mad or intemperate pleasure must never be allowed to come near the lover and his beloved; neither of them can have any part in it if their love is of the right sort?

No, indeed, Socrates, it must never come near them.
Then I suppose that in the city which we are founding you would make a law to the effect that a friend should use no other familiarity to his love than a father would use to his son, and then only for a noble purpose, and he must first have the other's consent; and this rule is to limit him in all his intercourse, and he is never to be seen going further, or, if he exceeds, he is to be deemed guilty of coarseness and bad taste.

I quite agree, he said.
Thus much of music, which makes a fair ending; for what should be the end of music if not the love of beauty?

I agree, he said.
After music comes gymnastic, in which our youth are next to be trained.

Certainly.
Gymnastic as well as music should begin in early years; the training in it should be careful and should continue through life. Now my belief is, --and this is a matter upon which I should like to have your opinion in confirmation of my own, but my own belief is, --not that the good body by any bodily excellence improves the soul, but, on the contrary, that the good soul, by her own excellence, improves the body as far as this may be possible. What do you say?

Yes, I agree.
Then, to the mind when adequately trained, we shall be right in handing over the more particular care of the body; and in order to avoid prolixity we will now only give the general outlines of the subject.

Very good.
That they must abstain from intoxication has been already remarked by us; for of all persons a guardian should be the last to get drunk and not know where in the world he is.

Yes, he said; that a guardian should require another guardian to take care of him is ridiculous indeed.

But next, what shall we say of their food; for the men are in training for the great contest of all --are they not?

Yes, he said.
And will the habit of body of our ordinary athletes be suited to them?

Why not?
I am afraid, I said, that a habit of body such as they have is but a sleepy sort of thing, and rather perilous to health. Do you not observe that these athletes sleep away their lives, and are liable to most dangerous illnesses if they depart, in ever so slight a degree, from their customary regimen?

Yes, I do.
Then, I said, a finer sort of training will be required for our warrior athletes, who are to be like wakeful dogs, and to see and hear with the utmost keenness; amid the many changes of water and also of food, of summer heat and winter cold, which they will have to endure when on a campaign, they must not be liable to break down in health.

That is my view.
The really excellent gymnastic is twin sister of that simple music which we were just now describing.

How so?
Why, I conceive that there is a gymnastic which, like our music, is simple and good; and especially the military gymnastic.

What do you mean?
My meaning may be learned from Homer; he, you know, feeds his heroes at their feasts, when they are campaigning, on soldiers' fare; they have no fish, although they are on the shores of the Hellespont, and they are not allowed boiled meats but only roast, which is the food most convenient for soldiers, requiring only that they should light a fire, and not involving the trouble of carrying about pots and pans.

True.
And I can hardly be mistaken in saying that sweet sauces are nowhere mentioned in Homer. In proscribing them, however, he is not singular; all professional athletes are well aware that a man who is to be in good condition should take nothing of the kind.

Yes, he said; and knowing this, they are quite right in not taking them.

Then you would not approve of Syracusan dinners, and the refinements of Sicilian cookery?

I think not.
Nor, if a man is to be in condition, would you allow him to have a Corinthian girl as his fair friend?

Certainly not.
Neither would you approve of the delicacies, as they are thought, of Athenian confectionery?

Certainly not.
All such feeding and living may be rightly compared by us to melody and song composed in the panharmonic style, and in all the rhythms. Exactly.

There complexity engendered license, and here disease; whereas simplicity in music was the parent of temperance in the soul; and simplicity in gymnastic of health in the body.

Most true, he said.
But when intemperance and disease multiply in a State, halls of justice and medicine are always being opened; and the arts of the doctor and the lawyer give themselves airs, finding how keen is the interest which not only the slaves but the freemen of a city take about them.

Of course.
And yet what greater proof can there be of a bad and disgraceful state of education than this, that not only artisans and the meaner sort of people need the skill of first-rate physicians and judges, but also those who would profess to have had a liberal education? Is it not disgraceful, and a great sign of want of good-breeding, that a man should have to go abroad for his law and physic because he has none of his own at home, and must therefore surrender himself into the hands of other men whom he makes lords and judges over him?

Of all things, he said, the most disgraceful.
Would you say 'most,' I replied, when you consider that there is a further stage of the evil in which a man is not only a life-long litigant, passing all his days in the courts, either as plaintiff or defendant, but is actually led by his bad taste to pride himself on his litigiousness; he imagines that he is a master in dishonesty; able to take every crooked turn, and wriggle into and out of every hole, bending like a withy and getting out of the way of justice: and all for what? --in order to gain small points not worth mentioning, he not knowing that so to order his life as to be able to do without a napping judge is a far higher and nobler sort of thing. Is not that still more disgraceful?

Yes, he said, that is still more disgraceful.
Well, I said, and to require the help of medicine, not when a wound has to be cured, or on occasion of an epidemic, but just because, by indolence and a habit of life such as we have been describing, men fill themselves with waters and winds, as if their bodies were a marsh, compelling the ingenious sons of Asclepius to find more names for diseases, such as flatulence and catarrh; is not this, too, a disgrace?

Yes, he said, they do certainly give very strange and newfangled names to diseases.

Yes, I said, and I do not believe that there were any such diseases in the days of Asclepius; and this I infer from the circumstance that the hero Eurypylus, after he has been wounded in Homer, drinks a posset of Pramnian wine well besprinkled with barley-meal and grated cheese, which are certainly inflammatory, and yet the sons of Asclepius who were at the Trojan war do not blame the damsel who gives him the drink, or rebuke Patroclus, who is treating his case.

Well, he said, that was surely an extraordinary drink to be given to a person in his condition.

Not so extraordinary, I replied, if you bear in mind that in former days, as is commonly said, before the time of Herodicus, the guild of Asclepius did not practise our present system of medicine, which may be said to educate diseases. But Herodicus, being a trainer, and himself of a sickly constitution, by a combination of training and doctoring found out a way of torturing first and chiefly himself, and secondly the rest of the world.

How was that? he said.
By the invention of lingering death; for he had a mortal disease which he perpetually tended, and as recovery was out of the question, he passed his entire life as a valetudinarian; he could do nothing but attend upon himself, and he was in constant torment whenever he departed in anything from his usual regimen, and so dying hard, by the help of science he struggled on to old age.

A rare reward of his skill!
Yes, I said; a reward which a man might fairly expect who never understood that, if Asclepius did not instruct his descendants in valetudinarian arts, the omission arose, not from ignorance or inexperience of such a branch of medicine, but because he knew that in all well-ordered states every individual has an occupation to which he must attend, and has therefore no leisure to spend in continually being ill. This we remark in the case of the artisan, but, ludicrously enough, do not apply the same rule to people of the richer sort.

How do you mean? he said.
I mean this: When a carpenter is ill he asks the physician for a rough and ready cure; an emetic or a purge or a cautery or the knife, --these are his remedies. And if some one prescribes for him a course of dietetics, and tells him that he must swathe and swaddle his head, and all that sort of thing, he replies at once that he has no time to be ill, and that he sees no good in a life which is spent in nursing his disease to the neglect of his customary employment; and therefore bidding good-bye to this sort of physician, he resumes his ordinary habits, and either gets well and lives and does his business, or, if his constitution falls, he dies and has no more trouble.

Yes, he said, and a man in his condition of life ought to use the art of medicine thus far only.

Has he not, I said, an occupation; and what profit would there be in his life if he were deprived of his occupation?

Quite true, he said.
But with the rich man this is otherwise; of him we do not say that he has any specially appointed work which he must perform, if he would live.

He is generally supposed to have nothing to do.
Then you never heard of the saying of Phocylides, that as soon as a man has a livelihood he should practise virtue?

Nay, he said, I think that he had better begin somewhat sooner.
Let us not have a dispute with him about this, I said; but rather ask ourselves: Is the practice of virtue obligatory on the rich man, or can he live without it? And if obligatory on him, then let us raise a further question, whether this dieting of disorders which is an impediment to the application of the mind t in carpentering and the mechanical arts, does not equally stand in the way of the sentiment of Phocylides?

Of that, he replied, there can be no doubt; such excessive care of the body, when carried beyond the rules of gymnastic, is most inimical to the practice of virtue.

Yes, indeed, I replied, and equally incompatible with the management of a house, an army, or an office of state; and, what is most important of all, irreconcilable with any kind of study or thought or self-reflection --there is a constant suspicion that headache and giddiness are to be ascribed to philosophy, and hence all practising or making trial of virtue in the higher sense is absolutely stopped; for a man is always fancying that he is being made ill, and is in constant anxiety about the state of his body.

Yes, likely enough.
And therefore our politic Asclepius may be supposed to have exhibited the power of his art only to persons who, being generally of healthy constitution and habits of life, had a definite ailment; such as these he cured by purges and operations, and bade them live as usual, herein consulting the interests of the State; but bodies which disease had penetrated through and through he would not have attempted to cure by gradual processes of evacuation and infusion: he did not want to lengthen out good-for-nothing lives, or to have weak fathers begetting weaker sons; --if a man was not able to live in the ordinary way he had no business to cure him; for such a cure would have been of no use either to himself, or to the State.

Then, he said, you regard Asclepius as a statesman.
Clearly; and his character is further illustrated by his sons. Note that they were heroes in the days of old and practised the medicines of which I am speaking at the siege of Troy: You will remember how, when Pandarus wounded Menelaus, they

Sucked the blood out of the wound, and sprinkled soothing remedies, but they never prescribed what the patient was afterwards to eat or drink in the case of Menelaus, any more than in the case of Eurypylus; the remedies, as they conceived, were enough to heal any man who before he was wounded was healthy and regular in habits; and even though he did happen to drink a posset of Pramnian wine, he might get well all the same. But they would have nothing to do with unhealthy and intemperate subjects, whose lives were of no use either to themselves or others; the art of medicine was not designed for their good, and though they were as rich as Midas, the sons of Asclepius would have declined to attend them.

They were very acute persons, those sons of Asclepius.
Naturally so, I replied. Nevertheless, the tragedians and Pindar disobeying our behests, although they acknowledge that Asclepius was the son of Apollo, say also that he was bribed into healing a rich man who was at the point of death, and for this reason he was struck by lightning. But we, in accordance with the principle already affirmed by us, will not believe them when they tell us both; --if he was the son of a god, we maintain that hd was not avaricious; or, if he was avaricious he was not the son of a god.

All that, Socrates, is excellent; but I should like to put a question to you: Ought there not to be good physicians in a State, and are not the best those who have treated the greatest number of constitutions good and bad? and are not the best judges in like manner those who are acquainted with all sorts of moral natures?

Yes, I said, I too would have good judges and good physicians. But do you know whom I think good?

Will you tell me?
I will, if I can. Let me however note that in the same question you join two things which are not the same.

How so? he asked.
Why, I said, you join physicians and judges. Now the most skilful physicians are those who, from their youth upwards, have combined with the knowledge of their art the greatest experience of disease; they had better not be robust in health, and should have had all manner of diseases in their own persons. For the body, as I conceive, is not the instrument with which they cure the body; in that case we could not allow them ever to be or to have been sickly; but they cure the body with the mind, and the mind which has become and is sick can cure nothing.

That is very true, he said.
But with the judge it is otherwise; since he governs mind by mind; he ought not therefore to have been trained among vicious minds, and to have associated with them from youth upwards, and to have gone through the whole calendar of crime, only in order that he may quickly infer the crimes of others as he might their bodily diseases from his own self-consciousness; the honourable mind which is to form a healthy judgment should have had no experience or contamination of evil habits when young. And this is the reason why in youth good men often appear to be simple, and are easily practised upon by the dishonest, because they have no examples of what evil is in their own souls.

Yes, he said, they are far too apt to be deceived.
Therefore, I said, the judge should not be young; he should have learned to know evil, not from his own soul, but from late and long observation of the nature of evil in others: knowledge should be his guide, not personal experience.

Yes, he said, that is the ideal of a judge.
Yes, I replied, and he will be a good man (which is my answer to your question); for he is good who has a good soul. But the cunning and suspicious nature of which we spoke, --he who has committed many crimes, and fancies himself to be a master in wickedness, when he is amongst his fellows, is wonderful in the precautions which he takes, because he judges of them by himself: but when he gets into the company of men of virtue, who have the experience of age, he appears to be a fool again, owing to his unseasonable suspicions; he cannot recognise an honest man, because he has no pattern of honesty in himself; at the same time, as the bad are more numerous than the good, and he meets with them oftener, he thinks himself, and is by others thought to be, rather wise than foolish.

Most true, he said.
Then the good and wise judge whom we are seeking is not this man, but the other; for vice cannot know virtue too, but a virtuous nature, educated by time, will acquire a knowledge both of virtue and vice: the virtuous, and not the vicious, man has wisdom --in my opinion.

And in mine also.
This is the sort of medicine, and this is the sort of law, which you sanction in your State. They will minister to better natures, giving health both of soul and of body; but those who are diseased in their bodies they will leave to die, and the corrupt and incurable souls they will put an end to themselves.

That is clearly the best thing both for the patients and for the State.

And thus our youth, having been educated only in that simple music which, as we said, inspires temperance, will be reluctant to go to law.

Clearly.
And the musician, who, keeping to the same track, is content to practise the simple gymnastic, will have nothing to do with medicine unless in some extreme case.

That I quite believe.
The very exercises and tolls which he undergoes are intended to stimulate the spirited element of his nature, and not to increase his strength; he will not, like common athletes, use exercise and regimen to develop his muscles.

Very right, he said.
Neither are the two arts of music and gymnastic really designed, as is often supposed, the one for the training of the soul, the other fir the training of the body.

What then is the real object of them?
I believe, I said, that the teachers of both have in view chiefly the improvement of the soul.

How can that be? he asked.
Did you never observe, I said, the effect on the mind itself of exclusive devotion to gymnastic, or the opposite effect of an exclusive devotion to music?

In what way shown? he said.
The one producing a temper of hardness and ferocity, the other of softness and effeminacy, I replied.

Yes, he said, I am quite aware that the mere athlete becomes too much of a savage, and that the mere musician is melted and softened beyond what is good for him.

Yet surely, I said, this ferocity only comes from spirit, which, if rightly educated, would give courage, but, if too much intensified, is liable to become hard and brutal.

That I quite think.
On the other hand the philosopher will have the quality of gentleness. And this also, when too much indulged, will turn to softness, but, if educated rightly, will be gentle and moderate.

True.
And in our opinion the guardians ought to have both these qualities?
Assuredly.
And both should be in harmony?
Beyond question.
And the harmonious soul is both temperate and courageous?
Yes.
And the inharmonious is cowardly and boorish?
Very true.
And, when a man allows music to play upon him and to pour into his soul through the funnel of his ears those sweet and soft and melancholy airs of which we were just now speaking, and his whole life is passed in warbling and the delights of song; in the first stage of the process the passion or spirit which is in him is tempered like iron, and made useful, instead of brittle and useless. But, if he carries on the softening and soothing process, in the next stage he begins to melt and waste, until he has wasted away his spirit and cut out the sinews of his soul; and he becomes a feeble warrior.

Very true.
If the element of spirit is naturally weak in him the change is speedily accomplished, but if he have a good deal, then the power of music weakening the spirit renders him excitable; --on the least provocation he flames up at once, and is speedily extinguished; instead of having spirit he grows irritable and passionate and is quite impracticable.

Exactly.
And so in gymnastics, if a man takes violent exercise and is a great feeder, and the reverse of a great student of music and philosophy, at first the high condition of his body fills him with pride and spirit, and lie becomes twice the man that he was.

Certainly.
And what happens? if he do nothing else, and holds no con-a verse with the Muses, does not even that intelligence which there may be in him, having no taste of any sort of learning or enquiry or thought or culture, grow feeble and dull and blind, his mind never waking up or receiving nourishment, and his senses not being purged of their mists?

True, he said.
And he ends by becoming a hater of philosophy, uncivilized, never using the weapon of persuasion, --he is like a wild beast, all violence and fierceness, and knows no other way of dealing; and he lives in all ignorance and evil conditions, and has no sense of propriety and grace.

That is quite true, he said.
And as there are two principles of human nature, one the spirited and the other the philosophical, some God, as I should say, has given mankind two arts answering to them (and only indirectly to the soul and body), in order that these two principles (like the strings of an instrument) may be relaxed or drawn tighter until they are duly harmonised.

That appears to be the intention.
And he who mingles music with gymnastic in the fairest proportions, and best attempers them to the soul, may be rightly called the true musician and harmonist in a far higher sense than the tuner of the strings.

You are quite right, Socrates.
And such a presiding genius will be always required in our State if the government is to last.

Yes, he will be absolutely necessary.
Such, then, are our principles of nurture and education: Where would be the use of going into further details about the dances of our citizens, or about their hunting and coursing, their gymnastic and equestrian contests? For these all follow the general principle, and having found that, we shall have no difficulty in discovering them.

I dare say that there will be no difficulty.
Very good, I said; then what is the next question? Must we not ask who are to be rulers and who subjects?

Certainly.
There can be no doubt that the elder must rule the younger.
Clearly.
And that the best of these must rule.
That is also clear.
Now, are not the best husbandmen those who are most devoted to husbandry?

Yes.
And as we are to have the best of guardians for our city, must they not be those who have most the character of guardians?

Yes.
And to this end they ought to be wise and efficient, and to have a special care of the State?

True.
And a man will be most likely to care about that which he loves?
To be sure.
And he will be most likely to love that which he regards as having the same interests with himself, and that of which the good or evil fortune is supposed by him at any time most to affect his own?

Very true, he replied.
Then there must be a selection. Let us note among the guardians those who in their whole life show the greatest eagerness to do what is for the good of their country, and the greatest repugnance to do what is against her interests.

Those are the right men.
And they will have to be watched at every age, in order that we may see whether they preserve their resolution, and never, under the influence either of force or enchantment, forget or cast off their sense of duty to the State.

How cast off? he said.
I will explain to you, I replied. A resolution may go out of a man's mind either with his will or against his will; with his will when he gets rid of a falsehood and learns better, against his will whenever he is deprived of a truth.

I understand, he said, the willing loss of a resolution; the meaning of the unwilling I have yet to learn.

Why, I said, do you not see that men are unwillingly deprived of good, and willingly of evil? Is not to have lost the truth an evil, and to possess the truth a good? and you would agree that to conceive things as they are is to possess the truth?

Yes, he replied; I agree with you in thinking that mankind are deprived of truth against their will.

And is not this involuntary deprivation caused either by theft, or force, or enchantment?

Still, he replied, I do not understand you.
I fear that I must have been talking darkly, like the tragedians. I only mean that some men are changed by persuasion and that others forget; argument steals away the hearts of one class, and time of the other; and this I call theft. Now you understand me?

Yes.
Those again who are forced are those whom the violence of some pain or grief compels to change their opinion.

I understand, he said, and you are quite right.
And you would also acknowledge that the enchanted are those who change their minds either under the softer influence of pleasure, or the sterner influence of fear?

Yes, he said; everything that deceives may be said to enchant.
Therefore, as I was just now saying, we must enquire who are the best guardians of their own conviction that what they think the interest of the State is to be the rule of their lives. We must watch them from their youth upwards, and make them perform actions in which they are most likely to forget or to be deceived, and he who remembers and is not deceived is to be selected, and he who falls in the trial is to be rejected. That will be the way?

Yes.
And there should also be toils and pains and conflicts prescribed for them, in which they will be made to give further proof of the same qualities.

Very right, he replied.
And then, I said, we must try them with enchantments that is the third sort of test --and see what will be their behaviour: like those who take colts amid noise and tumult to see if they are of a timid nature, so must we take our youth amid terrors of some kind, and again pass them into pleasures, and prove them more thoroughly than gold is proved in the furnace, that we may discover whether they are armed against all enchantments, and of a noble bearing always, good guardians of themselves and of the music which they have learned, and retaining under all circumstances a rhythmical and harmonious nature, such as will be most serviceable to the individual and to the State. And he who at every age, as boy and youth and in mature life, has come out of the trial victorious and pure, shall be appointed a ruler and guardian of the State; he shall be honoured in life and death, and shall receive sepulture and other memorials of honour, the greatest that we have to give. But him who fails, we must reject. I am inclined to think that this is the sort of way in which our rulers and guardians should be chosen and appointed. I speak generally, and not with any pretension to exactness.

And, speaking generally, I agree with you, he said.
And perhaps the word 'guardian' in the fullest sense ought to be applied to this higher class only who preserve us against foreign enemies and maintain peace among our citizens at home, that the one may not have the will, or the others the power, to harm us. The young men whom we before called guardians may be more properly designated auxiliaries and supporters of the principles of the rulers.

I agree with you, he said.
How then may we devise one of those needful falsehoods of which we lately spoke --just one royal lie which may deceive the rulers, if that be possible, and at any rate the rest of the city?

What sort of lie? he said.
Nothing new, I replied; only an old Phoenician tale of what has often occurred before now in other places, (as the poets say, and have made the world believe,) though not in our time, and I do not know whether such an event could ever happen again, or could now even be made probable, if it did.

How your words seem to hesitate on your lips!
You will not wonder, I replied, at my hesitation when you have heard.

Speak, he said, and fear not.
Well then, I will speak, although I really know not how to look you in the face, or in what words to utter the audacious fiction, which I propose to communicate gradually, first to the rulers, then to the soldiers, and lastly to the people. They are to be told that their youth was a dream, and the education and training which they received from us, an appearance only; in reality during all that time they were being formed and fed in the womb of the earth, where they themselves and their arms and appurtenances were manufactured; when they were completed, the earth, their mother, sent them up; and so, their country being their mother and also their nurse, they are bound to advise for her good, and to defend her against attacks, and her citizens they are to regard as children of the earth and their own brothers.

You had good reason, he said, to be ashamed of the lie which you were going to tell.

True, I replied, but there is more coming; I have only told you half. Citizens, we shall say to them in our tale, you are brothers, yet God has framed you differently. Some of you have the power of command, and in the composition of these he has mingled gold, wherefore also they have the greatest honour;