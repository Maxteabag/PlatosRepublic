\chapter{Book IX}

Socrates - ADEIMANTUS

Last of all comes the tyrannical man; about whom we have once more to ask, how is he formed out of the democratical? and how does he live, in happiness or in misery?

Yes, he said, he is the only one remaining.
There is, however, I said, a previous question which remains unanswered.

What question?
I do not think that we have adequately determined the nature and number of the appetites, and until this is accomplished the enquiry will always be confused.

Well, he said, it is not too late to supply the omission.
Very true, I said; and observe the point which I want to understand: Certain of the unnecessary pleasures and appetites I conceive to be unlawful; every one appears to have them, but in some persons they are controlled by the laws and by reason, and the better desires prevail over them-either they are wholly banished or they become few and weak; while in the case of others they are stronger, and there are more of them.

Which appetites do you mean?
I mean those which are awake when the reasoning and human and ruling power is asleep; then the wild beast within us, gorged with meat or drink, starts up and having shaken off sleep, goes forth to satisfy his desires; and there is no conceivable folly or crime --not excepting incest or any other unnatural union, or parricide, or the eating of forbidden food --which at such a time, when he has parted company with all shame and sense, a man may not be ready to commit.

Most true, he said.
But when a man's pulse is healthy and temperate, and when before going to sleep he has awakened his rational powers, and fed them on noble thoughts and enquiries, collecting himself in meditation; after having first indulged his appetites neither too much nor too little, but just enough to lay them to sleep, and prevent them and their enjoyments and pains from interfering with the higher principle --which he leaves in the solitude of pure abstraction, free to contemplate and aspire to the knowledge of the unknown, whether in past, present, or future: when again he has allayed the passionate element, if he has a quarrel against any one --I say, when, after pacifying the two irrational principles, he rouses up the third, which is reason, before he takes his rest, then, as you know, he attains truth most nearly, and is least likely to be the sport of fantastic and lawless visions.

I quite agree.
In saying this I have been running into a digression; but the point which I desire to note is that in all of us, even in good men, there is a lawless wild-beast nature, which peers out in sleep. Pray, consider whether I am right, and you agree with me.

Yes, I agree.
And now remember the character which we attributed to the democratic man. He was supposed from his youth upwards to have been trained under a miserly parent, who encouraged the saving appetites in him, but discountenanced the unnecessary, which aim only at amusement and ornament?

True.
And then he got into the company of a more refined, licentious sort of people, and taking to all their wanton ways rushed into the opposite extreme from an abhorrence of his father's meanness. At last, being a better man than his corruptors, he was drawn in both directions until he halted midway and led a life, not of vulgar and slavish passion, but of what he deemed moderate indulgence in various pleasures. After this manner the democrat was generated out of the oligarch?

Yes, he said; that was our view of him, and is so still.
And now, I said, years will have passed away, and you must conceive this man, such as he is, to have a son, who is brought up in his father's principles.

I can imagine him.
Then you must further imagine the same thing to happen to the son which has already happened to the father: --he is drawn into a perfectly lawless life, which by his seducers is termed perfect liberty; and his father and friends take part with his moderate desires, and the opposite party assist the opposite ones. As soon as these dire magicians and tyrant-makers find that they are losing their hold on him, they contrive to implant in him a master passion, to be lord over his idle and spendthrift lusts --a sort of monstrous winged drone --that is the only image which will adequately describe him.

Yes, he said, that is the only adequate image of him.
And when his other lusts, amid clouds of incense and perfumes and garlands and wines, and all the pleasures of a dissolute life, now let loose, come buzzing around him, nourishing to the utmost the sting of desire which they implant in his drone-like nature, then at last this lord of the soul, having Madness for the captain of his guard, breaks out into a frenzy: and if he finds in himself any good opinions or appetites in process of formation, and there is in him any sense of shame remaining, to these better principles he puts an end, and casts them forth until he has purged away temperance and brought in madness to the full.

Yes, he said, that is the way in which the tyrannical man is generated.

And is not this the reason why of old love has been called a tyrant?
I should not wonder.
Further, I said, has not a drunken man also the spirit of a tyrant?
He has.
And you know that a man who is deranged and not right in his mind, will fancy that he is able to rule, not only over men, but also over the gods?

That he will.
And the tyrannical man in the true sense of the word comes into being when, either under the influence of nature, or habit, or both, he becomes drunken, lustful, passionate? O my friend, is not that so?

Assuredly.
Such is the man and such is his origin. And next, how does he live?
Suppose, as people facetiously say, you were to tell me.
I imagine, I said, at the next step in his progress, that there will be feasts and carousals and revellings and courtezans, and all that sort of thing; Love is the lord of the house within him, and orders all the concerns of his soul.

That is certain.
Yes; and every day and every night desires grow up many and formidable, and their demands are many.

They are indeed, he said.
His revenues, if he has any, are soon spent.
True.
Then comes debt and the cutting down of his property.
Of course.
When he has nothing left, must not his desires, crowding in the nest like young ravens, be crying aloud for food; and he, goaded on by them, and especially by love himself, who is in a manner the captain of them, is in a frenzy, and would fain discover whom he can defraud or despoil of his property, in order that he may gratify them?

Yes, that is sure to be the case.
He must have money, no matter how, if he is to escape horrid pains and pangs.

He must.
And as in himself there was a succession of pleasures, and the new got the better of the old and took away their rights, so he being younger will claim to have more than his father and his mother, and if he has spent his own share of the property, he will take a slice of theirs.

No doubt he will.
And if his parents will not give way, then he will try first of all to cheat and deceive them.

Very true.
And if he fails, then he will use force and plunder them.
Yes, probably.
And if the old man and woman fight for their own, what then, my friend? Will the creature feel any compunction at tyrannizing over them?

Nay, he said, I should not feel at all comfortable about his parents.

But, O heavens! Adeimantus, on account of some newfangled love of a harlot, who is anything but a necessary connection, can you believe that he would strike the mother who is his ancient friend and necessary to his very existence, and would place her under the authority of the other, when she is brought under the same roof with her; or that, under like circumstances, he would do the same to his withered old father, first and most indispensable of friends, for the sake of some newly found blooming youth who is the reverse of indispensable?

Yes, indeed, he said; I believe that he would.
Truly, then, I said, a tyrannical son is a blessing to his father and mother.

He is indeed, he replied.
He first takes their property, and when that falls, and pleasures are beginning to swarm in the hive of his soul, then he breaks into a house, or steals the garments of some nightly wayfarer; next he proceeds to clear a temple. Meanwhile the old opinions which he had when a child, and which gave judgment about good and evil, are overthrown by those others which have just been emancipated, and are now the bodyguard of love and share his empire. These in his democratic days, when he was still subject to the laws and to his father, were only let loose in the dreams of sleep. But now that he is under the dominion of love, he becomes always and in waking reality what he was then very rarely and in a dream only; he will commit the foulest murder, or eat forbidden food, or be guilty of any other horrid act. Love is his tyrant, and lives lordly in him and lawlessly, and being himself a king, leads him on, as a tyrant leads a State, to the performance of any reckless deed by which he can maintain himself and the rabble of his associates, whether those whom evil communications have brought in from without, or those whom he himself has allowed to break loose within him by reason of a similar evil nature in himself. Have we not here a picture of his way of life?

Yes, indeed, he said.
And if there are only a few of them in the State, the rest of the people are well disposed, they go away and become the bodyguard or mercenary soldiers of some other tyrant who may probably want them for a war; and if there is no war, they stay at home and do many little pieces of mischief in the city.

What sort of mischief?
For example, they are the thieves, burglars, cutpurses, footpads, robbers of temples, man-stealers of the community; or if they are able to speak they turn informers, and bear false witness, and take bribes.

A small catalogue of evils, even if the perpetrators of them are few in number.

Yes, I said; but small and great are comparative terms, and all these things, in the misery and evil which they inflict upon a State, do not come within a thousand miles of the tyrant; when this noxious class and their followers grow numerous and become conscious of their strength, assisted by the infatuation of the people, they choose from among themselves the one who has most of the tyrant in his own soul, and him they create their tyrant.

Yes, he said, and he will be the most fit to be a tyrant.
If the people yield, well and good; but if they resist him, as he began by beating his own father and mother, so now, if he has the power, he beats them, and will keep his dear old fatherland or motherland, as the Cretans say, in subjection to his young retainers whom he has introduced to be their rulers and masters. This is the end of his passions and desires.

Exactly.
When such men are only private individuals and before they get power, this is their character; they associate entirely with their own flatterers or ready tools; or if they want anything from anybody, they in their turn are equally ready to bow down before them: they profess every sort of affection for them; but when they have gained their point they know them no more.

Yes, truly.
They are always either the masters or servants and never the friends of anybody; the tyrant never tastes of true freedom or friendship.

Certainly not.
And may we not rightly call such men treacherous?
No question.
Also they are utterly unjust, if we were right in our notion of justice?

Yes, he said, and we were perfectly right.
Let us then sum up in a word, I said, the character of the worst man: he is the waking reality of what we dreamed.

Most true.
And this is he who being by nature most of a tyrant bears rule, and the longer he lives the more of a tyrant he becomes.

Socrates - GLAUCON

That is certain, said Glaucon, taking his turn to answer.
And will not he who has been shown to be the wickedest, be also the most miserable? and he who has tyrannized longest and most, most continually and truly miserable; although this may not be the opinion of men in general?

Yes, he said, inevitably.
And must not the tyrannical man be like the tyrannical, State, and the democratical man like the democratical State; and the same of the others?

Certainly.
And as State is to State in virtue and happiness, so is man in relation to man?

To be sure.
Then comparing our original city, which was under a king, and the city which is under a tyrant, how do they stand as to virtue?

They are the opposite extremes, he said, for one is the very best and the other is the very worst.

There can be no mistake, I said, as to which is which, and therefore I will at once enquire whether you would arrive at a similar decision about their relative happiness and misery. And here we must not allow ourselves to be panic-stricken at the apparition of the tyrant, who is only a unit and may perhaps have a few retainers about him; but let us go as we ought into every corner of the city and look all about, and then we will give our opinion.

A fair invitation, he replied; and I see, as every one must, that a tyranny is the wretchedest form of government, and the rule of a king the happiest.

And in estimating the men too, may I not fairly make a like request, that I should have a judge whose mind can enter into and see through human nature? He must not be like a child who looks at the outside and is dazzled at the pompous aspect which the tyrannical nature assumes to the beholder, but let him be one who has a clear insight. May I suppose that the judgment is given in the hearing of us all by one who is able to judge, and has dwelt in the same place with him, and been present at his dally life and known him in his family relations, where he may be seen stripped of his tragedy attire, and again in the hour of public danger --he shall tell us about the happiness and misery of the tyrant when compared with other men?

That again, he said, is a very fair proposal.
Shall I assume that we ourselves are able and experienced judges and have before now met with such a person? We shall then have some one who will answer our enquiries.

By all means.
Let me ask you not to forget the parallel of the individual and the State; bearing this in mind, and glancing in turn from one to the other of them, will you tell me their respective conditions?

What do you mean? he asked.
Beginning with the State, I replied, would you say that a city which is governed by a tyrant is free or enslaved?

No city, he said, can be more completely enslaved.
And yet, as you see, there are freemen as well as masters in such a State?

Yes, he said, I see that there are --a few; but the people, speaking generally, and the best of them, are miserably degraded and enslaved.

Then if the man is like the State, I said, must not the same rule prevail? his soul is full of meanness and vulgarity --the best elements in him are enslaved; and there is a small ruling part, which is also the worst and maddest.

Inevitably.
And would you say that the soul of such an one is the soul of a freeman, or of a slave?

He has the soul of a slave, in my opinion.
And the State which is enslaved under a tyrant is utterly incapable of acting voluntarily?

Utterly incapable.
And also the soul which is under a tyrant (I am speaking of the soul taken as a whole) is least capable of doing what she desires; there is a gadfly which goads her, and she is full of trouble and remorse?

Certainly.
And is the city which is under a tyrant rich or poor?
Poor.
And the tyrannical soul must be always poor and insatiable?
True.
And must not such a State and such a man be always full of fear?
Yes, indeed.
Is there any State in which you will find more of lamentation and sorrow and groaning and pain?

Certainly not.
And is there any man in whom you will find more of this sort of misery than in the tyrannical man, who is in a fury of passions and desires?

Impossible.
Reflecting upon these and similar evils, you held the tyrannical State to be the most miserable of States?

And I was right, he said.
Certainly, I said. And when you see the same evils in the tyrannical man, what do you say of him?

I say that he is by far the most miserable of all men.
There, I said, I think that you are beginning to go wrong.
What do you mean?
I do not think that he has as yet reached the utmost extreme of misery.

Then who is more miserable?
One of whom I am about to speak.
Who is that?
He who is of a tyrannical nature, and instead of leading a private life has been cursed with the further misfortune of being a public tyrant.

From what has been said, I gather that you are right.
Yes, I replied, but in this high argument you should be a little more certain, and should not conjecture only; for of all questions, this respecting good and evil is the greatest.

Very true, he said.
Let me then offer you an illustration, which may, I think, throw a light upon this subject.

What is your illustration?
The case of rich individuals in cities who possess many slaves: from them you may form an idea of the tyrant's condition, for they both have slaves; the only difference is that he has more slaves.

Yes, that is the difference.
You know that they live securely and have nothing to apprehend from their servants?

What should they fear?
Nothing. But do you observe the reason of this?
Yes; the reason is, that the whole city is leagued together for the protection of each individual.

Very true, I said. But imagine one of these owners, the master say of some fifty slaves, together with his family and property and slaves, carried off by a god into the wilderness, where there are no freemen to help him --will he not be in an agony of fear lest he and his wife and children should be put to death by his slaves?

Yes, he said, he will be in the utmost fear.
The time has arrived when he will be compelled to flatter divers of his slaves, and make many promises to them of freedom and other things, much against his will --he will have to cajole his own servants.

Yes, he said, that will be the only way of saving himself.
And suppose the same god, who carried him away, to surround him with neighbours who will not suffer one man to be the master of another, and who, if they could catch the offender, would take his life?

His case will be still worse, if you suppose him to be everywhere surrounded and watched by enemies.

And is not this the sort of prison in which the tyrant will be bound --he who being by nature such as we have described, is full of all sorts of fears and lusts? His soul is dainty and greedy, and yet alone, of all men in the city, he is never allowed to go on a journey, or to see the things which other freemen desire to see, but he lives in his hole like a woman hidden in the house, and is jealous of any other citizen who goes into foreign parts and sees anything of interest.

Very true, he said.
And amid evils such as these will not he who is ill-governed in his own person --the tyrannical man, I mean --whom you just now decided to be the most miserable of all --will not he be yet more miserable when, instead of leading a private life, he is constrained by fortune to be a public tyrant? He has to be master of others when he is not master of himself: he is like a diseased or paralytic man who is compelled to pass his life, not in retirement, but fighting and combating with other men.

Yes, he said, the similitude is most exact.
Is not his case utterly miserable? and does not the actual tyrant lead a worse life than he whose life you determined to be the worst?

Certainly.
He who is the real tyrant, whatever men may think, is the real slave, and is obliged to practise the greatest adulation and servility, and to be the flatterer of the vilest of mankind. He has desires which he is utterly unable to satisfy, and has more wants than any one, and is truly poor, if you know how to inspect the whole soul of him: all his life long he is beset with fear and is full of convulsions, and distractions, even as the State which he resembles: and surely the resemblance holds?

Very true, he said.
Moreover, as we were saying before, he grows worse from having power: he becomes and is of necessity more jealous, more faithless, more unjust, more friendless, more impious, than he was at first; he is the purveyor and cherisher of every sort of vice, and the consequence is that he is supremely miserable, and that he makes everybody else as miserable as himself.

No man of any sense will dispute your words.
Come then, I said, and as the general umpire in theatrical contests proclaims the result, do you also decide who in your opinion is first in the scale of happiness, and who second, and in what order the others follow: there are five of them in all --they are the royal, timocratical, oligarchical, democratical, tyrannical.

The decision will be easily given, he replied; they shall be choruses coming on the stage, and I must judge them in the order in which they enter, by the criterion of virtue and vice, happiness and misery.

Need we hire a herald, or shall I announce, that the son of Ariston (the best) has decided that the best and justest is also the happiest, and that this is he who is the most royal man and king over himself; and that the worst and most unjust man is also the most miserable, and that this is he who being the greatest tyrant of himself is also the greatest tyrant of his State?

Make the proclamation yourself, he said.
And shall I add, 'whether seen or unseen by gods and men'?
Let the words be added.
Then this, I said, will be our first proof; and there is another, which may also have some weight.

What is that?
The second proof is derived from the nature of the soul: seeing that the individual soul, like the State, has been divided by us into three principles, the division may, I think, furnish a new demonstration.

Of what nature?
It seems to me that to these three principles three pleasures correspond; also three desires and governing powers.

How do you mean? he said.
There is one principle with which, as we were saying, a man learns, another with which he is angry; the third, having many forms, has no special name, but is denoted by the general term appetitive, from the extraordinary strength and vehemence of the desires of eating and drinking and the other sensual appetites which are the main elements of it; also money-loving, because such desires are generally satisfied by the help of money.

That is true, he said.
If we were to say that the loves and pleasures of this third part were concerned with gain, we should then be able to fall back on a single notion; and might truly and intelligibly describe this part of the soul as loving gain or money.

I agree with you.
Again, is not the passionate element wholly set on ruling and conquering and getting fame?

True.
Suppose we call it the contentious or ambitious --would the term be suitable?

Extremely suitable.
On the other hand, every one sees that the principle of knowledge is wholly directed to the truth, and cares less than either of the others for gain or fame.

Far less.
'Lover of wisdom,' 'lover of knowledge,' are titles which we may fitly apply to that part of the soul?

Certainly.
One principle prevails in the souls of one class of men, another in others, as may happen?

Yes.
Then we may begin by assuming that there are three classes of men --lovers of wisdom, lovers of honour, lovers of gain?

Exactly.
And there are three kinds of pleasure, which are their several objects?

Very true.
Now, if you examine the three classes of men, and ask of them in turn which of their lives is pleasantest, each will be found praising his own and depreciating that of others: the money-maker will contrast the vanity of honour or of learning if they bring no money with the solid advantages of gold and silver?

True, he said.
And the lover of honour --what will be his opinion? Will he not think that the pleasure of riches is vulgar, while the pleasure of learning, if it brings no distinction, is all smoke and nonsense to him?

Very true.
And are we to suppose, I said, that the philosopher sets any value on other pleasures in comparison with the pleasure of knowing the truth, and in that pursuit abiding, ever learning, not so far indeed from the heaven of pleasure? Does he not call the other pleasures necessary, under the idea that if there were no necessity for them, he would rather not have them?

There can be no doubt of that, he replied.
Since, then, the pleasures of each class and the life of each are in dispute, and the question is not which life is more or less honourable, or better or worse, but which is the more pleasant or painless --how shall we know who speaks truly?

I cannot myself tell, he said.
Well, but what ought to be the criterion? Is any better than experience and wisdom and reason?

There cannot be a better, he said.
Then, I said, reflect. Of the three individuals, which has the greatest experience of all the pleasures which we enumerated? Has the lover of gain, in learning the nature of essential truth, greater experience of the pleasure of knowledge than the philosopher has of the pleasure of gain?

The philosopher, he replied, has greatly the advantage; for he has of necessity always known the taste of the other pleasures from his childhood upwards: but the lover of gain in all his experience has not of necessity tasted --or, I should rather say, even had he desired, could hardly have tasted --the sweetness of learning and knowing truth.

Then the lover of wisdom has a great advantage over the lover of gain, for he has a double experience?

Yes, very great.
Again, has he greater experience of the pleasures of honour, or the lover of honour of the pleasures of wisdom?

Nay, he said, all three are honoured in proportion as they attain their object; for the rich man and the brave man and the wise man alike have their crowd of admirers, and as they all receive honour they all have experience of the pleasures of honour; but the delight which is to be found in the knowledge of true being is known to the philosopher only.

His experience, then, will enable him to judge better than any one?
Far better.
And he is the only one who has wisdom as well as experience?
Certainly.
Further, the very faculty which is the instrument of judgment is not possessed by the covetous or ambitious man, but only by the philosopher?

What faculty?
Reason, with whom, as we were saying, the decision ought to rest.
Yes.
And reasoning is peculiarly his instrument?
Certainly.
If wealth and gain were the criterion, then the praise or blame of the lover of gain would surely be the most trustworthy?

Assuredly.
Or if honour or victory or courage, in that case the judgement of the ambitious or pugnacious would be the truest?

Clearly.
But since experience and wisdom and reason are the judges--
The only inference possible, he replied, is that pleasures which are approved by the lover of wisdom and reason are the truest.

And so we arrive at the result, that the pleasure of the intelligent part of the soul is the pleasantest of the three, and that he of us in whom this is the ruling principle has the pleasantest life.

Unquestionably, he said, the wise man speaks with authority when he approves of his own life.

And what does the judge affirm to be the life which is next, and the pleasure which is next?

Clearly that of the soldier and lover of honour; who is nearer to himself than the money-maker.

Last comes the lover of gain?
Very true, he said.
Twice in succession, then, has the just man overthrown the unjust in this conflict; and now comes the third trial, which is dedicated to Olympian Zeus the saviour: a sage whispers in my ear that no pleasure except that of the wise is quite true and pure --all others are a shadow only; and surely this will prove the greatest and most decisive of falls?

Yes, the greatest; but will you explain yourself?
I will work out the subject and you shall answer my questions.
Proceed.
Say, then, is not pleasure opposed to pain?
True.
And there is a neutral state which is neither pleasure nor pain?
There is.
A state which is intermediate, and a sort of repose of the soul about either --that is what you mean?

Yes.
You remember what people say when they are sick?
What do they say?
That after all nothing is pleasanter than health. But then they never knew this to be the greatest of pleasures until they were ill.

Yes, I know, he said.
And when persons are suffering from acute pain, you must. have heard them say that there is nothing pleasanter than to get rid of their pain?

I have.
And there are many other cases of suffering in which the mere rest and cessation of pain, and not any positive enjoyment, is extolled by them as the greatest pleasure?

Yes, he said; at the time they are pleased and well content to be at rest.

Again, when pleasure ceases, that sort of rest or cessation will be painful?

Doubtless, he said.
Then the intermediate state of rest will be pleasure and will also be pain?

So it would seem.
But can that which is neither become both?
I should say not.
And both pleasure and pain are motions of the soul, are they not?
Yes.
But that which is neither was just now shown to be rest and not motion, and in a mean between them?

Yes.
How, then, can we be right in supposing that the absence of pain is pleasure, or that the absence of pleasure is pain?

Impossible.
This then is an appearance only and not a reality; that is tc say, the rest is pleasure at the moment and in comparison of what is painful, and painful in comparison of what is pleasant; but all these representations, when tried by the test of true pleasure, are not real but a sort of imposition?

That is the inference.
Look at the other class of pleasures which have no antecedent pains and you will no longer suppose, as you perhaps may at present, that pleasure is only the cessation of pain, or pain of pleasure.

What are they, he said, and where shall I find them?
There are many of them: take as an example the pleasures, of smell, which are very great and have no antecedent pains; they come in a moment, and when they depart leave no pain behind them.

Most true, he said.
Let us not, then, be induced to believe that pure pleasure is the cessation of pain, or pain of pleasure.

No.
Still, the more numerous and violent pleasures which reach the soul through the body are generally of this sort --they are reliefs of pain.

That is true.
And the anticipations of future pleasures and pains are of a like nature?

Yes.
Shall I give you an illustration of them?
Let me hear.
You would allow, I said, that there is in nature an upper and lower and middle region?

I should.
And if a person were to go from the lower to the middle region, would he not imagine that he is going up; and he who is standing in the middle and sees whence he has come, would imagine that he is already in the upper region, if he has never seen the true upper world?

To be sure, he said; how can he think otherwise?
But if he were taken back again he would imagine, and truly imagine, that he was descending?

No doubt.
All that would arise out of his ignorance of the true upper and middle and lower regions?

Yes.
Then can you wonder that persons who are inexperienced in the truth, as they have wrong ideas about many other things, should also have wrong ideas about pleasure and pain and the intermediate state; so that when they are only being drawn towards the painful they feel pain and think the pain which they experience to be real, and in like manner, when drawn away from pain to the neutral or intermediate state, they firmly believe that they have reached the goal of satiety and pleasure; they, not knowing pleasure, err in contrasting pain with the absence of pain. which is like contrasting black with grey instead of white --can you wonder, I say, at this?

No, indeed; I should be much more disposed to wonder at the opposite.

Look at the matter thus: --Hunger, thirst, and the like, are inanitions of the bodily state?

Yes.
And ignorance and folly are inanitions of the soul?
True.
And food and wisdom are the corresponding satisfactions of either?
Certainly.
And is the satisfaction derived from that which has less or from that which has more existence the truer?

Clearly, from that which has more.
What classes of things have a greater share of pure existence in your judgment --those of which food and drink and condiments and all kinds of sustenance are examples, or the class which contains true opinion and knowledge and mind and all the different kinds of virtue? Put the question in this way: --Which has a more pure being --that which is concerned with the invariable, the immortal, and the true, and is of such a nature, and is found in such natures; or that which is concerned with and found in the variable and mortal, and is itself variable and mortal?

Far purer, he replied, is the being of that which is concerned with the invariable.

And does the essence of the invariable partake of knowledge in the same degree as of essence?

Yes, of knowledge in the same degree.
And of truth in the same degree?
Yes.
And, conversely, that which has less of truth will also have less of essence?

Necessarily.
Then, in general, those kinds of things which are in the service of the body have less of truth and essence than those which are in the service of the soul?

Far less.
And has not the body itself less of truth and essence than the soul?
Yes.
What is filled with more real existence, and actually has a more real existence, is more really filled than that which is filled with less real existence and is less real?

Of course.
And if there be a pleasure in being filled with that which is according to nature, that which is more really filled with more real being will more really and truly enjoy true pleasure; whereas that which participates in less real being will be less truly and surely satisfied, and will participate in an illusory and less real pleasure?

Unquestionably.
Those then who know not wisdom and virtue, and are always busy with gluttony and sensuality, go down and up again as far as the mean; and in this region they move at random throughout life, but they never pass into the true upper world; thither they neither look, nor do they ever find their way, neither are they truly filled with true being, nor do they taste of pure and abiding pleasure. Like cattle, with their eyes always looking down and their heads stooping to the earth, that is, to the dining-table, they fatten and feed and breed, and, in their excessive love of these delights, they kick and butt at one another with horns and hoofs which are made of iron; and they kill one another by reason of their insatiable lust. For they fill themselves with that which is not substantial, and the part of themselves which they fill is also unsubstantial and incontinent.

Verily, Socrates, said Glaucon, you describe the life of the many like an oracle.

Their pleasures are mixed with pains --how can they be otherwise? For they are mere shadows and pictures of the true, and are coloured by contrast, which exaggerates both light and shade, and so they implant in the minds of fools insane desires of themselves; and they are fought about as Stesichorus says that the Greeks fought about the shadow of Helen at Troy in ignorance of the truth.

Something of that sort must inevitably happen.
And must not the like happen with the spirited or passionate element of the soul? Will not the passionate man who carries his passion into action, be in the like case, whether he is envious and ambitious, or violent and contentious, or angry and discontented, if he be seeking to attain honour and victory and the satisfaction of his anger without reason or sense?

Yes, he said, the same will happen with the spirited element also.
Then may we not confidently assert that the lovers of money and honour, when they seek their pleasures under the guidance and in the company of reason and knowledge, and pursue after and win the pleasures which wisdom shows them, will also have the truest pleasures in the highest degree which is attainable to them, inasmuch as they follow truth; and they will have the pleasures which are natural to them, if that which is best for each one is also most natural to him?

Yes, certainly; the best is the most natural.
And when the whole soul follows the philosophical principle, and there is no division, the several parts are just, and do each of them their own business, and enjoy severally the best and truest pleasures of which they are capable?

Exactly.
But when either of the two other principles prevails, it fails in attaining its own pleasure, and compels the rest to pursue after a pleasure which is a shadow only and which is not their own?

True.
And the greater the interval which separates them from philosophy and reason, the more strange and illusive will be the pleasure?

Yes.
And is not that farthest from reason which is at the greatest distance from law and order?

Clearly.
And the lustful and tyrannical desires are, as we saw, at the greatest distance? Yes.

And the royal and orderly desires are nearest?
Yes.
Then the tyrant will live at the greatest distance from true or natural pleasure, and the king at the least?

Certainly.
But if so, the tyrant will live most unpleasantly, and the king most pleasantly?

Inevitably.
Would you know the measure of the interval which separates them?
Will you tell me?
There appear to be three pleasures, one genuine and two spurious: now the transgression of the tyrant reaches a point beyond the spurious; he has run away from the region of law and reason, and taken up his abode with certain slave pleasures which are his satellites, and the measure of his inferiority can only be expressed in a figure.

How do you mean?
I assume, I said, that the tyrant is in the third place from the oligarch; the democrat was in the middle?

Yes.
And if there is truth in what has preceded, he will be wedded to an image of pleasure which is thrice removed as to truth from the pleasure of the oligarch?

He will.
And the oligarch is third from the royal; since we count as one royal and aristocratical?

Yes, he is third.
Then the tyrant is removed from true pleasure by the space of a number which is three times three?

Manifestly.
The shadow then of tyrannical pleasure determined by the number of length will be a plane figure.

Certainly.
And if you raise the power and make the plane a solid, there is no difficulty in seeing how vast is the interval by which the tyrant is parted from the king.

Yes; the arithmetician will easily do the sum.
Or if some person begins at the other end and measures the interval by which the king is parted from the tyrant in truth of pleasure, he will find him, when the multiplication is complete, living 729 times more pleasantly, and the tyrant more painfully by this same interval.

What a wonderful calculation! And how enormous is the distance which separates the just from the unjust in regard to pleasure and pain!

Yet a true calculation, I said, and a number which nearly concerns human life, if human beings are concerned with days and nights and months and years.

Yes, he said, human life is certainly concerned with them.
Then if the good and just man be thus superior in pleasure to the evil and unjust, his superiority will be infinitely greater in propriety of life and in beauty and virtue?

Immeasurably greater.
Well, I said, and now having arrived at this stage of the argument, we may revert to the words which brought us hither: Was not some one saying that injustice was a gain to the perfectly unjust who was reputed to be just?

Yes, that was said.
Now then, having determined the power and quality of justice and injustice, let us have a little conversation with him.

What shall we say to him?
Let us make an image of the soul, that he may have his own words presented before his eyes.

Of what sort?
An ideal image of the soul, like the composite creations of ancient mythology, such as the Chimera or Scylla or Cerberus, and there are many others in which two or more different natures are said to grow into one.

There are said of have been such unions.
Then do you now model the form of a multitudinous, many-headed monster, having a ring of heads of all manner of beasts, tame and wild, which he is able to generate and metamorphose at will.

You suppose marvellous powers in the artist; but, as language is more pliable than wax or any similar substance, let there be such a model as you propose.

Suppose now that you make a second form as of a lion, and a third of a man, the second smaller than the first, and the third smaller than the second.

That, he said, is an easier task; and I have made them as you say.
And now join them, and let the three grow into one.
That has been accomplished.
Next fashion the outside of them into a single image, as of a man, so that he who is not able to look within, and sees only the outer hull, may believe the beast to be a single human creature. I have done so, he said.

And now, to him who maintains that it is profitable for the human creature to be unjust, and unprofitable to be just, let us reply that, if he be right, it is profitable for this creature to feast the multitudinous monster and strengthen the lion and the lion-like qualities, but to starve and weaken the man, who is consequently liable to be dragged about at the mercy of either of the other two; and he is not to attempt to familiarize or harmonize them with one another --he ought rather to suffer them to fight and bite and devour one another.

Certainly, he said; that is what the approver of injustice says.
To him the supporter of justice makes answer that he should ever so speak and act as to give the man within him in some way or other the most complete mastery over the entire human creature.

He should watch over the many-headed monster like a good husbandman, fostering and cultivating the gentle qualities, and preventing the wild ones from growing; he should be making the lion-heart his ally, and in common care of them all should be uniting the several parts with one another and with himself.

Yes, he said, that is quite what the maintainer of justice say.
And so from every point of view, whether of pleasure, honour, or advantage, the approver of justice is right and speaks the truth, and the disapprover is wrong and false and ignorant.

Yes, from every point of view.
Come, now, and let us gently reason with the unjust, who is not intentionally in error. 'Sweet Sir,' we will say to him, what think you of things esteemed noble and ignoble? Is not the noble that which subjects the beast to the man, or rather to the god in man; and the ignoble that which subjects the man to the beast?' He can hardly avoid saying yes --can he now?

Not if he has any regard for my opinion.
But, if he agree so far, we may ask him to answer another question: 'Then how would a man profit if he received gold and silver on the condition that he was to enslave the noblest part of him to the worst? Who can imagine that a man who sold his son or daughter into slavery for money, especially if he sold them into the hands of fierce and evil men, would be the gainer, however large might be the sum which he received? And will any one say that he is not a miserable caitiff who remorselessly sells his own divine being to that which is most godless and detestable? Eriphyle took the necklace as the price of her husband's life, but he is taking a bribe in order to compass a worse ruin.'

Yes, said Glaucon, far worse --I will answer for him.
Has not the intemperate been censured of old, because in him the huge multiform monster is allowed to be too much at large?

Clearly.
And men are blamed for pride and bad temper when the lion and serpent element in them disproportionately grows and gains strength?

Yes.
And luxury and softness are blamed, because they relax and weaken this same creature, and make a coward of him?

Very true.
And is not a man reproached for flattery and meanness who subordinates the spirited animal to the unruly monster, and, for the sake of money, of which he can never have enough, habituates him in the days of his youth to be trampled in the mire, and from being a lion to become a monkey?

True, he said.
And why are mean employments and manual arts a reproach Only because they imply a natural weakness of the higher principle; the individual is unable to control the creatures within him, but has to court them, and his great study is how to flatter them.

Such appears to be the reason.
And therefore, being desirous of placing him under a rule like that of the best, we say that he ought to be the servant of the best, in whom the Divine rules; not, as Thrasymachus supposed, to the injury of the servant, but because every one had better be ruled by divine wisdom dwelling within him; or, if this be impossible, then by an external authority, in order that we may be all, as far as possible, under the same government, friends and equals.

True, he said.
And this is clearly seen to be the intention of the law, which is the ally of the whole city; and is seen also in the authority which we exercise over children, and the refusal to let them be free until we have established in them a principle analogous to the constitution of a state, and by cultivation of this higher element have set up in their hearts a guardian and ruler like our own, and when this is done they may go their ways.

Yes, he said, the purpose of the law is manifest.
From what point of view, then, and on what ground can we say that a man is profited by injustice or intemperance or other baseness, which will make him a worse man, even though he acquire money or power by his wickedness?

From no point of view at all.
What shall he profit, if his injustice be undetected and unpunished? He who is undetected only gets worse, whereas he who is detected and punished has the brutal part of his nature silenced and humanized; the gentler element in him is liberated, and his whole soul is perfected and ennobled by the acquirement of justice and temperance and wisdom, more than the body ever is by receiving gifts of beauty, strength and health, in proportion as the soul is more honourable than the body.

Certainly, he said.
To this nobler purpose the man of understanding will devote the energies of his life. And in the first place, he will honour studies which impress these qualities on his soul and disregard others?

Clearly, he said.
In the next place, he will regulate his bodily habit and training, and so far will he be from yielding to brutal and irrational pleasures, that he will regard even health as quite a secondary matter; his first object will be not that he may be fair or strong or well, unless he is likely thereby to gain temperance, but he will always desire so to attemper the body as to preserve the harmony of the soul?

Certainly he will, if he has true music in him.
And in the acquisition of wealth there is a principle of order and harmony which he will also observe; he will not allow himself to be dazzled by the foolish applause of the world, and heap up riches to his own infinite harm?

Certainly not, he said.
He will look at the city which is within him, and take heed that no disorder occur in it, such as might arise either from superfluity or from want; and upon this principle he will regulate his property and gain or spend according to his means.

Very true.
And, for the same reason, he will gladly accept and enjoy such honours as he deems likely to make him a better man; but those, whether private or public, which are likely to disorder his life, he will avoid?

Then, if that is his motive, he will not be a statesman.
By the dog of Egypt, he will! in the city which 's his own he certainly will, though in the land of his birth perhaps not, unless he have a divine call.

I understand; you mean that he will be a ruler in the city of which we are the founders, and which exists in idea only; for I do not believe that there is such an one anywhere on earth?

In heaven, I replied, there is laid up a pattern of it, methinks, which he who desires may behold, and beholding, may set his own house in order. But whether such an one exists, or ever will exist in fact, is no matter; for he will live after the manner of that city, having nothing to do with any other.

I think so, he said.