\chapter{Book X}

Socrates - GLAUCON

Of he many excellences which I perceive in the order of our State, there is none which upon reflection pleases me better than the rule about poetry.

To what do you refer?
To the rejection of imitative poetry, which certainly ought not to be received; as I see far more clearly now that the parts of the soul have been distinguished.

What do you mean?
Speaking in confidence, for I should not like to have my words repeated to the tragedians and the rest of the imitative tribe --but I do not mind saying to you, that all poetical imitations are ruinous to the understanding of the hearers, and that the knowledge of their true nature is the only antidote to them.

Explain the purport of your remark.
Well, I will tell you, although I have always from my earliest youth had an awe and love of Homer, which even now makes the words falter on my lips, for he is the great captain and teacher of the whole of that charming tragic company; but a man is not to be reverenced more than the truth, and therefore I will speak out.

Very good, he said.
Listen to me then, or rather, answer me.
Put your question.
Can you tell me what imitation is? for I really do not know.
A likely thing, then, that I should know.
Why not? for the duller eye may often see a thing sooner than the keener.

Very true, he said; but in your presence, even if I had any faint notion, I could not muster courage to utter it. Will you enquire yourself?

Well then, shall we begin the enquiry in our usual manner: Whenever a number of individuals have a common name, we assume them to have also a corresponding idea or form. Do you understand me?

I do.
Let us take any common instance; there are beds and tables in the world --plenty of them, are there not?

Yes.
But there are only two ideas or forms of them --one the idea of a bed, the other of a table.

True.
And the maker of either of them makes a bed or he makes a table for our use, in accordance with the idea --that is our way of speaking in this and similar instances --but no artificer makes the ideas themselves: how could he?

Impossible.
And there is another artist, --I should like to know what you would say of him.

Who is he?
One who is the maker of all the works of all other workmen.
What an extraordinary man!
Wait a little, and there will be more reason for your saying so. For this is he who is able to make not only vessels of every kind, but plants and animals, himself and all other things --the earth and heaven, and the things which are in heaven or under the earth; he makes the gods also.

He must be a wizard and no mistake.
Oh! you are incredulous, are you? Do you mean that there is no such maker or creator, or that in one sense there might be a maker of all these things but in another not? Do you see that there is a way in which you could make them all yourself?

What way?
An easy way enough; or rather, there are many ways in which the feat might be quickly and easily accomplished, none quicker than that of turning a mirror round and round --you would soon enough make the sun and the heavens, and the earth and yourself, and other animals and plants, and all the, other things of which we were just now speaking, in the mirror.

Yes, he said; but they would be appearances only.
Very good, I said, you are coming to the point now. And the painter too is, as I conceive, just such another --a creator of appearances, is he not?

Of course.
But then I suppose you will say that what he creates is untrue. And yet there is a sense in which the painter also creates a bed?

Yes, he said, but not a real bed.
And what of the maker of the bed? Were you not saying that he too makes, not the idea which, according to our view, is the essence of the bed, but only a particular bed?

Yes, I did.
Then if he does not make that which exists he cannot make true existence, but only some semblance of existence; and if any one were to say that the work of the maker of the bed, or of any other workman, has real existence, he could hardly be supposed to be speaking the truth.

At any rate, he replied, philosophers would say that he was not speaking the truth.

No wonder, then, that his work too is an indistinct expression of truth.

No wonder.
Suppose now that by the light of the examples just offered we enquire who this imitator is?

If you please.
Well then, here are three beds: one existing in nature, which is made by God, as I think that we may say --for no one else can be the maker?

No.
There is another which is the work of the carpenter?
Yes.
And the work of the painter is a third?
Yes.
Beds, then, are of three kinds, and there are three artists who superintend them: God, the maker of the bed, and the painter?

Yes, there are three of them.
God, whether from choice or from necessity, made one bed in nature and one only; two or more such ideal beds neither ever have been nor ever will be made by God.

Why is that?
Because even if He had made but two, a third would still appear behind them which both of them would have for their idea, and that would be the ideal bed and the two others.

Very true, he said.
God knew this, and He desired to be the real maker of a real bed, not a particular maker of a particular bed, and therefore He created a bed which is essentially and by nature one only.

So we believe.
Shall we, then, speak of Him as the natural author or maker of the bed?

Yes, he replied; inasmuch as by the natural process of creation He is the author of this and of all other things.

And what shall we say of the carpenter --is not he also the maker of the bed?

Yes.
But would you call the painter a creator and maker?
Certainly not.
Yet if he is not the maker, what is he in relation to the bed?
I think, he said, that we may fairly designate him as the imitator of that which the others make.

Good, I said; then you call him who is third in the descent from nature an imitator?

Certainly, he said.
And the tragic poet is an imitator, and therefore, like all other imitators, he is thrice removed from the king and from the truth?

That appears to be so.
Then about the imitator we are agreed. And what about the painter? --I would like to know whether he may be thought to imitate that which originally exists in nature, or only the creations of artists?

The latter.
As they are or as they appear? You have still to determine this.
What do you mean?
I mean, that you may look at a bed from different points of view, obliquely or directly or from any other point of view, and the bed will appear different, but there is no difference in reality. And the same of all things.

Yes, he said, the difference is only apparent.
Now let me ask you another question: Which is the art of painting designed to be --an imitation of things as they are, or as they appear --of appearance or of reality?

Of appearance.
Then the imitator, I said, is a long way off the truth, and can do all things because he lightly touches on a small part of them, and that part an image. For example: A painter will paint a cobbler, carpenter, or any other artist, though he knows nothing of their arts; and, if he is a good artist, he may deceive children or simple persons, when he shows them his picture of a carpenter from a distance, and they will fancy that they are looking at a real carpenter.

Certainly.
And whenever any one informs us that he has found a man knows all the arts, and all things else that anybody knows, and every single thing with a higher degree of accuracy than any other man --whoever tells us this, I think that we can only imagine to be a simple creature who is likely to have been deceived by some wizard or actor whom he met, and whom he thought all-knowing, because he himself was unable to analyse the nature of knowledge and ignorance and imitation.

Most true.
And so, when we hear persons saying that the tragedians, and Homer, who is at their head, know all the arts and all things human, virtue as well as vice, and divine things too, for that the good poet cannot compose well unless he knows his subject, and that he who has not this knowledge can never be a poet, we ought to consider whether here also there may not be a similar illusion. Perhaps they may have come across imitators and been deceived by them; they may not have remembered when they saw their works that these were but imitations thrice removed from the truth, and could easily be made without any knowledge of the truth, because they are appearances only and not realities? Or, after all, they may be in the right, and poets do really know the things about which they seem to the many to speak so well?

The question, he said, should by all means be considered.
Now do you suppose that if a person were able to make the original as well as the image, he would seriously devote himself to the image-making branch? Would he allow imitation to be the ruling principle of his life, as if he had nothing higher in him?

I should say not.
The real artist, who knew what he was imitating, would be interested in realities and not in imitations; and would desire to leave as memorials of himself works many and fair; and, instead of being the author of encomiums, he would prefer to be the theme of them.

Yes, he said, that would be to him a source of much greater honour and profit.

Then, I said, we must put a question to Homer; not about medicine, or any of the arts to which his poems only incidentally refer: we are not going to ask him, or any other poet, whether he has cured patients like Asclepius, or left behind him a school of medicine such as the Asclepiads were, or whether he only talks about medicine and other arts at second hand; but we have a right to know respecting military tactics, politics, education, which are the chiefest and noblest subjects of his poems, and we may fairly ask him about them. 'Friend Homer,' then we say to him, 'if you are only in the second remove from truth in what you say of virtue, and not in the third --not an image maker or imitator --and if you are able to discern what pursuits make men better or worse in private or public life, tell us what State was ever better governed by your help? The good order of Lacedaemon is due to Lycurgus, and many other cities great and small have been similarly benefited by others; but who says that you have been a good legislator to them and have done them any good? Italy and Sicily boast of Charondas, and there is Solon who is renowned among us; but what city has anything to say about you?' Is there any city which he might name?

I think not, said Glaucon; not even the Homerids themselves pretend that he was a legislator.

Well, but is there any war on record which was carried on successfully by him, or aided by his counsels, when he was alive?

There is not.
Or is there any invention of his, applicable to the arts or to human life, such as Thales the Milesian or Anacharsis the Scythian, and other ingenious men have conceived, which is attributed to him?

There is absolutely nothing of the kind.
But, if Homer never did any public service, was he privately a guide or teacher of any? Had he in his lifetime friends who loved to associate with him, and who handed down to posterity an Homeric way of life, such as was established by Pythagoras who was so greatly beloved for his wisdom, and whose followers are to this day quite celebrated for the order which was named after him?

Nothing of the kind is recorded of him. For surely, Socrates, Creophylus, the companion of Homer, that child of flesh, whose name always makes us laugh, might be more justly ridiculed for his stupidity, if, as is said, Homer was greatly neglected by him and others in his own day when he was alive?

Yes, I replied, that is the tradition. But can you imagine, Glaucon, that if Homer had really been able to educate and improve mankind --if he had possessed knowledge and not been a mere imitator --can you imagine, I say, that he would not have had many followers, and been honoured and loved by them? Protagoras of Abdera, and Prodicus of Ceos, and a host of others, have only to whisper to their contemporaries: 'You will never be able to manage either your own house or your own State until you appoint us to be your ministers of education' --and this ingenious device of theirs has such an effect in making them love them that their companions all but carry them about on their shoulders. And is it conceivable that the contemporaries of Homer, or again of Hesiod, would have allowed either of them to go about as rhapsodists, if they had really been able to make mankind virtuous? Would they not have been as unwilling to part with them as with gold, and have compelled them to stay at home with them? Or, if the master would not stay, then the disciples would have followed him about everywhere, until they had got education enough?

Yes, Socrates, that, I think, is quite true.
Then must we not infer that all these poetical individuals, beginning with Homer, are only imitators; they copy images of virtue and the like, but the truth they never reach? The poet is like a painter who, as we have already observed, will make a likeness of a cobbler though he understands nothing of cobbling; and his picture is good enough for those who know no more than he does, and judge only by colours and figures.

Quite so.
In like manner the poet with his words and phrases may be said to lay on the colours of the several arts, himself understanding their nature only enough to imitate them; and other people, who are as ignorant as he is, and judge only from his words, imagine that if he speaks of cobbling, or of military tactics, or of anything else, in metre and harmony and rhythm, he speaks very well --such is the sweet influence which melody and rhythm by nature have. And I think that you must have observed again and again what a poor appearance the tales of poets make when stripped of the colours which music puts upon them, and recited in simple prose.

Yes, he said.
They are like faces which were never really beautiful, but only blooming; and now the bloom of youth has passed away from them?

Exactly.
Here is another point: The imitator or maker of the image knows nothing of true existence; he knows appearances only. Am I not right?

Yes.
Then let us have a clear understanding, and not be satisfied with half an explanation.

Proceed.
Of the painter we say that he will paint reins, and he will paint a bit?

Yes.
And the worker in leather and brass will make them?
Certainly.
But does the painter know the right form of the bit and reins? Nay, hardly even the workers in brass and leather who make them; only the horseman who knows how to use them --he knows their right form.

Most true.
And may we not say the same of all things?
What?
That there are three arts which are concerned with all things: one which uses, another which makes, a third which imitates them?

Yes.
And the excellence or beauty or truth of every structure, animate or inanimate, and of every action of man, is relative to the use for which nature or the artist has intended them.

True.
Then the user of them must have the greatest experience of them, and he must indicate to the maker the good or bad qualities which develop themselves in use; for example, the flute-player will tell the flute-maker which of his flutes is satisfactory to the performer; he will tell him how he ought to make them, and the other will attend to his instructions?

Of course.
The one knows and therefore speaks with authority about the goodness and badness of flutes, while the other, confiding in him, will do what he is told by him?

True.
The instrument is the same, but about the excellence or badness of it the maker will only attain to a correct belief; and this he will gain from him who knows, by talking to him and being compelled to hear what he has to say, whereas the user will have knowledge?

True.
But will the imitator have either? Will he know from use whether or no his drawing is correct or beautiful? Or will he have right opinion from being compelled to associate with another who knows and gives him instructions about what he should draw?

Neither.
Then he will no more have true opinion than he will have knowledge about the goodness or badness of his imitations?

I suppose not.
The imitative artist will be in a brilliant state of intelligence about his own creations?

Nay, very much the reverse.
And still he will go on imitating without knowing what makes a thing good or bad, and may be expected therefore to imitate only that which appears to be good to the ignorant multitude?

Just so.
Thus far then we are pretty well agreed that the imitator has no knowledge worth mentioning of what he imitates. Imitation is only a kind of play or sport, and the tragic poets, whether they write in iambic or in Heroic verse, are imitators in the highest degree?

Very true.
And now tell me, I conjure you, has not imitation been shown by us to be concerned with that which is thrice removed from the truth?

Certainly.
And what is the faculty in man to which imitation is addressed?
What do you mean?
I will explain: The body which is large when seen near, appears small when seen at a distance?

True.
And the same object appears straight when looked at out of the water, and crooked when in the water; and the concave becomes convex, owing to the illusion about colours to which the sight is liable. Thus every sort of confusion is revealed within us; and this is that weakness of the human mind on which the art of conjuring and of deceiving by light and shadow and other ingenious devices imposes, having an effect upon us like magic.

True.
And the arts of measuring and numbering and weighing come to the rescue of the human understanding-there is the beauty of them --and the apparent greater or less, or more or heavier, no longer have the mastery over us, but give way before calculation and measure and weight?

Most true.
And this, surely, must be the work of the calculating and rational principle in the soul

To be sure.
And when this principle measures and certifies that some things are equal, or that some are greater or less than others, there occurs an apparent contradiction?

True.
But were we not saying that such a contradiction is the same faculty cannot have contrary opinions at the same time about the same thing?

Very true.
Then that part of the soul which has an opinion contrary to measure is not the same with that which has an opinion in accordance with measure?

True.
And the better part of the soul is likely to be that which trusts to measure and calculation?

Certainly.
And that which is opposed to them is one of the inferior principles of the soul?

No doubt.
This was the conclusion at which I was seeking to arrive when I said that painting or drawing, and imitation in general, when doing their own proper work, are far removed from truth, and the companions and friends and associates of a principle within us which is equally removed from reason, and that they have no true or healthy aim.

Exactly.
The imitative art is an inferior who marries an inferior, and has inferior offspring.

Very true.
And is this confined to the sight only, or does it extend to the hearing also, relating in fact to what we term poetry?

Probably the same would be true of poetry.
Do not rely, I said, on a probability derived from the analogy of painting; but let us examine further and see whether the faculty with which poetical imitation is concerned is good or bad.

By all means.
We may state the question thus: --Imitation imitates the actions of men, whether voluntary or involuntary, on which, as they imagine, a good or bad result has ensued, and they rejoice or sorrow accordingly. Is there anything more?

No, there is nothing else.
But in all this variety of circumstances is the man at unity with himself --or rather, as in the instance of sight there was confusion and opposition in his opinions about the same things, so here also is there not strife and inconsistency in his life? Though I need hardly raise the question again, for I remember that all this has been already admitted; and the soul has been acknowledged by us to be full of these and ten thousand similar oppositions occurring at the same moment?

And we were right, he said.
Yes, I said, thus far we were right; but there was an omission which must now be supplied.

What was the omission?
Were we not saying that a good man, who has the misfortune to lose his son or anything else which is most dear to him, will bear the loss with more equanimity than another?

Yes.
But will he have no sorrow, or shall we say that although he cannot help sorrowing, he will moderate his sorrow?

The latter, he said, is the truer statement.
Tell me: will he be more likely to struggle and hold out against his sorrow when he is seen by his equals, or when he is alone?

It will make a great difference whether he is seen or not.
When he is by himself he will not mind saying or doing many things which he would be ashamed of any one hearing or seeing him do?

True.
There is a principle of law and reason in him which bids him resist, as well as a feeling of his misfortune which is forcing him to indulge his sorrow?

True.
But when a man is drawn in two opposite directions, to and from the same object, this, as we affirm, necessarily implies two distinct principles in him?

Certainly.
One of them is ready to follow the guidance of the law?
How do you mean?
The law would say that to be patient under suffering is best, and that we should not give way to impatience, as there is no knowing whether such things are good or evil; and nothing is gained by impatience; also, because no human thing is of serious importance, and grief stands in the way of that which at the moment is most required.

What is most required? he asked.
That we should take counsel about what has happened, and when the dice have been thrown order our affairs in the way which reason deems best; not, like children who have had a fall, keeping hold of the part struck and wasting time in setting up a howl, but always accustoming the soul forthwith to apply a remedy, raising up that which is sickly and fallen, banishing the cry of sorrow by the healing art.

Yes, he said, that is the true way of meeting the attacks of fortune.

Yes, I said; and the higher principle is ready to follow this suggestion of reason?

Clearly.
And the other principle, which inclines us to recollection of our troubles and to lamentation, and can never have enough of them, we may call irrational, useless, and cowardly?

Indeed, we may.
And does not the latter --I mean the rebellious principle --furnish a great variety of materials for imitation? Whereas the wise and calm temperament, being always nearly equable, is not easy to imitate or to appreciate when imitated, especially at a public festival when a promiscuous crowd is assembled in a theatre. For the feeling represented is one to which they are strangers.

Certainly.
Then the imitative poet who aims at being popular is not by nature made, nor is his art intended, to please or to affect the principle in the soul; but he will prefer the passionate and fitful temper, which is easily imitated?

Clearly.
And now we may fairly take him and place him by the side of the painter, for he is like him in two ways: first, inasmuch as his creations have an inferior degree of truth --in this, I say, he is like him; and he is also like him in being concerned with an inferior part of the soul; and therefore we shall be right in refusing to admit him into a well-ordered State, because he awakens and nourishes and strengthens the feelings and impairs the reason. As in a city when the evil are permitted to have authority and the good are put out of the way, so in the soul of man, as we maintain, the imitative poet implants an evil constitution, for he indulges the irrational nature which has no discernment of greater and less, but thinks the same thing at one time great and at another small-he is a manufacturer of images and is very far removed from the truth.

Exactly.
But we have not yet brought forward the heaviest count in our accusation: --the power which poetry has of harming even the good (and there are very few who are not harmed), is surely an awful thing?

Yes, certainly, if the effect is what you say.
Hear and judge: The best of us, as I conceive, when we listen to a passage of Homer, or one of the tragedians, in which he represents some pitiful hero who is drawling out his sorrows in a long oration, or weeping, and smiting his breast --the best of us, you know, delight in giving way to sympathy, and are in raptures at the excellence of the poet who stirs our feelings most.

Yes, of course I know.
But when any sorrow of our own happens to us, then you may observe that we pride ourselves on the opposite quality --we would fain be quiet and patient; this is the manly part, and the other which delighted us in the recitation is now deemed to be the part of a woman.

Very true, he said.
Now can we be right in praising and admiring another who is doing that which any one of us would abominate and be ashamed of in his own person?

No, he said, that is certainly not reasonable.
Nay, I said, quite reasonable from one point of view.
What point of view?
If you consider, I said, that when in misfortune we feel a natural hunger and desire to relieve our sorrow by weeping and lamentation, and that this feeling which is kept under control in our own calamities is satisfied and delighted by the poets;-the better nature in each of us, not having been sufficiently trained by reason or habit, allows the sympathetic element to break loose because the sorrow is another's; and the spectator fancies that there can be no disgrace to himself in praising and pitying any one who comes telling him what a good man he is, and making a fuss about his troubles; he thinks that the pleasure is a gain, and why should he be supercilious and lose this and the poem too? Few persons ever reflect, as I should imagine, that from the evil of other men something of evil is communicated to themselves. And so the feeling of sorrow which has gathered strength at the sight of the misfortunes of others is with difficulty repressed in our own.

How very true!
And does not the same hold also of the ridiculous? There are jests which you would be ashamed to make yourself, and yet on the comic stage, or indeed in private, when you hear them, you are greatly amused by them, and are not at all disgusted at their unseemliness; --the case of pity is repeated; --there is a principle in human nature which is disposed to raise a laugh, and this which you once restrained by reason, because you were afraid of being thought a buffoon, is now let out again; and having stimulated the risible faculty at the theatre, you are betrayed unconsciously to yourself into playing the comic poet at home.

Quite true, he said.
And the same may be said of lust and anger and all the other affections, of desire and pain and pleasure, which are held to be inseparable from every action ---in all of them poetry feeds and waters the passions instead of drying them up; she lets them rule, although they ought to be controlled, if mankind are ever to increase in happiness and virtue.

I cannot deny it.
Therefore, Glaucon, I said, whenever you meet with any of the eulogists of Homer declaring that he has been the educator of Hellas, and that he is profitable for education and for the ordering of human things, and that you should take him up again and again and get to know him and regulate your whole life according to him, we may love and honour those who say these things --they are excellent people, as far as their lights extend; and we are ready to acknowledge that Homer is the greatest of poets and first of tragedy writers; but we must remain firm in our conviction that hymns to the gods and praises of famous men are the only poetry which ought to be admitted into our State. For if you go beyond this and allow the honeyed muse to enter, either in epic or lyric verse, not law and the reason of mankind, which by common consent have ever been deemed best, but pleasure and pain will be the rulers in our State.

That is most true, he said.
And now since we have reverted to the subject of poetry, let this our defence serve to show the reasonableness of our former judgment in sending away out of our State an art having the tendencies which we have described; for reason constrained us. But that she may impute to us any harshness or want of politeness, let us tell her that there is an ancient quarrel between philosophy and poetry; of which there are many proofs, such as the saying of 'the yelping hound howling at her lord,' or of one 'mighty in the vain talk of fools,' and 'the mob of sages circumventing Zeus,' and the 'subtle thinkers who are beggars after all'; and there are innumerable other signs of ancient enmity between them. Notwithstanding this, let us assure our sweet friend and the sister arts of imitation that if she will only prove her title to exist in a well-ordered State we shall be delighted to receive her --we are very conscious of her charms; but we may not on that account betray the truth. I dare say, Glaucon, that you are as much charmed by her as I am, especially when she appears in Homer?

Yes, indeed, I am greatly charmed.
Shall I propose, then, that she be allowed to return from exile, but upon this condition only --that she make a defence of herself in lyrical or some other metre?

Certainly.
And we may further grant to those of her defenders who are lovers of poetry and yet not poets the permission to speak in prose on her behalf: let them show not only that she is pleasant but also useful to States and to human life, and we will listen in a kindly spirit; for if this can be proved we shall surely be the gainers --I mean, if there is a use in poetry as well as a delight?

Certainly, he said, we shall the gainers.
If her defence fails, then, my dear friend, like other persons who are enamoured of something, but put a restraint upon themselves when they think their desires are opposed to their interests, so too must we after the manner of lovers give her up, though not without a struggle. We too are inspired by that love of poetry which the education of noble States has implanted in us, and therefore we would have her appear at her best and truest; but so long as she is unable to make good her defence, this argument of ours shall be a charm to us, which we will repeat to ourselves while we listen to her strains; that we may not fall away into the childish love of her which captivates the many. At all events we are well aware that poetry being such as we have described is not to be regarded seriously as attaining to the truth; and he who listens to her, fearing for the safety of the city which is within him, should be on his guard against her seductions and make our words his law.

Yes, he said, I quite agree with you.
Yes, I said, my dear Glaucon, for great is the issue at stake, greater than appears, whether a man is to be good or bad. And what will any one be profited if under the influence of honour or money or power, aye, or under the excitement of poetry, he neglect justice and virtue?

Yes, he said; I have been convinced by the argument, as I believe that any one else would have been.

And yet no mention has been made of the greatest prizes and rewards which await virtue.

What, are there any greater still? If there are, they must be of an inconceivable greatness.

Why, I said, what was ever great in a short time? The whole period of threescore years and ten is surely but a little thing in comparison with eternity?

Say rather 'nothing,' he replied.
And should an immortal being seriously think of this little space rather than of the whole?

Of the whole, certainly. But why do you ask?
Are you not aware, I said, that the soul of man is immortal and imperishable?

He looked at me in astonishment, and said: No, by heaven: And are you really prepared to maintain this?

Yes, I said, I ought to be, and you too --there is no difficulty in proving it.

I see a great difficulty; but I should like to hear you state this argument of which you make so light.

Listen then.
I am attending.
There is a thing which you call good and another which you call evil?

Yes, he replied.
Would you agree with me in thinking that the corrupting and destroying element is the evil, and the saving and improving element the good?

Yes.
And you admit that every thing has a good and also an evil; as ophthalmia is the evil of the eyes and disease of the whole body; as mildew is of corn, and rot of timber, or rust of copper and iron: in everything, or in almost everything, there is an inherent evil and disease?

Yes, he said.
And anything which is infected by any of these evils is made evil, and at last wholly dissolves and dies?

True.
The vice and evil which is inherent in each is the destruction of each; and if this does not destroy them there is nothing else that will; for good certainly will not destroy them, nor again, that which is neither good nor evil.

Certainly not.
If, then, we find any nature which having this inherent corruption cannot be dissolved or destroyed, we may be certain that of such a nature there is no destruction?

That may be assumed.
Well, I said, and is there no evil which corrupts the soul?
Yes, he said, there are all the evils which we were just now passing in review: unrighteousness, intemperance, cowardice, ignorance.

But does any of these dissolve or destroy her? --and here do not let us fall into the error of supposing that the unjust and foolish man, when he is detected, perishes through his own injustice, which is an evil of the soul. Take the analogy of the body: The evil of the body is a disease which wastes and reduces and annihilates the body; and all the things of which we were just now speaking come to annihilation through their own corruption attaching to them and inhering in them and so destroying them. Is not this true?

Yes.
Consider the soul in like manner. Does the injustice or other evil which exists in the soul waste and consume her? Do they by attaching to the soul and inhering in her at last bring her to death, and so separate her from the body ?

Certainly not.
And yet, I said, it is unreasonable to suppose that anything can perish from without through affection of external evil which could not be destroyed from within by a corruption of its own?

It is, he replied.
Consider, I said, Glaucon, that even the badness of food, whether staleness, decomposition, or any other bad quality, when confined to the actual food, is not supposed to destroy the body; although, if the badness of food communicates corruption to the body, then we should say that the body has been destroyed by a corruption of itself, which is disease, brought on by this; but that the body, being one thing, can be destroyed by the badness of food, which is another, and which does not engender any natural infection --this we shall absolutely deny?

Very true.
And, on the same principle, unless some bodily evil can produce an evil of the soul, we must not suppose that the soul, which is one thing, can be dissolved by any merely external evil which belongs to another?

Yes, he said, there is reason in that.
Either then, let us refute this conclusion, or, while it remains unrefuted, let us never say that fever, or any other disease, or the knife put to the throat, or even the cutting up of the whole body into the minutest pieces, can destroy the soul, until she herself is proved to become more unholy or unrighteous in consequence of these things being done to the body; but that the soul, or anything else if not destroyed by an internal evil, can be destroyed by an external one, is not to. be affirmed by any man.

And surely, he replied, no one will ever prove that the souls of men become more unjust in consequence of death.

But if some one who would rather not admit the immortality of the soul boldly denies this, and says that the dying do really become more evil and unrighteous, then, if the speaker is right, I suppose that injustice, like disease, must be assumed to be fatal to the unjust, and that those who take this disorder die by the natural inherent power of destruction which evil has, and which kills them sooner or later, but in quite another way from that in which, at present, the wicked receive death at the hands of others as the penalty of their deeds?

Nay, he said, in that case injustice, if fatal to the unjust, will not be so very terrible to him, for he will be delivered from evil. But I rather suspect the opposite to be the truth, and that injustice which, if it have the power, will murder others, keeps the murderer alive --aye, and well awake too; so far removed is her dwelling-place from being a house of death.

True, I said; if the inherent natural vice or evil of the soul is unable to kill or destroy her, hardly will that which is appointed to be the destruction of some other body, destroy a soul or anything else except that of which it was appointed to be the destruction.

Yes, that can hardly be.
But the soul which cannot be destroyed by an evil, whether inherent or external, must exist for ever, and if existing for ever, must be immortal?

Certainly.
That is the conclusion, I said; and, if a true conclusion, then the souls must always be the same, for if none be destroyed they will not diminish in number. Neither will they increase, for the increase of the immortal natures must come from something mortal, and all things would thus end in immortality.

Very true.
But this we cannot believe --reason will not allow us --any more than we can believe the soul, in her truest nature, to be full of variety and difference and dissimilarity.

What do you mean? he said.
The soul, I said, being, as is now proven, immortal, must be the fairest of compositions and cannot be compounded of many elements?

Certainly not.
Her immortality is demonstrated by the previous argument, and there are many other proofs; but to see her as she really is, not as we now behold her, marred by communion with the body and other miseries, you must contemplate her with the eye of reason, in her original purity; and then her beauty will be revealed, and justice and injustice and all the things which we have described will be manifested more clearly. Thus far, we have spoken the truth concerning her as she appears at present, but we must remember also that we have seen her only in a condition which may be compared to that of the sea-god Glaucus, whose original image can hardly be discerned because his natural members are broken off and crushed and damaged by the waves in all sorts of ways, and incrustations have grown over them of seaweed and shells and stones, so that he is more like some monster than he is to his own natural form. And the soul which we behold is in a similar condition, disfigured by ten thousand ills. But not there, Glaucon, not there must we look.

Where then?
At her love of wisdom. Let us see whom she affects, and what society and converse she seeks in virtue of her near kindred with the immortal and eternal and divine; also how different she would become if wholly following this superior principle, and borne by a divine impulse out of the ocean in which she now is, and disengaged from the stones and shells and things of earth and rock which in wild variety spring up around her because she feeds upon earth, and is overgrown by the good things of this life as they are termed: then you would see her as she is, and know whether she has one shape only or many, or what her nature is. Of her affections and of the forms which she takes in this present life I think that we have now said enough.

True, he replied.
And thus, I said, we have fulfilled the conditions of the argument; we have not introduced the rewards and glories of justice, which, as you were saying, are to be found in Homer and Hesiod; but justice in her own nature has been shown to be best for the soul in her own nature. Let a man do what is just, whether he have the ring of Gyges or not, and even if in addition to the ring of Gyges he put on the helmet of Hades.

Very true.
And now, Glaucon, there will be no harm in further enumerating how many and how great are the rewards which justice and the other virtues procure to the soul from gods and men, both in life and after death.

Certainly not, he said.
Will you repay me, then, what you borrowed in the argument?
What did I borrow?
The assumption that the just man should appear unjust and the unjust just: for you were of opinion that even if the true state of the case could not possibly escape the eyes of gods and men, still this admission ought to be made for the sake of the argument, in order that pure justice might be weighed against pure injustice. Do you remember?

I should be much to blame if I had forgotten.
Then, as the cause is decided, I demand on behalf of justice that the estimation in which she is held by gods and men and which we acknowledge to be her due should now be restored to her by us; since she has been shown to confer reality, and not to deceive those who truly possess her, let what has been taken from her be given back, that so she may win that palm of appearance which is hers also, and which she gives to her own.

The demand, he said, is just.
In the first place, I said --and this is the first thing which you will have to give back --the nature both of the just and unjust is truly known to the gods.

Granted.
And if they are both known to them, one must be the friend and the other the enemy of the gods, as we admitted from the beginning?

True.
And the friend of the gods may be supposed to receive from them all things at their best, excepting only such evil as is the necessary consequence of former sins?

Certainly.
Then this must be our notion of the just man, that even when he is in poverty or sickness, or any other seeming misfortune, all things will in the end work together for good to him in life and death: for the gods have a care of any one whose desire is to become just and to be like God, as far as man can attain the divine likeness, by the pursuit of virtue?

Yes, he said; if he is like God he will surely not be neglected by him.

And of the unjust may not the opposite be supposed?
Certainly.
Such, then, are the palms of victory which the gods give the just?
That is my conviction.
And what do they receive of men? Look at things as they really are, and you will see that the clever unjust are in the case of runners, who run well from the starting-place to the goal but not back again from the goal: they go off at a great pace, but in the end only look foolish, slinking away with their ears draggling on their shoulders, and without a crown; but the true runner comes to the finish and receives the prize and is crowned. And this is the way with the just; he who endures to the end of every action and occasion of his entire life has a good report and carries off the prize which men have to bestow.

True.
And now you must allow me to repeat of the just the blessings which you were attributing to the fortunate unjust. I shall say of them, what you were saying of the others, that as they grow older, they become rulers in their own city if they care to be; they marry whom they like and give in marriage to whom they will; all that you said of the others I now say of these. And, on the other hand, of the unjust I say that the greater number, even though they escape in their youth, are found out at last and look foolish at the end of their course, and when they come to be old and miserable are flouted alike by stranger and citizen; they are beaten and then come those things unfit for ears polite, as you truly term them; they will be racked and have their eyes burned out, as you were saying. And you may suppose that I have repeated the remainder of your tale of horrors. But will you let me assume, without reciting them, that these things are true?

Certainly, he said, what you say is true.
These, then, are the prizes and rewards and gifts which are bestowed upon the just by gods and men in this present life, in addition to the other good things which justice of herself provides.

Yes, he said; and they are fair and lasting.
And yet, I said, all these are as nothing, either in number or greatness in comparison with those other recompenses which await both just and unjust after death. And you ought to hear them, and then both just and unjust will have received from us a full payment of the debt which the argument owes to them.

Speak, he said; there are few things which I would more gladly hear.

Socrates

Well, I said, I will tell you a tale; not one of the tales which Odysseus tells to the hero Alcinous, yet this too is a tale of a hero, Er the son of Armenius, a Pamphylian by birth. He was slain in battle, and ten days afterwards, when the bodies of the dead were taken up already in a state of corruption, his body was found unaffected by decay, and carried away home to be buried. And on the twelfth day, as he was lying on the funeral pile, he returned to life and told them what he had seen in the other world. He said that when his soul left the body he went on a journey with a great company, and that they came to a mysterious place at which there were two openings in the earth; they were near together, and over against them were two other openings in the heaven above. In the intermediate space there were judges seated, who commanded the just, after they had given judgment on them and had bound their sentences in front of them, to ascend by the heavenly way on the right hand; and in like manner the unjust were bidden by them to descend by the lower way on the left hand; these also bore the symbols of their deeds, but fastened on their backs. He drew near, and they told him that he was to be the messenger who would carry the report of the other world to men, and they bade him hear and see all that was to be heard and seen in that place. Then he beheld and saw on one side the souls departing at either opening of heaven and earth when sentence had been given on them; and at the two other openings other souls, some ascending out of the earth dusty and worn with travel, some descending out of heaven clean and bright. And arriving ever and anon they seemed to have come from a long journey, and they went forth with gladness into the meadow, where they encamped as at a festival; and those who knew one another embraced and conversed, the souls which came from earth curiously enquiring about the things above, and the souls which came from heaven about the things beneath. And they told one another of what had happened by the way, those from below weeping and sorrowing at the remembrance of the things which they had endured and seen in their journey beneath the earth (now the journey lasted a thousand years), while those from above were describing heavenly delights and visions of inconceivable beauty. The Story, Glaucon, would take too long to tell; but the sum was this: --He said that for every wrong which they had done to any one they suffered tenfold; or once in a hundred years --such being reckoned to be the length of man's life, and the penalty being thus paid ten times in a thousand years. If, for example, there were any who had been the cause of many deaths, or had betrayed or enslaved cities or armies, or been guilty of any other evil behaviour, for each and all of their offences they received punishment ten times over, and the rewards of beneficence and justice and holiness were in the same proportion. I need hardly repeat what he said concerning young children dying almost as soon as they were born. Of piety and impiety to gods and parents, and of murderers, there were retributions other and greater far which he described. He mentioned that he was present when one of the spirits asked another, 'Where is Ardiaeus the Great?' (Now this Ardiaeus lived a thousand years before the time of Er: he had been the tyrant of some city of Pamphylia, and had murdered his aged father and his elder brother, and was said to have committed many other abominable crimes.) The answer of the other spirit was: 'He comes not hither and will never come. And this,' said he, 'was one of the dreadful sights which we ourselves witnessed. We were at the mouth of the cavern, and, having completed all our experiences, were about to reascend, when of a sudden Ardiaeus appeared and several others, most of whom were tyrants; and there were also besides the tyrants private individuals who had been great criminals: they were just, as they fancied, about to return into the upper world, but the mouth, instead of admitting them, gave a roar, whenever any of these incurable sinners or some one who had not been sufficiently punished tried to ascend; and then wild men of fiery aspect, who were standing by and heard the sound, seized and carried them off; and Ardiaeus and others they bound head and foot and hand, and threw them down and flayed them with scourges, and dragged them along the road at the side, carding them on thorns like wool, and declaring to the passers-by what were their crimes, and that they were being taken away to be cast into hell.' And of all the many terrors which they had endured, he said that there was none like the terror which each of them felt at that moment, lest they should hear the voice; and when there was silence, one by one they ascended with exceeding joy. These, said Er, were the penalties and retributions, and there were blessings as great.

Now when the spirits which were in the meadow had tarried seven days, on the eighth they were obliged to proceed on their journey, and, on the fourth day after, he said that they came to a place where they could see from above a line of light, straight as a column, extending right through the whole heaven and through the earth, in colour resembling the rainbow, only brighter and purer; another day's journey brought them to the place, and there, in the midst of the light, they saw the ends of the chains of heaven let down from above: for this light is the belt of heaven, and holds together the circle of the universe, like the under-girders of a trireme. From these ends is extended the spindle of Necessity, on which all the revolutions turn. The shaft and hook of this spindle are made of steel, and the whorl is made partly of steel and also partly of other materials. Now the whorl is in form like the whorl used on earth; and the description of it implied that there is one large hollow whorl which is quite scooped out, and into this is fitted another lesser one, and another, and another, and four others, making eight in all, like vessels which fit into one another; the whorls show their edges on the upper side, and on their lower side all together form one continuous whorl. This is pierced by the spindle, which is driven home through the centre of the eighth. The first and outermost whorl has the rim broadest, and the seven inner whorls are narrower, in the following proportions --the sixth is next to the first in size, the fourth next to the sixth; then comes the eighth; the seventh is fifth, the fifth is sixth, the third is seventh, last and eighth comes the second. The largest (of fixed stars) is spangled, and the seventh (or sun) is brightest; the eighth (or moon) coloured by the reflected light of the seventh; the second and fifth (Saturn and Mercury) are in colour like one another, and yellower than the preceding; the third (Venus) has the whitest light; the fourth (Mars) is reddish; the sixth (Jupiter) is in whiteness second. Now the whole spindle has the same motion; but, as the whole revolves in one direction, the seven inner circles move slowly in the other, and of these the swiftest is the eighth; next in swiftness are the seventh, sixth, and fifth, which move together; third in swiftness appeared to move according to the law of this reversed motion the fourth; the third appeared fourth and the second fifth. The spindle turns on the knees of Necessity; and on the upper surface of each circle is a siren, who goes round with them, hymning a single tone or note. The eight together form one harmony; and round about, at equal intervals, there is another band, three in number, each sitting upon her throne: these are the Fates, daughters of Necessity, who are clothed in white robes and have chaplets upon their heads, Lachesis and Clotho and Atropos, who accompany with their voices the harmony of the sirens --Lachesis singing of the past, Clotho of the present, Atropos of the future; Clotho from time to time assisting with a touch of her right hand the revolution of the outer circle of the whorl or spindle, and Atropos with her left hand touching and guiding the inner ones, and Lachesis laying hold of either in turn, first with one hand and then with the other.

When Er and the spirits arrived, their duty was to go at once to Lachesis; but first of all there came a prophet who arranged them in order; then he took from the knees of Lachesis lots and samples of lives, and having mounted a high pulpit, spoke as follows: 'Hear the word of Lachesis, the daughter of Necessity. Mortal souls, behold a new cycle of life and mortality. Your genius will not be allotted to you, but you choose your genius; and let him who draws the first lot have the first choice, and the life which he chooses shall be his destiny. Virtue is free, and as a man honours or dishonours her he will have more or less of her; the responsibility is with the chooser --God is justified.' When the Interpreter had thus spoken he scattered lots indifferently among them all, and each of them took up the lot which fell near him, all but Er himself (he was not allowed), and each as he took his lot perceived the number which he had obtained. Then the Interpreter placed on the ground before them the samples of lives; and there were many more lives than the souls present, and they were of all sorts. There were lives of every animal and of man in every condition. And there were tyrannies among them, some lasting out the tyrant's life, others which broke off in the middle and came to an end in poverty and exile and beggary; and there were lives of famous men, some who were famous for their form and beauty as well as for their strength and success in games, or, again, for their birth and the qualities of their ancestors; and some who were the reverse of famous for the opposite qualities. And of women likewise; there was not, however, any definite character them, because the soul, when choosing a new life, must of necessity become different. But there was every other quality, and the all mingled with one another, and also with elements of wealth and poverty, and disease and health; and there were mean states also. And here, my dear Glaucon, is the supreme peril of our human state; and therefore the utmost care should be taken. Let each one of us leave every other kind of knowledge and seek and follow one thing only, if peradventure he may be able to learn and may find some one who will make him able to learn and discern between good and evil, and so to choose always and everywhere the better life as he has opportunity. He should consider the bearing of all these things which have been mentioned severally and collectively upon virtue; he should know what the effect of beauty is when combined with poverty or wealth in a particular soul, and what are the good and evil consequences of noble and humble birth, of private and public station, of strength and weakness, of cleverness and dullness, and of all the soul, and the operation of them when conjoined; he will then look at the nature of the soul, and from the consideration of all these qualities he will be able to determine which is the better and which is the worse; and so he will choose, giving the name of evil to the life which will make his soul more unjust, and good to the life which will make his soul more just; all else he will disregard. For we have seen and know that this is the best choice both in life and after death. A man must take with him into the world below an adamantine faith in truth and right, that there too he may be undazzled by the desire of wealth or the other allurements of evil, lest, coming upon tyrannies and similar villainies, he do irremediable wrongs to others and suffer yet worse himself; but let him know how to choose the mean and avoid the extremes on either side, as far as possible, not only in this life but in all that which is to come. For this is the way of happiness.

And according to the report of the messenger from the other world this was what the prophet said at the time: 'Even for the last comer, if he chooses wisely and will live diligently, there is appointed a happy and not undesirable existence. Let not him who chooses first be careless, and let not the last despair.' And when he had spoken, he who had the first choice came forward and in a moment chose the greatest tyranny; his mind having been darkened by folly and sensuality, he had not thought out the whole matter before he chose, and did not at first sight perceive that he was fated, among other evils, to devour his own children. But when he had time to reflect, and saw what was in the lot, he began to beat his breast and lament over his choice, forgetting the proclamation of the prophet; for, instead of throwing the blame of his misfortune on himself, he accused chance and the gods, and everything rather than himself. Now he was one of those who came from heaven, and in a former life had dwelt in a well-ordered State, but his virtue was a matter of habit only, and he had no philosophy. And it was true of others who were similarly overtaken, that the greater number of them came from heaven and therefore they had never been schooled by trial, whereas the pilgrims who came from earth, having themselves suffered and seen others suffer, were not in a hurry to choose. And owing to this inexperience of theirs, and also because the lot was a chance, many of the souls exchanged a good destiny for an evil or an evil for a good. For if a man had always on his arrival in this world dedicated himself from the first to sound philosophy, and had been moderately fortunate in the number of the lot, he might, as the messenger reported, be happy here, and also his journey to another life and return to this, instead of being rough and underground, would be smooth and heavenly. Most curious, he said, was the spectacle --sad and laughable and strange; for the choice of the souls was in most cases based on their experience of a previous life. There he saw the soul which had once been Orpheus choosing the life of a swan out of enmity to the race of women, hating to be born of a woman because they had been his murderers; he beheld also the soul of Thamyras choosing the life of a nightingale; birds, on the other hand, like the swan and other musicians, wanting to be men. The soul which obtained the twentieth lot chose the life of a lion, and this was the soul of Ajax the son of Telamon, who would not be a man, remembering the injustice which was done him the judgment about the arms. The next was Agamemnon, who took the life of an eagle, because, like Ajax, he hated human nature by reason of his sufferings. About the middle came the lot of Atalanta; she, seeing the great fame of an athlete, was unable to resist the temptation: and after her there followed the soul of Epeus the son of Panopeus passing into the nature of a woman cunning in the arts; and far away among the last who chose, the soul of the jester Thersites was putting on the form of a monkey. There came also the soul of Odysseus having yet to make a choice, and his lot happened to be the last of them all. Now the recollection of former tolls had disenchanted him of ambition, and he went about for a considerable time in search of the life of a private man who had no cares; he had some difficulty in finding this, which was lying about and had been neglected by everybody else; and when he saw it, he said that he would have done the had his lot been first instead of last, and that he was delighted to have it. And not only did men pass into animals, but I must also mention that there were animals tame and wild who changed into one another and into corresponding human natures --the good into the gentle and the evil into the savage, in all sorts of combinations.

All the souls had now chosen their lives, and they went in the order of their choice to Lachesis, who sent with them the genius whom they had severally chosen, to be the guardian of their lives and the fulfiller of the choice: this genius led the souls first to Clotho, and drew them within the revolution of the spindle impelled by her hand, thus ratifying the destiny of each; and then, when they were fastened to this, carried them to Atropos, who spun the threads and made them irreversible, whence without turning round they passed beneath the throne of Necessity; and when they had all passed, they marched on in a scorching heat to the plain of Forgetfulness, which was a barren waste destitute of trees and verdure; and then towards evening they encamped by the river of Unmindfulness, whose water no vessel can hold; of this they were all obliged to drink a certain quantity, and those who were not saved by wisdom drank more than was necessary; and each one as he drank forgot all things. Now after they had gone to rest, about the middle of the night there was a thunderstorm and earthquake, and then in an instant they were driven upwards in all manner of ways to their birth, like stars shooting. He himself was hindered from drinking the water. But in what manner or by what means he returned to the body he could not say; only, in the morning, awaking suddenly, he found himself lying on the pyre.

And thus, Glaucon, the tale has been saved and has not perished, and will save us if we are obedient to the word spoken; and we shall pass safely over the river of Forgetfulness and our soul will not be defiled. Wherefore my counsel is that we hold fast ever to the heavenly way and follow after justice and virtue always, considering that the soul is immortal and able to endure every sort of good and every sort of evil. Thus shall we live dear to one another and to the gods, both while remaining here and when, like conquerors in the games who go round to gather gifts, we receive our reward. And it shall be well with us both in this life and in the pilgrimage of a thousand years which we have been describing.


THE END